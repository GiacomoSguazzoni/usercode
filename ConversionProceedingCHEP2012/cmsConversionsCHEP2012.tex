\documentclass[a4paper]{jpconf}
\usepackage{graphicx}


\usepackage{slashed}
\usepackage{subfigure}
%\usepackage{rotating}
\usepackage{multirow}
\usepackage{amsmath}
%\usepackage{units}
\setkeys{Gin}{width=\linewidth,totalheight=\textheight,keepaspectratio}
\graphicspath{{fig/}}

%
% My Macros
%
\usepackage{graphicx}
%\usepackage{drftcite}
\usepackage{pstricks}
\usepackage[figuresright]{rotating}
%
%
% Macro declarations
%
%---- SLASH
\def\slasha#1{#1\hskip-0.65em /}  %slasha per caratteri piccoli
\def\slashb#1{#1\hskip-1.3em /}   %slashb per quelli grandi
\def\slashc#1{#1\hskip-.4em /}
%
%---- UNITA` DI MISURA
\def \pb        {{\rm \, pb}}
\def \fb        {{\rm \, fb}}
\def \ipb       {{\rm \, pb^{-1}}}
\def \ifb       {{\rm \, fb^{-1}}}
\def \eV        {{\rm \,  eV}}
\def \keV       {{\rm \, keV}}
\def \MeV       {{\rm \, MeV}}
\def \GeV       {{\rm \, GeV}}
\def \TeV       {{\rm \, TeV}}
\def \TeVc      {\TeV/c}
\def \TeVcc     {\TeV/c^2}
\def \GeVc      {\GeV/c}
\def \GeVcc     {\GeV/c^2}
\def \MeVc      {\MeV/c}
\def \MeVcc     {\MeV/c^2}
%
%---- SIMBOLI
\def\ga{\mathrel{\raise.3ex\hbox{$>$\kern-.75em\lower1ex\hbox{$\sim$}}}}
\def\la{\mathrel{\raise.3ex\hbox{$<$\kern-.75em\lower1ex\hbox{$\sim$}}}}
\newcommand {\lesssim}
     {\,\raisebox{-0.6ex}{$\stackrel{\textstyle<}{\textstyle\sim}$}\,}
\newcommand {\gtrsim}
     {\,\raisebox{-0.6ex}{$\stackrel{\textstyle>}{\textstyle\sim}$}\,}
\newcommand{\ckm}{$\checkmark$}
%
%---- MISCELLANEA
%\newcommand {\slashed}[1] { \mbox{\rlap{\hbox{/}} #1 }}
\newcommand {\onehalf}    {\raisebox{0.1ex}{${\frac{1}{2}}$}}
\newcommand {\fivethirds} {\raisebox{0.1ex}{${\frac{5}{3}}$}}
\newcommand {\OR}         {{\tt OR}\,}
\newcommand {\BR}         {{\rm BR}\,}
\newcommand {\rts}        {\sqrt{s}}
\newcommand {\lumi}       {\mathcal{L}}
\newcommand {\Lumi}       {\int\lumi\mathrm{d}t}
\newcommand {\gradi}    {^\circ}
\newcommand {\de}         {\partial}
\newcommand {\um}         {\, \mu \rm m}
\newcommand {\nm}         {\rm \, nm}
\newcommand {\us}         {\, \mu \rm s}
\newcommand {\cm}         {\rm \, cm}
\newcommand {\mm}         {\rm \, mm}
\newcommand {\m}          {\rm \, m}
\newcommand {\km}         {\rm \, km}
\newcommand {\V}          {\rm \, V}
\newcommand {\T}          {\rm \, T}
\newcommand {\kV}         {\rm \, kV}
\newcommand {\kVm}        {\rm \, kV\! / \! m} 
\newcommand {\MVm}        {\rm \, MV\! / \! m} 
\newcommand {\ns}         {\rm \, ns} 
\newcommand {\ps}         {\rm \, ps} 
%
%---- THEORY groups & AOB
\newcommand {\gws}        {\mathrm{SU(2)_L \otimes U(1)_Y}}
\newcommand {\sul}        {\mathrm{SU(2)_L}}
\newcommand {\suc}        {\mathrm{SU(3)_C}}
\newcommand {\ul}         {\mathrm{U(1)_Y}}
\newcommand {\uem}        {\mathrm{U(1)_{em}}}
\newcommand {\sigmabar}   {\overline{\sigma}}
\newcommand {\gmunu}      {g^{\mu \nu}}
\newcommand {\munu}       {{\mu \nu}}
\newcommand {\obra}       {\langle 0 |}
\newcommand {\oket}       {| 0 \rangle}
%
%---- THEORY lepton fields
\newcommand {\LL}         {L^{\alpha}_{\mathrm L}}
\newcommand {\LLd}        {L^{\dagger \alpha}_{\mathrm L}}
\newcommand {\lL}         {\ell^{\alpha}_{\mathrm L}}
\newcommand {\lLd}        {\ell^{\dagger \alpha}_{\mathrm L}}
\newcommand {\ld}         {\ell^{\dagger \alpha}}
\newcommand {\lb}         {\overline{\ell}^{\alpha}}
\newcommand {\lR}         {\ell^{\alpha}_{\mathrm R}}
\newcommand {\lRd}        {\ell^{\dagger \alpha}_{\mathrm R}}
\newcommand {\nuL}        {\nu^{\alpha}_{\mathrm L}}
\newcommand {\nuLb}       {\overline{\nu}^{\alpha}_{\mathrm L}}
\newcommand {\nub}        {\overline{\nu}^{\alpha}}
\newcommand {\lept}       {\ell^\alpha}
\newcommand {\neut}       {\nu^{\alpha}}
\newcommand {\nuLd}       {\nu^{\dagger \alpha}_{\mathrm L}}
\newcommand {\Phid}       {\Phi^\dagger}
%
%---- THEORY quark fields
\newcommand {\up}         {u^{\alpha}}
\newcommand {\ub}         {\overline{u}^{\alpha}}
\newcommand {\down}       {d^{\alpha}}
\newcommand {\db}         {\overline{d}^{\alpha}}
\newcommand {\QL}         {Q^{\alpha}_{\mathrm L}}
\newcommand {\QLd}        {Q^{\dagger \alpha}_{\mathrm L}}
\newcommand {\UL}         {U^{\alpha}_{\mathrm L}}
\newcommand {\ULd}        {U^{\dagger \alpha}_{\mathrm L}}
\newcommand {\UR}         {U^{\alpha}_{\mathrm R}}
\newcommand {\URd}        {U^{\dagger \alpha}_{\mathrm R}}
\newcommand {\DL}         {D^{\alpha}_{\mathrm L}}
\newcommand {\DLd}        {D^{\dagger \alpha}_{\mathrm L}}
\newcommand {\DR}         {D^{\alpha}_{\mathrm R}}
\newcommand {\DRd}        {D^{\dagger \alpha}_{\mathrm R}}
\newcommand {\bfell}      {\ell\kern-0.4em
                           \ell\kern-0.4em
                           \ell\kern-0.4em
                           \ell }
\newcommand {\obfell}     {\overline{\ell}\kern-0.4em
                           \overline{\ell}\kern-0.4em
                           \overline{\ell}\kern-0.4em
                           \overline{\ell}}
\newcommand {\bfH}      {\, {\cal H}\kern-0.5em \kern-0.4em
                           {\cal H}\kern-0.5em \kern-0.4em
                           {\cal H}\kern0.1em }
\newcommand {\obfH}     {\, \overline{\cal H}\kern-0.5em \kern-0.4em 
                           \overline{\cal H}\kern-0.5em \kern-0.4em 
                           \overline{\cal H}\kern0.1em }
%
%---- PARTICELLE
\def \b             {{\mathrm b}}
\def \t             {{\mathrm t}}
\def \charm         {{\mathrm c}}
\def \d             {{\mathrm d}}
\def \u             {{\mathrm u}}
\def \e             {{\mathrm e}}
\def \q             {{\mathrm q}}
\def \g             {{\mathrm g}}
\def \p             {{\mathrm p}}
\def \s             {{\mathrm s}}
\def \n             {{\mathrm n}}
\def \h             {{\mathrm h}}
\def \l             {\ell} 
\def \f             {{\mathrm f}} 
%\def \f             {{f}} 
\def \A             {{\mathrm A}}
\def \B             {{\mathrm B}}
\def \D             {{\mathrm D}}
\def \K             {{\mathrm K}}
\def \X             {{\mathrm X}}
\def \Y             {{\mathrm Y}}
\def \W             {{\mathrm W}}
\def \H             {{\mathrm H}}
\def \Z             {{\mathrm Z}}
\def \S             {{\mathrm S}}
\def \N             {{\mathrm N}}
\def \L             {{\mathrm L}}
\def \R             {{\mathrm R}}
\def \P             {{\mathrm P}}
\def \G             {{\mathrm G}}
%
%---- Higgs
\newcommand {\ho}         {{\h^0}}
\newcommand {\Ho}         {{\H^0}}
\newcommand {\Ao}         {{\A^0}}
\newcommand {\Hpm}        {{\H^\pm}}
\newcommand {\clsb}       {{\mathrm CL_{\rm s+b}}}
\newcommand {\clb}        {{\mathrm CL_{\rm b}}}
%
%---- SUSY
\newcommand {\dm}         {\Delta m}
\newcommand {\dM}         {\Delta M}
\newcommand {\ldm}        {\mbox{``low $\dm$''}}
\newcommand {\hdm}        {\mbox{``high $\dm$''}}
\newcommand {\nnc}        {{\overline{\mathrm N}_{95}}}
\newcommand {\snc}        {{\overline{\sigma}_{95}}}
\newcommand {\susy}       {{supersymmetry}}
\newcommand {\susyc}      {{supersymmetric}}
\newcommand {\aj}         {\mbox{\sf AJ}}
\newcommand {\ajl}        {\mbox{\sf AJL}}
\newcommand {\llh}        {\mbox{\sf LLH}}
%
%---- SPARTICELLE
\newcommand {\rpc}     {{\rm RPC}}
\newcommand {\rpv}     {{\rm RPV}}
\newcommand {\sfe}     {{\tilde{\f}}}
\newcommand {\sfL}     {{\tilde{\f}_{\mathrm L}}}
\newcommand {\sfR}     {{\tilde{\f}_{\mathrm R}}}
\newcommand {\sfone}   {{\tilde{\f}_{1}}}
\newcommand {\sftwo}   {{\tilde{\f}_{2}}}
\newcommand {\sneu}    {{\tilde{\nu}}}
\newcommand {\wino}    {{\mathrm{\widetilde{W}}}}
\newcommand {\bino}    {{\mathrm{\widetilde{B}}}}
\newcommand {\se}      {{\mathrm{\tilde{e}}}}
\newcommand {\seR}     {{\mathrm{\tilde{e}_{R}}}}
\newcommand {\seL}     {{\mathrm{\tilde{e}_{L}}}}
\newcommand {\st}      {{\mathrm{\tilde{\tau}}}}
\newcommand {\stR}     {{\mathrm{\tilde{\tau}_{R}}}}
\newcommand {\stL}     {{\mathrm{\tilde{\tau}_{L}}}}
\newcommand {\stone}   {{\mathrm{\tilde{\tau}_{1}}}}
\newcommand {\sttwo}   {{\mathrm{\tilde{\tau}_{2}}}}
\newcommand {\sm}      {{\mathrm{\tilde{\mu}}}}
\newcommand {\smR}     {{\mathrm{\tilde{\mu}_{R}}}}
\newcommand {\smL}     {{\mathrm{\tilde{\mu}_{L}}}}
\newcommand {\Sup}     {{\mathrm{\tilde{u}}}}
\newcommand {\suR}     {{\mathrm{\tilde{u}_{R}}}}
\newcommand {\suL}     {{\mathrm{\tilde{u}_{L}}}}
\newcommand {\sdo}     {{\mathrm{\tilde{d}}}}
\newcommand {\sdR}     {{\mathrm{\tilde{d}_{R}}}}
\newcommand {\sdL}     {{\mathrm{\tilde{d}_{L}}}}
\newcommand {\sch}     {{\mathrm{\tilde{c}}}}
\newcommand {\scR}     {{\mathrm{\tilde{c}_{R}}}}
\newcommand {\scL}     {{\mathrm{\tilde{c}_{L}}}}
\newcommand {\sst}     {{\mathrm{\tilde{s}}}}
\newcommand {\ssR}     {{\mathrm{\tilde{s}_{R}}}}
\newcommand {\ssL}     {{\mathrm{\tilde{s}_{L}}}}
\newcommand {\stopR}   {{\tilde{\mathrm{t}}_{R}}}
\newcommand {\stopL}   {{\tilde{\mathrm{t}}_{L}}}
\newcommand {\stopone} {{\tilde{\mathrm{t}}_{1}}}
\newcommand {\stoptwo} {{\mathrm{\tilde{t}_{2}}}}
\newcommand {\sto}     {{\tilde{\mathrm{t}}}}
\newcommand {\SQ}      {{\mathrm{\widetilde{Q}}}}
\newcommand {\STO}     {{\mathrm{\widetilde{T}}}}
\newcommand {\glu}     {{\mathrm{\tilde{g}}}}
\newcommand {\sbotR}   {{\mathrm{\tilde{b}_{R}}}}
\newcommand {\sbotL}   {{\mathrm{\tilde{b}_{L}}}}
\newcommand {\sbotone} {{\mathrm{\tilde{b}_{1}}}}
\newcommand {\sbottwo} {{\mathrm{\tilde{b}_{2}}}}
\newcommand {\sbot}    {{\tilde{\mathrm{b}}}}
\newcommand {\squa}    {{\tilde{\mathrm{q}}}}
\newcommand {\squal}   {{\tilde{\mathrm{q}}_{\rm L}}}
\newcommand {\squar}   {{\tilde{\mathrm{q}}_{\rm R}}}
\newcommand {\sqL}     {{\tilde{\mathrm{q}}_{\rm L}}}
\newcommand {\sqR}     {{\tilde{\mathrm{q}}_{\rm R}}}
\newcommand {\snu}     {{\tilde{\nu}}}
\newcommand {\snue}    {{\tilde{\nu}_{\mathrm e}}}
\newcommand {\snum}    {{\tilde{\nu}_{\mu}}}
\newcommand {\snut}    {{\tilde{\nu}_{\tau}}}
\newcommand {\neu}     {{\chi}}
\newcommand {\chap}    {{\chi^+}}
\newcommand {\cham}    {{\chi^-}}
\newcommand {\chapm}   {{\chi^\pm}}

%
%---- SUSY PARAMETRI
\newcommand {\thstop} {\mathrm{\theta_{\tilde{t}}}}
\newcommand {\thsbot} {\mathrm{\theta_{\tilde{b}}}}
\newcommand {\thsqua} {\mathrm{\theta_{\tilde{q}}}}
\newcommand {\Mcha}{M_{\chi^\pm}}
\newcommand {\Mchi}{M_\chi}
\newcommand {\Msnu}{M_{\tilde{\nu}}}
\newcommand {\tanb}{\tan\beta}
%
%---- ABBREVIAZIONI

%
%---- PROCESSI FISICI
\newcommand {\rb}    {{\rm R_{\b}}}
\newcommand {\qq}    {{\q \overline{\q}}}
\newcommand {\bb}    {{\b \overline{\b}}}
\newcommand {\cc}    {{\charm \overline{\charm}}}
\newcommand {\ff}    {{\f \overline{\f}}}
\newcommand {\el}    {{\e ^+}}
\newcommand {\po}    {{\e ^-}}
\newcommand {\ee}    {{\e ^+ \e ^-}}
\newcommand {\fbody} {{\sto \to \b \chi {\rm f \bar{f}'}}}
\newcommand {\gaga}  {\gamma\gamma}
\newcommand {\ggqq}  {\gamma\gamma \rightarrow \q\overline{\q}}
\newcommand {\ggtt}  {\gamma\gamma \rightarrow \tau^{+}\tau^{-}}
\newcommand {\qqg}   {\q\overline{\q}\gamma}
\newcommand {\ttg}   {\tau^{+}\tau^{-}\gamma}
\newcommand {\wenu}  {{\rm We\nu_\e}}
\newcommand {\gsZ}   {\gamma^\star\mathrm{Z}}
\newcommand {\ggh}   {\gamma\gamma\rightarrow{\mathrm{hadrons}}}
\newcommand {\ZZg}   {\mathrm ZZ^{*}/\gamma^{*}}
\newcommand {\ZZ}    {{\mathrm ZZ}}
%
%---- VARIABILI
\newcommand {\zo}      {{z_0}}
\newcommand {\ip}      {{d_0}}
%\newcommand {\thr}     {{T_{\rm thrust}}}
\newcommand {\thr}     {{{\rm thrust}}}
\newcommand {\athr}    {{\hat{\rm a}_{\rm thrust}}}
\newcommand {\ththr}   {{\theta_{\rm thrust}}}
\newcommand {\acol}    {{\Phi_{\rm acol}}}
\newcommand {\acop}    {{\Phi_{\rm acop}}}
\newcommand {\acopt}   {{\Phi_{\rm acop_T}}}
\newcommand {\thpoint} {\theta_{\rm point}}
\newcommand {\thscat}  {\theta_{\rm scat}}
\newcommand {\etwelve} {E_{12\gradi}}
\newcommand {\ethirty} {E_{30\gradi}}
\newcommand {\eiso}[1] {E^{\, \triangleleft 30\gradi}_{#1}}
\newcommand {\phimiss} {{\phi_{\vec{p}_{\rm miss}}}}
\newcommand {\ewedge}  {E(\phi_{\vec{p}_{\rm miss}}\pm 15\gradi)}
%\newcommand {\ewedge}  {{E_{\rm w}}}
\newcommand {\evis}    {E_{\rm vis}}
\newcommand {\etot}    {E_{\rm vis}}
\newcommand {\emis}    {E_{\rm miss}}
\newcommand {\mvis}    {M_{\rm vis}}
\newcommand {\mtot}    {M_{\rm vis}}
\newcommand {\mmis}    {M_{\rm miss}}
\newcommand {\mhad}    {M^{\rm ex \, \ell_1}_{\rm vis}}
\newcommand {\mhadtwo} {M^{\rm ex \, \ell_1\ell_2}_{\rm vis}}
\newcommand {\ehad}    {E^{\rm NH}_{\rm vis}}
\newcommand {\epho}    {E^{\gamma}_{\rm vis}}
\newcommand {\echa}    {E^{\rm ch}_{\rm vis}}
\newcommand {\nch}     {{N_{\rm ch}}}
\newcommand {\elept}   {E_{\rm lept}}
\newcommand {\elepone} {E_{\ell _1}}
\newcommand {\eleptwo} {E_{\ell _2}}
\newcommand {\pvis}    {{\vec{p}_{\rm vis}}}
\newcommand {\pmis}    {{\vec{p}_{\rm miss}}}
\newcommand {\thmiss}  {{\theta_{\pmis}}}
\newcommand {\pt}      {{p_{\rm t}}}
\newcommand {\ptch}    {{p_{\rm t}^{\rm ch}}}
\newcommand {\pch}    {{p^{\rm ch}}}
\newcommand {\pz}      {{p_z}}
\newcommand {\ptnoNH}  {{p_{\rm t}^{\rm ex \, NH}}}
\newcommand {\puds}    {{P_{\rm uds}}}
%
\newcommand {\pmiss}   {{P\!\!\!\,\!/ }}
\newcommand {\emiss}   {{E\!\!\!\,\!/ }}
%
%
% no more of Christian's random capitalization!
% more of mine
\newcommand{\brchal}{\cal{B}($\PCha \rightarrow \ell\nu\PChi\ $)}
\newcommand{\M}{M_{2}}
\newcommand{\Mp}{M_{2}}
\newcommand{\sigbg}{\sigma_{\mathrm{bg}}}
\newcommand{\ww}   {\mathrm {WW}}
\newcommand{\zz}   {\mathrm Z\gamma^{*}}
\newcommand{\ewnu} {\mathrm{eW}\nu}
\newcommand{\eez}  {\mathrm {eeZ}}
\newcommand{\gagall}{{\gamma\gamma\rightarrow \ell\ell }}
\newcommand{\Pstaup}{{\widetilde{\tau}_{1}}}
\newcommand{\Pstaul}{{\widetilde{\tau}_{L}}}
\newcommand{\Pstaur}{{\widetilde{\tau}_{R}}}
\newcommand{\mzero}{m_{0}}
\newcommand{\msnu}{M_{\tilde{\nu}}}
\newcommand{\mcha}{M_{\chi^{\pm}}}
\newcommand{\mchi}{M_{\chi}}
\newcommand{\mstau}{M_{{\widetilde{\tau}_{1}}}}
\newcommand{\atau}{A_{\tau}}
\newcommand{\chsnu}{\PCha \rightarrow \ell \tilde{\nu}}
\newcommand{\chstau}{\PCha \rightarrow \tilde{\tau}_{1}\nu}
\newcommand{\chlep}{\PCha \rightarrow \ell\nu\chi}
\newcommand{\Tcsq}{\mathrm{TeV}/c^2}
% new for thesis
\newcommand{\nobs}{N_{\mathrm{obs}}}
\newcommand{\nlim}{N_{\mathrm{lim}}}
\newcommand{\Brl}{\cal{B}_{\ell}}
\newcommand{\leff} {\mathcal{L}_{\mathrm{eff}}}
\newcommand{\dedx}{{\mathrm{d}}E/{\mathrm{d}}x}
\newcommand{\chtau}{\PCha \rightarrow \tau\nu\chi}
\newcommand{\ssqtw}{\sin^{2}\theta_{\mathrm W}}
%\newcommand{\PSql}{\tilde{\mathrm q}_L}
%\newcommand{\PSqr}{\tilde{\mathrm q}_R}
%\newcommand{\PSq1}{\tilde{\mathrm q}_1}
%\newcommand{\PSq2}{\tilde{\mathrm q}_2}
%\newcommand{\ww}{{\mathrm WW}}
%\newcommand{\zz}{{\mathrm Z\gamma^{*}}}
%\newcommand{\eez}{{\mathrm eeZ}}
\newcommand{\nnz}{{\mathrm \nu\bar{\nu}Z}}
% added by bill
\def \ggll    {\gamma\gamma \rightarrow \ell^{+}{\ell}^{-}}
\def \tautau  {\mathrm \tau^{+}\tau^{-}}
\def \ffg  {f\bar{f}(\gamma)}
\def \lll   {\ell^{+}{\ell}^{-}}
\def \ww   {\mathrm WW}
\def \zz   {\mathrm Z\gamma^{*}}
\def \znn  {\mathrm Z\nu\nu}
\def \zee  {\mathrm Zee}
\def \rts  {\sqrt{s}}
\def \mstop {m_{\tilde{\mathrm{t}}}}
\def \msnu  {m_{\tilde{\nu}}}
\def \elow   {E_{12^{\circ}}}
\def \gev    { \, \mathrm{GeV}/\it{c}^{\mathrm{2}}}
\def \gvm    { \, \mathrm{GeV}/\it{c}}
\def \mx     {M_{\mathrm{eff}}} 
\newcommand{\neutr}{\chi}
%end fabio



%dalla mia pretesi

%\def \X             {\mathrm X} 
%\def \V             {\mathrm V} 
\def \Zcc           {\Z \to \charm \bar{\charm} }
\def \Zbb           {\Z \to \b \bar{\b} }
\def \decDS         {\D^{*+} \to \D^0 \pi^+}
\def \decsDS        {\D^{*+} \to \D^0 \pi^+_s}
\def \deckp         {\D^{0} \to \K^- \pi^+}
\def \deckppp       {\D^{0} \to \K^- \pi^+ \pi^+ \pi^-}
\def \deckpp        {\D^{0} \to \K^- \pi^+ \pi^0}
\def \deckpS        {\D^{0} \to \K^- \pi^+ (\pi^0)}
\def \decskp        {\D^{*+} \to \pi^{+}_{s} \K^- \pi^+}
\def \decskppp      {\D^{*+} \to \pi^{+}_{s} \K^- \pi^+ \pi^+ \pi^-}
\def \decskpp       {\D^{*+} \to \pi^{+}_{s} \K^- \pi^+ \pi^0}
\def \decskpS       {\D^{*+} \to \pi^{+}_{s} \K^- \pi^+ (\pi^0)}
\def \epsc          {\varepsilon_{\charm}}
\def \epsb          {\varepsilon_{\b}}
\def \pctod         {P_{\charm \to \D^*}}
\def \pbtod         {P_{\b \to \D^*}}
%\def \R             {{\mathrm R}}
\def \Gbb           {\Gamma_{\b\bar{\b}}}
\def \Gcc           {\Gamma_{\charm\bar{\charm}}}
\def \Gh            {\Gamma_{\mathrm h}}
%
% End of my macros
%

\def\centeron#1#2{{\setbox0=\hbox{#1}\setbox1=\hbox{#2}\ifdim
\wd1>\wd0\kern.5\wd1\kern-.5\wd0\fi
\copy0\kern-.5\wd0\kern-.5\wd1\copy1\ifdim\wd0>\wd1
\kern.5\wd0\kern-.5\wd1\fi}}
\def\ltap{\;\centeron{\raise.35ex\hbox{$<$}}{\lower.65ex\hbox{$\sim$}}\;}
\def\gtap{\;\centeron{\raise.35ex\hbox{$>$}}{\lower.65ex\hbox{$\sim$}}\;}
\def\gsim{\mathrel{\gtap}}
\def\lsim{\mathrel{\ltap}}


% Definitions
\def \usedlumi {4.51\fbinv}
\def \chic {\chi_{c}}
\def \Chizero {\chi_{c0}}
\def \Chione {\chi_{c1}}
\def \Chitwo {\chi_{c2}}
\def \theratio {N_{\Chitwo}/N_{\Chione}}
\def \eoneetwo {{\varepsilon_{1}/\varepsilon_{2}}} 
\def \JPsi{\rm J/\psi}
\def \cPgg{\gamma}
\def \pt{$\rm p_T$}

\begin{document}
\title{An innovative seeding technique for photon conversion reconstruction at CMS}

\author{$^1$D~Giordano and $^2$G~Sguazzoni}

\address{$^1$CERN, Information Technology Department, Experiment Support Group, Geneva, Switzerland}
\address{$^2$INFN, Firenze, Italy}


\ead{domenico.giordano@cern.ch, giacomo.sguazzoni@cern.ch}

\begin{abstract}
The conversion of photons into electron-positron pairs in the detector material is a nuisance in the event reconstruction of high energy physics experiments, since the measurement of the electromagnetic component of interaction products results degraded. Nonetheless this unavoidable detector effect can be also extremely useful. The reconstruction of photon conversions can be used to probe the detector material and to accurately measure soft photons that come from radiative decays in heavy flavor physics. In fact a converted photon can be measured with very high momentum resolution by exploiting the excellent reconstruction of charged tracks of a tracking detector as the one of CMS at LHC. The main issue is that photon conversion tracks are difficult to reconstruct for standard reconstruction algorithms. They are typically soft and very displaced from primary interaction vertex. An innovative seeding technique that exploits the peculiar photon conversion topology, successfully applied in the CMS track reconstruction sequence, is presented. The performances of this technique and the substantial enhancement of photon conversion reconstruction efficiency are discussed. Application examples are given.
\end{abstract}


%\thispagestyle{empty}
\section{Introduction}
\label{introductions}

The precise and efficient determination of charged-particle momenta is a
critical component of the physics program of the LHC~\cite{lhc} experiments, 
as it impacts the ability to
reconstruct leptons, charged hadrons, jets, and photon conversions, which
are the basic physics objects needed to study $pp$ collisions.
Achieving the necessary momentum resolution requires precise tracking in a high magnetic field. This has been obtained by ATLAS~\cite{altas} and CMS~\cite{cms} -- the two LHC general purpose experiments  --  adopting a design with the inner tracking systems in a solenoidal magnetic field, followed by electromagnetic and hadronic calorimeters, and finally by muon detectors. 

At LHC 
%the inner detectors operates in a hostile environment: 
the large number of tracks in each bunch crossing, which result from several proton-proton interactions, produce many hits in the tracking detectors, making track reconstruction difficult. 
In order to resolve adjacent tracks, the hit occupancy is kept low at the cost of using highly granular sensors and, consequently, a high number of electronic channels (millions) for their front-end readout. These channels need to be powered, controlled, read-out through  complex systems of cables and optical fibers. Moreover they have to dissipate a considerable amount of heat, hence a capillary (permeating??) cooling system is also required.
The consequent amount of material  in the tracking volume (contained in the detectors and support structures, the associated electronics, and the power-supply and cooling services) is not negligible and has effect on the tracking performance because of multiple scattering, energy loss and electron bremsstrahlung.  
The material within the tracking volume affects also the overall
event topology and its reconstruction, because of photon conversions and nuclear interactions that modify the energy flow through the inner detector.

 Nonetheless these unavoidable detector effects can be also turned into extremely useful tool. For instance the reconstruction of photon conversions can be used to probe the detector material and to accurately measure soft photons produced in radiative decays of heavy flavor particles. In fact a converted photon can be measured with very high momentum resolution by exploiting the excellent reconstruction of charged tracks by a tracking detector. 
 
In this paper we will discuss the impact of the photon conversions and their reconstruction on tracking.  We will refer to the CMS detector, even if most of the concepts discussed can be applied also to other detectors. 
In the following sections a description of the CMS detector will be provided, as well as some example of the  problems and bonus due to the photon conversions.
We will introduce the current techniques for the reconstruction of photon conversions and describe an innovative technique that allows to increase their reconstruction efficiency.


\section{The CMS detector}



The Compact Muon Solenoid (CMS)~\cite{JINST}  features an all-silicon tracker, a lead tungstate crystal electromagnetic calorimeter (ECAL), and a brass-scintillator hadronic calorimeter (HCAL), all contained inside a  $3.8\,{\rm T}$  superconducting solenoid.
The strong magnetic field  enables the measurement of charged
particle momenta over more than four orders of magnitude, from less than
$100\MeVc$ to more than $1\TeVc$, by reconstructing their trajectories as they
traverse the CMS inner tracking system.  
The calorimeter towers are projective and finely segmented, with $\rm \Delta\phi \sim \Delta\eta \sim 0.087$ in the central region, allowing precise reconstruction of the $\rm e/\gamma$ position and energy. The pseudorapidity $\eta$ is defined as $\rm  -log[tan(\theta/2)]$, where $\theta$ is the polar angle with respect to the direction of the counter-clockwise beam, and $\phi$ is the azimuthal angle. The iron return yoke of the solenoid is interspersed with gas detectors that are used to identify muons.


The CMS tracker, shown
in~Fig.~\ref{fig:tklayout}, consists of 1440 silicon pixel and 15\,148 silicon strip detector modules, covering
the region from $4\cm$ to $110\cm$ in radius, and within $280\cm$ on either
side of the collision point along the LHC beam axis. The tracker
acceptance extends up to a pseudo-rapidity of $\left | \eta \right | < 2.5$.
It  provides an impact parameter resolution of $\sim$\,15~$\mu$m and a transverse momentum ($p_{\rm T}$) resolution of about 1.5\,\% for 100~GeV/$c$ particles.



\begin{figure}[h!]
  \begin{center}
    \includegraphics[width=0.8\textwidth]{fig/general_layout.pdf}
    \caption{Schematic cross section of the CMS Tracker.}
    \label{fig:tklayout}
  \end{center}
\end{figure}

\section{Impact of the photon conversions}
For instance converted photons inside jets can affect the jet energy measurement, if not properly reconstructed. For energetic photon conversions, with both tracks impacting the electromagnetic calorimeter, the effect is mitigated by the algorithms used to evaluate the energy deposition that group the energy deposits produced along the azimuthal coordinate, to keep into account the opening of the $e^+e^-$ tracks in the plane transverse to the magnetic field.  

A different challenge is to reconstruct soft-\pt photon conversions ($\rm \lsim 3 GeV$) some of which do not even reach the electromagnetic calorimeter.

Even if energy of high-\pt converted photons  can be entirely collected by the electromagnetic calorimeter, for many analyses it is essential to be able to reconstruct these objects through the two tracks, to profit of the higher direction resolution of these tracks, in order to well better associate the primary vertex from which the original photon belong.  


Reconstruction of converted photons is hence a crucial step in CMS, and dedicated algorithms have been developed, which make use either of a seeding step which starts from calorimetric cluster [? ] or pairing standard reconstructed tracks [? ].

We use only tracker driven conversion reconstruction since we are dealing with low pT photons (�3 GeV/c) and some of them does not even reach the electromagnetic calorimeter; in this transverse momentum region ECAL seeded conversion reconstruction has a very low reconstruction efficiency, being optimized to reconstruct conversion in a region of pT > 10 GeV/c.



\subsection{Material budget}
A robust method for the Material Budget estimation, exploited by many
past experiments, is based on photon conversions. The idea is that the
material radiography, provided by the position of reconstructed photon
conversion vertices, allows for the visualisation of detector layers
and service structures and that the conversions rate, if properly
accounted, provides an estimate of the amount of material in the
detector volume.


Conversion vertices are reconstructed with an excellent precision:
the angular resolution is about $1{\rm mrad}$ while the radial resolution varies from about~$0.2\cm$ to about~$0.5\cm$, primarily
as a function of pseudo-rapidity.
In Fig.~\ref{fig:convXY} the position of conversion vertices reconstructed in data is shown in the $(x,y)$ plane:
in Fig.~\ref{subfig:convXY_a} the structure at the very centre is the Pixel detector,
surrounded by the shell and rails supporting the Pixel detector, four layers of the Inner Tracker and the first layer of the Outer Tracker.
When restricting the $(x,y)$ view to $\pm 12\cm$, Fig.~\ref{subfig:convXY_c}, the beam pipe is clearly visible, off-centered with respect to
the Pixel detector. More details on the features visible on the ``radiographies'' are given in Sec.~\ref{sec:mat}.

The  $(z, R)$  view of conversion vertices reconstructed in data is finally shown in Fig.~\ref{fig:convRZ};  the less populated
areas  around $|\eta|\sim1.2$, also present in simulation, correspond to transition regions between the Tracker
barrel and endcap sub-components for which the additional track seeding
iterations described in Sec.~\ref{sec:newSeedingSteps} have not been optimized and the conversion reconstruction efficiency is smaller.

\begin{figure}[h!]
  \begin{center}
   \vspace{-0.6cm}
   \subfigure[]{
   \label{subfig:convXY_a}
    \includegraphics[width=0.40\textwidth]{fig/conversions/ptCut/data_xy.png}}
   \subfigure[]{
   \label{subfig:convXY_b}
    \includegraphics[width=0.40\textwidth]{fig/conversions/ptCut/data_xy_zoom.png}}
   \subfigure[]{
   \label{subfig:convXY_c}
    \includegraphics[width=0.40\textwidth]{fig/conversions/ptCut/data_xy_pixel_eta.png}}
    \caption{Conversion vertices in data in the $(x,y)$ plane for $|z|<26\cm$; zoom increases from (a) to (c).}
    \label{fig:convXY}
  \end{center}
\end{figure}

\begin{figure}[h!]
  \begin{center}
     \includegraphics[width=15cm,height=5.5cm]{fig/conversions/ptCut/data_rz.png}
      \caption{Conversion vertices in data the $(z,R)$ plane.}
    \label{fig:convRZ}
  \end{center}
\end{figure}




\begin{figure}[!hbtp]
\centering
%\includegraphics[width=.45\textwidth]{ptMC.png}
\caption{Transverse momentum spectrum of converted photons in minimum
  bias MC events at $\sqrt{s}=900\GeV$.}
\label{ptMC}
\end{figure}


\subsection{Radiative decay of heavy flavor particles}

The measurement is based on the 
reconstruction of the radiative decays to $\JPsi$ plus photon, with the
(relatively low energetic) photons being detected through their 
conversion in electron-positron pairs. 

We select $\Chione$ and $\Chitwo$ candidates by searching for their radiative
decays into the $\JPsi + \cPgg$ final state, with the $\JPsi$ decaying
into two muons. 
Given the small difference between the $\JPsi$ mass, $3096.916\pm0.011\MeVcc$, and the $\Chione$ and $\Chitwo$ masses, $3510.66\pm0.07\MeVcc$ and $3556.20\pm0.09\MeVcc$ respecively~\cite{PDG}, an accurate reconstruction of the photon  
is then needed to finalize the reconstruction of the $\chic$ with
sufficient resolution. In the center of mass of the charmonium states, the photon has an energy of
390\MeV when emitted by the $\Chione$ and of 430\MeV when emitted by the
$\Chitwo$, which results in a $p_T$ of the photon to be measured mostly
between 0.5 and 6\GeVc in the laboratory frame. 


\begin{figure}[t]
    \centering
    \includegraphics[width=0.6\textwidth]{fig/Chic1fb.png}
    \caption{Invariant mass spectrum for $\chi_c$ candidates with $p_T^{\JPsi}$ between 7.0 and 25.0\GeVc.}
    \label{fig:chic}
\end{figure} 


\begin{figure}[t]
    \centering
    \includegraphics[width=0.6\textwidth]{fig/EvtDisplay.png}
    \caption{Event Display}
    \label{fig:evtdisplay}
\end{figure} 




At such low energies (rarely above 
2.5\GeV in the laboratory), the calorimetric
measurements do not have precisions comparable to those obtainable
when the photon energy is measured through the tracking of the 
electron-positron pair originating from a conversion of the photon, in
the beam pipe or in the inner layers of the silicon tracker. 
Furthermore, the calorimeter (``particle flow") photons do not have 
an accurate assignment of the interaction vertex where they come from,
contrary to the converted photons; this vertexing capability is useful to 
reject combinations of $\Upsilon$ dimuons produced in one pp collision 
with photons produced in another (something especially important in the
presence of a large number of pileup collisions). 


to disentangle the two states, whose masses differ by only
45\MeV.

 On the contrary, a measurement of the momentum of the
electron-positron pair originating from a conversion of the photon, in
the beam pipe or in the inner layers of the Tracker, results in a very
accurate measurement of the photon energy. 

\section{difficulties}

 The drawback 
is the reduced yield, caused by the the low efficiency of their reconstruction as pairs of low momentum tracks displaced with respect to the beam axis.




\section{Photon Conversion Reconstruction}
\label{standard}


Up to 70\% of photons traversing the Tracker material converts into  $e^+ e^-$ pairs.


At the LHC many photons are produced from $\pi^0$ decays in minimum bias events; 
as shown in Figure~\ref{ptMC}, the $p_T$ spectrum of such photons is
very soft and the electron and positron produced in the conversion 
do not have enough transverse momentum to reach the CMS electromagnetic calorimeter.
Therefore, conversions need to be reconstructed with a tracker standalone algorithm.


Standard reconstruction of tracks in
the CMS Tracker is seeded by the hits in the
detector~\cite{TRK-10-001}.  


The detection of photon conversions  relies
on the reconstruction of displaced secondary vertices, and the track
reconstruction algorithm described in~\cite{TRK-10-001} has been tuned
to allow reconstruction of tracks originated into the tracking volume,
as far as at a radius of $60\cm$.


In the Minimum Bias events  photons, mainly coming from
$\pi^0$ decays, are expected to have a very soft spectrum. The electron pairs from conversions are very unlikely to
reach the Electromagnetic Calorimeter (ECAL) and the ECAL cluster-driven track and seed finding method~\cite{NOTE2006005, EGM-10-005}
cannot be applied. The development of the iterative tracking described in~\cite{TRK-10-001}
largely extended the capability of reconstructing low-\pt tracks and displaced vertices.
Furthermore, for the work presented in this Analysis Summary, additional seeding steps were introduced as described in Sec.~\ref{sec:newSeedingSteps}
and exploited here to improve the identification of conversions at large radii.
The tracker-only conversion reconstruction was already partially commissioned with limited statistics
during the LHC runs at $\sqrt{s}=900\GeV$ data~\cite{TRK-10-001}.


Photon conversions are characterized by a pair of
oppositely charged secondary tracks, originating from the photon vertex with an
invariant mass consistent with zero,  which are therefore parallel
to each other at production vertex. The electron-positron pair, then,
opens only in the transverse plane because of the solenoidal magnetic field.


Two methods are used in CMS, ECAL-seeded conversion method and combined conversion method. One method only uses the conversion ECAL-seeded tracks and the 	other takes into account conversion track pairs reconstructed from a combination of standard 	tracks, Gaussian sum filter (Gsf) tracks and conversion ECAL-seeded tracks. 

Both methods (ECAL-seeded and combined) 	fit two oppositely charged tracks to a common vertex with the constraint that the two tracks 	are parallel at the vertex, in both the transverse and longitudinal planes. The methods differ 	mainly in the preselection of the track pairs.

%\subsection{Selection and results}                                                                                                                                                                



In this analysis we use the tracker-driven conversion reconstruction,
already described in~\cite{TRK-10-001}, in~\cite{trk10001} and in~\cite{TRK-10-003}. We
summarize the method here. The algorithm relies on the capability of iterative tracking,
discussed in~\cite{TRK-10-001}, to efficiently reconstruct low-\pt and
displaced tracks, as the ones coming from a photon conversion.

Opposite-sign track pairs are firstly required to satisfy basic
quality criteria, i.e. have more than four hits and a normalised
$\chi^2$ less than 10. Then
the tracker-only conversion finding exploits the conversion pair
signature to distinguish genuine pairs from fake pairs.
Tracks are required to have positive charge-signed transverse impact
parameter (the primary vertex lies outside the track trajectory helix)
and the distance of minimum approach in the $xy$ plane, $d_m$, between $-0.25\cm$
and $1\cm$ where $d_m$ is
%defined as $d_{O_1-O_2} - (R_1 - R_2)$ where
%$d_{O_1-O_2}$ is the distance between the centres of the two track
%circles in the transverse plane and $R_1$ and $R_2$ are the two
%circles radii.
the distance between the two points of tangent approach in the
transverse plane for the helices of the two tracks.



Surviving track pairs are then fitted by a 3D-constrained kinematic vertex 
fitter that imposes the
tracks to be parallel in both the transverse and longitudinal planes.
The pair is retained if the fit converges and its $\chi^2$ probability
is greater than $5\times10^{-4}$ (value used in the $\chi_{c2} / \chi_{c1}$ 
cross-section ratio analysis~\cite{bib:AN-11-332}).
%


\section{Tracking Improvement}
\label{singleleg}


The standard CMS tracking is made of six iterative steps, numbered
from 0 to 5, designed to obtain high efficiency and low fake rate for
tracks coming either from the primary vertex or from displaced decay
vertices while maintaining the overall computing time within the
requirements of CMS offline reconstruction centre.
Because of this constraint, the standard implementation is not optimal
to reconstruct with high efficiency tracks from photon conversions
 since the cuts applied in the standard
reconstruction are too tight for these processes. In fact, those
tracks have usually very low momentum and, especially for displaced
vertices at large radii, they do not point  back to the primary
vertex; therefore, they could be reconstructed only with very relaxed
cuts that would result into unacceptably large computing time during
the pattern recognition.

For the purpose of the present study, an additional dedicated
tracking step is added to the track reconstruction sequence.


 is seeded from pairs of hits
in the Pixel barrel and/or in the Strip Tracker Inner barrel
detectors; 

in the Strip Tracker barrel detectors and/or in the Strip Tracker Inner disks.

The seed trajectories are required
%to originate from a region of radius $25\cm$ and longitudinal half length of $0.5\cm$ and are required                                                                                            
to have a minimum transverse momentum of $0.1$ and $0.2\GeVc$ for the
step 6 and 7, respectively. 

To limit  the large number of seeds is
reduced by selecting topologies compatible with a photon conversion
pair: the total charge has to be zero,



Seeds are then
propagated outward, adding compatible hits and updating the trajectory
until either the detector boundary is reached, or no additional
compatible hits can be found.  In the final stage, the collection of
hits is fit to obtain the best estimate of the track parameters.


 
 
The pattern recognition is then performed allowing for at most one
lost hit and requiring at least three hits. Finally the standard track
fit provides the best estimate of the track parameters.

The impact of the additional step to the reconstruction of conversion
and nuclear interaction vertices can be seen on Fig.~\ref{fig:matItTk}
showing the contribution of the different steps to the vertices
reconstruction, as estimated by the simulation.
The additional step increases by more than a factor two the number of conversions
outside the Pixel detector region and contributes significantly to nuclear
interaction vertices at all radii; step 6 mainly helps finding
conversions in the second and third Pixel detector layers.


\begin{figure}[!hbtp]
\centering
\subfigure[]{
\label{subfig:convItTk}
\includegraphics[width=.7\textwidth]{fig/r_algo_BasicCutsJuly12.png}}
\subfigure[]{
\label{subfig:nuclItTk}
\includegraphics[width=.7\textwidth]{fig/r_algo_2010-06-04_NI.png}}
\caption{Fraction of reconstructed vertices as a function of the radius of the vertex
for conversions~\subref{subfig:convItTk} and nuclear
interactions~\subref{subfig:nuclItTk}, for  $|\eta|<1.4$, as estimated
from simulation. The different colors correspond to the largest
iterative step needed to reconstruct the tracks at the vertex.}
\label{fig:matItTk}
\end{figure}

\section{otherseed}

\section{Conclusions}
\label{section_conclusions}


The method here presented provides ...

\section*{References}

\begin{thebibliography}{99}

\bibitem{steve} S.~Wasserbaech, {\em Computeraized tomography of the
    ALEPH detector}, ALEPH Internal Note, {\tt ALEPH 97-08} (1997);
\bibitem{nancy} N.~Marinelli et al, {\em Photon conversion algorithm},
  CMS Analysis note in preparation;
\bibitem{integrals} H.B.~Dwight, {\em Tables of Integrals and Other
    Mathematical Data}, New York and London, The Macmillan Company (1957);\\
  M.~Abramowitz, I.A.~Stegun, {\em Handbook of Mathematical
  Functions with Formulas, Graphs, and Mathematical Tables}, New York,
  Dover Publications (1972), ISBN 978-0-486-61272-0;
%\emph{}

\end{thebibliography}
\end{document}
