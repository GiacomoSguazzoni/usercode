\section{Introduction}

The modules of the CMS Strip Silicon Tracker (SST) consist of one or
two silicon sensors and a front-end
hybrid~\cite{tdr,modules,sensors,hybrid}. The various types of  
silicon sensors have strip pitches ranging from $80$ to $180\um$. The
front-end hybrid, made up of a multi-layer kapton circuit laminated
onto a ceramic support, host four 
or six APVs, the CMS front-end chips, and the ancillary ASICs. The APV
readout channel input pads are arranged into two staggerd rows with a
pitch of $XX\um$. Thus a pitch adapter is needed to accommodate the 
strips' pitch and the readout chips' pitch~\cite{pa}.
 
The production of the $\sim18000$ modules for the SST, given the huge number of
pieces to be assembled and qualified, was shared between many 
laboratories. CERN had the resposibility for three distinct tasks: 1)
assembly of the pitch adapter onto the hybrid carrier using the CERN
gantry, 2) wire bonding of the APV chip inputs to the pitch adapter (PA),
and 3) thorough quality control (QC) test of the completed
assembly. Part of the latter two steps have been performed also in US
labs. Nevertheless the test was done by using a replica of the
test-station developed at CERN. 

The mounting of the glass pitch adapter (PA) onto the ceramic hybrid
carrier is performed by an automatic machine called
gantry~\cite{gantry}. Then the wire bonding between the APV inputs to
the PA follows. After the gluing and bonding steps, a series of QC
tests was made to ensure that the hybrid was fully functioning and
that the PA was successfully bonded. More in detail the aim of these 
tests are:
\begin{itemize}
\item ensure that the hybrid did not suffer any damage that could
  possibly affect its electrical functionality as a consequence of bad
  handling, mechanical stress or accidental electrostatic discharge
  during the gantry PA glueing or the wire bonding;
\item identify and tag all possible single channel defects resulting
  from bonding errors and/or failures; in particular\\
- opens, channels non connected
  to the corresponding PA lines;\\
- shorts, two or more neighboring
  channels electrically shorted;
\item identify and tag all possible single channel defects due to
  microscopic damage of the chips, as may result, for example, from
  accidental electrostatic discharges:\\
- noisy channels, channels with abnormal intrinsinc noise;\\
- dead channels, channels not responding to input signals.
\end{itemize}

The verification of the wire bonding between the PA and the APVs on
the hybrid is important. Opens result in as many unusable 
channels. Shorts result in 'pin-hole' like effect; at
least the corresponding channels are unuseful but there is the
possibility for the entire APV to be affected~\cite{hip}. 

Since the SST will run at a temperature of $-10\Cdegree$ to reduce the
effects of radiation damage. All hybrid components and the wire bonds
will suffer significant thermal-induced stress in passing from
shutdown/manteinance period (warm) to running periods (cold) and viceversa.
To emulate these conditions the setup is capable to perform QC tests
during a thermal cycle within the range of temperatures expected
during the LHC activity. The setup
can continously readout the hybrid during the cycles, to precisely
identify the conditions that could bring to failures.

On a fraction of pieces the thermal cycle is repeted $\sim10$ times
for a realistic simulation of the average number of running-shutdown cycles
expected during the CMS lifetime. Each single module will be liquid cooled,
the coolant liquid will be flowing at a temperature around
$-20\Cdegree$. In normal conditions the hybrid, where a large fraction 
of power is dissipated, will be $6$ to $8\Cdegree$ warmer that the
coolant despite being the module element with the best thermal contact
with the cooling circuit. In case of power failures, however, it will
easily reach the temperature of the coolant. These events are
expected to be rare, but the temperature of $-20\Cdegree$ has been
conservatively choosen to perform qualification tests.

(\fixme Probably could be dropped) Thus, given the large number
of pieces, also the following point can be added to the QC test
objectives:
\begin{itemize}
\item a large statistics (well beyond the limited R\&D scale) cross
  check that the hybrid-plus-PA assembly is able to operate within as
  expected within the CMS running conditions; potential problem related 
  to the operating temperature even at a level of few percent would
  have been discovered and eventually investigated.
\end{itemize}

The described procedures have been carried out at CERN for a large
initial sample of hybrids (about $\sim1000$). Since very few 
failures of pitch adapter gluing and bonding were discovered, the
level of testing was relaxed and a large fraction of the bonding and
testing work has been be given to US \fixme.

This note describes the test setup that has been developed and built
at CERN to test and qualify the front-end hybrids assembly.
This setup allows a full automatic test sequence to be carried out on
four hybrids of any type at the same time in a controlled environment.

The note is organized as follows. A description of the hardware
components of the test setup and of the software needed to run are
given in Sections~\ref{sec:} and~\ref{sec:} respectevely.

As a consequence of the well defined scope of this note, in the
following with {\em hybrid} the hybrid-plus-PA assembly is intended;
the hybrid without the PA will be referred to with {\em bare hybrid}.