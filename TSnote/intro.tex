\section{Introduction}

The CMS microstrip Silicon tracker modules consist of one or two and a
front-end hybrid. The various types of silicon sensors have strip
pitches ranging from 80 and 180. The front-end hybrid, made up of a
multi-layer kapton circuit laminated onto a ceramic support, host four
or six APVs, the CMS front-end chips, and the ancillary ASICs. The APV
readout channel input pads are arranged into two staggerd rows with a
pitch of . Thus a pitch adapter is needed to accommodate the
microstrips' pitch and the readout chips' pitch.
 
The production of the $\sim18000$ modules for the CMS Silicon Tracker
has to be considered on industrial scale, given the huge number of
pieces to be assembled and qualified. This task is shared between many
labs. CERN had the resposibility for the three distinct tasks are: 1)
assembly of the pitch adapter onto the hybrid carrier using the CERN
gantry, 2) wire bonding of the APV chip inputs to the pitch adapter (PA),
and 3) thorough testing of the completed assembly. Part of the latter
two steps have been performed also in US labs. Nevertheless the
testing is done by using a replica of the test-station developed at CERN.

The mounting of the glass pitch adapter (PA) onto the ceramic hybrid
carrier is performed by an automatic machine called gantry []. Then
follows the wire bonding from the APV inputs to the PA. After the
gluing and bonding steps, a series of quality control tests are made
to ensure the continued correct functioning of the hybrid and that the
PA has been successfully bonded. The aim of these tests are:
\begin{itemize}
\item ensure that the hybrid did not suffer any damage that could
  possibly affect its electrical functionality as a consequence of bad
  handling, mechanical stress or accidental electrostatic discharge
  during the gantry picth adapter or the wire bonding;
\item identify and tag all possible single channel defects resulting
  from bonding errors and failure:\\
- open, a channel non connected
  to the corresponding PA line;\\
- short, two or more neighboring
  channels shorted;
\item identify and tag all possible single channel defects due to
  microscopic damage of the chips, as may result, for example, from
  accidental electrostatic discharges:\\
- noisy channel, a channel with abnormal intrinsinc noise;\\
- dead channel, a channel not responding to an input signal.
\end{itemize}

The verification of the wire bonding between the PA and the APVs on
the hybrid is important: opens will result in as many unusable 
channels; shorts could result in 'pin-hole' like effect making at
least the corresponding channels unuseful but possibly the entire APV 
could be affected [some refs]. 

Since the CMS Silicon tracker will run at a
temperature of $-10^\circ$ to reduce the effects of radiation
damage, to improve quality control reliability has been decided to
have a setup capable to perform the described tests during a thermal
cycle with continuous read-out within the range of temperatures
expected during the real data taking to simulate the thermal-induced
stress that all hybrid components and the wire bonds will suffer. On a
fraction of pieces the thermal cycle is repeted $\sim10$ times for a
realistic simulation of the average number of running-shutdown cycles
expected during the CMS lifetime. Each single module will be liquid cooled,
the coolant liquid will be flowing at a temperature around $-20^\circ$
[Ref to TDR]. In normal condition the hybrid, where a large fraction
of power is dissipated, will be $6$ to $8^\circ$ warmer that the
coolant despite being the module element with the best thermal contact
with the cooling circuit, but, in case of power failures, it will
easily reach the temperature of the coolant. These events are
expected to be rare, the temperature of $-20^\circ$ has been
conservatively choosen to perform qualification tests.

Thus, as a matter of fact, also the following point can be added to the PA
assembly electrical quality control test target: 
\begin{itemize}
\item a large statistics cross check going beyond small scale R\&D
  tests or single component tests that the entire hybrid plus pitch
  adapter assembly is able to operate within the expectation at the
  CMS running conditions; potential problem affecting hybrids related
  to the operating temperature even at a level of few percent would
  have been discovered and eventually investigated.
\end{itemize}

The described procedures are planned to be carried out at
CERN for a large initial sample of hybrids (about 1000). If very few
failures of pitch adapter gluing and bonding are discovered, the level
of testing may be relaxed and a large fraction of the bonding and
testing work may be given to one or two collaborating institutes.

This note describes the test setup that has been developed and built
at CERN to test and qualify the front-end hybrids assembly.
This setup allows a full automatic test sequence to be carried out on
four hybrids of any type at the same time in a controlled environment.

The note is organized as follows. A description of the hardware components of the test setup and of the software needed to run are given in Sections \ref{sec:} and \ref{sec:} respectevely.

As a consequence of the well defined subject of this note, in the following with {\em hybrid} is intended the assembly made up of the hybrid and the pitch adapter; the hybrid without the PA will be referred to with {\em bare hybrid}.