The production of the $\sim18000$ modules for the CMS Silicon Tracker
has to be considered on industrial scale, given the huge number of
pieces to be assembled and qualified. This task is shared between many
labs. CERN has the resposibility for the three distinct tasks are: 1)
assembly of the pitch adapter onto the hybrid carrier using the CERN
gantry, 2) wire bonding of the APV chip inputs to the pitch adapter (PA),
and 3) thorough testing of the completed assembly. 

The mounting of the glass pitch adapter (PA) onto the ceramic hybrid
carrier is followed by the wire bonding from the APV25 read-out chip
inputs to the PA. After the gluing and bonding steps, a series of
tests are made to ensure the continued correct functioning of the
hybrid and that the PA has been successfully bonded. 

A crucial part of the test is the verification of the wire bonding
between the PA and the APVs on the hybrid. The test has to be able to
detect opens, that will result in as many unusable channels, as well
as short that could result in 'pin-hole' like effect making at least the
corresponding channels unuseful but also the entire APV could be
affected [some refs].  

This includes a thermal cycle with continuous read-out to the minimal
expected operating temperature to simulate the thermal-induced stress
that all hybrid components and the wire bonds will suffer. On a
fraction of pieces the thermal cycle is repeted $\sim10$ times for a
realistic simulation of the average number of running-shutdown cycles
expected during the CMS lifetime. 

The CMS Silicon tracker will run at a
temperature of $-10^\circ$ to reduce the effects of radiation
damage. To accomplish this the coolant liquid will be flowing at a
temperature around $-20^\circ$ [Ref to TDR]. In normal condition the
Silicon detectors will be at around $$. The hybrid, where the
largest fraction of power is dissipated, is the module element with
the best thermal contact with the cooling pipe. Nevertheless, it will
be $6$ to $8^\circ$ warmer that the coolant except in case of power
failures when it will easily reach the temperature of the
coolant. Despite these events are expected to be rare, this
temperature has been conservatively choosen to perform qualification
tests.

The described procedures are planned to be carried out at
CERN for a large initial sample of hybrids (about 1000). If very few
failures of pitch adapter gluing and bonding are discovered, the level
of testing may be relaxed and a large fraction of the bonding and
testing work may be given to one or two collaborating institutes.

The thermal cycling of large number of hybrids has also the side
beneficial fallout to qualify their behaviour on a consistent
statistics at low temperature. Tests of this type has never been done
on such a big scale. Potential problem affecting hybrids even at a
level of few percent would be discovered and eventually investigated.

This note describes the test setup that has been developed and built
at CERN to test and qualify the front-end hybrids assembly.
This setup allows a full automatic test sequence to be carried out on
four hybrids of any type at the same time in a controlled environment.


