%The 1.6.x version series are the first CMSSW releases in which the
%simulation of TIB/TID passive materials of services is realistic
%although simplified in the description of volumes. Previous version
%rely on prototipe designs of services completely obsolete. 

All TIB/TID services outside the active volumes have been
enclosed in two identical dedicated GEANT4 physical volumes {\tt
  TIBTIDServiceF} and {\tt TIBTIDServicesB} for the forward and the
backward side, respectively. The solid of this physical volumes is a
rather complicated polycone sketched in Figure~\ref{fig:layout}. 

Several daughter physical volumes with appropriate materials are then needed to describe the various regions. For the moment a simplified geometry has been implemented in the simulation. In particular:
\begin{itemize}
\item all volumes within {\tt TIBTIDServices*} are $\phi$ symmetric, although this is not quite true in some case (TST, Margherita and, to a lesser extent Service Cylinder);
\item in the reality TIB/TID backward services layout is a mirror version of the forward layout; this aspect is also lost in the simplified $\phi$ symmetric implementation;
\item services getting out orthogonally with respect to the $z$-axis have, tipically, a density that depends on the radius: a) because they have a constant angular density, resulting in a local density that decreases with increasing radius; b) because of the mechanical structure of the detector itself (i.e. at the interface between TIB and TID at a given radius only services of innermost layers are found); these radial variations of material concentration has been almost completely neglected in 1.6.x;
\end{itemize}

Further improvements on both these aspects that would require a more elaborate radial and $\phi$-wise  granularity for a more accurate description will be evaluated in terms of costs vs. benefits and in case included in next releases. This is especially true for the vertical disk-shaped service support panel build up of several petals, and thus know as  {\em margherita disk}, visible in Figure~\ref{fig:margherita} and occupying the region between 560 mm and 950 mm in radius. The implementation of any $\phi$-wise pattern would imply also to take into account the mirror-like relation between forward and backward services.

On the other hand, to the best of the present knowledge of the assembled detector, all components have been taken into account, at least within a mixture. The overall weight of the simulated passive material within services is pretty close to the actual one. As said, the spatial distribution could be improved to better match the reality.

\subsubsection{TIBTIDServices daugther volumes}
%
\begin{sidewaysfigure*}[p]
\includegraphics*[width=0.99\textwidth]{fig/TIBTIDServicesF_draw.pdf}
\caption{$rz$ cut through one quarter of the TIBTIDServicesF. The black dot
shows the collision point. Letters indicate the various daughter volumes (see text and Table~\ref{tab:TIBTIDServicesF} for the volumes list.}
\label{fig:layout}
\end{sidewaysfigure*}
%
%
\begin{sidewaystable}[p]
  \caption{\textsf {TIBTIDServicesF} volume list. Mass values refer to one half of TIB/TID}
  \label{tab:TIBTIDServicesF}
  \begin{center}
    \begin{tabular}{cccccrr}
      Ref.  & Volume Name  & Solid Name  & Material Name  & Density                  & Mass [kg] & X$_0$ [cm]     \\ 
      \\
 a & TIBFlange             &  TIBFlange        & TIB\_Flange             & 404.32 mg/cm$^3$  &       6.156  &  44.96 \\ 
 b & TIBTIDServiceCylinder & ServiceCylinder   & TIBTID\_ServiceCylinder & 424.86 mg/cm$^3$  &      12.054  &  51.25 \\ 
 c & TIBTIDPxlFlange       & PixelSupportFlange  & T\_FR4                & 1.7 g/cm$^3$      &       0.480  &  12.00 \\ 
 d & TIBTIDOptoPanelFront  & OptoPanelFront    & TIBTID\_OptoPanelFront  & 55.68 mg/cm$^3$   &       1.270  & 688.20 \\ 
 e & TIBTIDOptoConnectors  & OpticalConnectors & TIBTID\_OptoConnectors  & 353.61 mg/cm$^3$  &      11.199  &  84.11 \\ 
 f & TIBTIDOptoPanelBackInner & OptoPanelBackInner & TIBTID\_OptoPanelBackInner & 112.83 mg/cm$^3$ & 2.216  & 279.35 \\ 
 g & TIDSupport            & TIDSupport        & Carbon fibre str.       & 1.69 g/cm$^3$     &       0.537  &  25.00 \\ 
 g & TIDSupportIn          & TIDSupportIn      & Nomex                   & 32 mg/cm$^3$      &       0.031  & $\infty$ \\ 
 h & TIBTIDServiceDisk     & ServiceDisk       & TIBTID\_ServiceDisk     & 475.36 mg/cm$^3$  &       4.667  &  46.75 \\ 
 i & TIBTIDOptoPanelBackOuter & OptoPanelBackOuter & TIBTID\_OptoPanelBackOuter & 133.27 mg/cm$^3$ & 4.155  & 255.07 \\ 
 j & TIBTIDMargherita      & Margherita        & TIBTID\_Margherita      & 1.09393 g/cm$^3$  &      37.278  &  15.91 \\ 
%\multicolumn{5}{r}{\em Total on weighed at TIF} & {\em 80.043} & \\
%\\
 k & TIBTIDPowerConnectors & PowerConnectors   & TIBTID\_PowerConnectors & 354.01 mg/cm$^3$  &      10.483  & 49.61 \\ 
 l & TIBTIDTSTCabRadial    & TSTCabRadial      & TIBTID\_TSTCabRadial    & 303.74 mg/cm$^3$  &      12.213  &  65.12 \\ 
 m & TIBTIDTSTCabAxial     & TSTCabAxial       & TIBTID\_TSTCabAxial     & 893.88 mg/cm$^3$  &     101.151  &  23.34 \\ 
\multicolumn{5}{r}{\em Total } & {\em 203.890} & \\
    \end{tabular}
  \end{center}
\end{sidewaystable}
%

The {\tt TIBTIDServices*} daughter physical volumes together are listed below. A brief explanations of the material  which they are filled with is also given. Further details can be found in Table~\ref{tab:TIBTIDServicesF}. 

\begin{enumerate}
\item {\tt TIBFlange}: the interface between TIB and TID containing services getting out from TIB layers: power cables (medusa cables), Al 6mm cooling pipes and AOH/DOH fiber pigtails;
\item {\tt  TIBTIDServiceCylinder}: the interface between TID and TOB
  containing services getting out from the TIB flange and from TID and laid down on the Service Cylinder, the rigid Carbon Fiber support of the TID: composition is similar to {\tt TIBFlange};
\item {\tt TIBTIDPxlFlange}: a FR4 circular crown shaped support structure for the pixel tube;
\item {\tt TIBTIDOptoPanelFront}: the front\footnote{Here and in the following {\em front} and {\em back} are referred to volumes as seen from the interaction point} portion of the Optical Panel region mainly filled with ruggedized ribbons;
\item {\tt TIBTIDOptoConnectors}: the intermediate portion of the Optical Panel region that contains the optical connectors (plastic and CuBe alloy of the MU connector ferrule) and their FR4 support;
\item {\tt TIBTIDOptoPanelBackInner}: the inner part ($r<512$mm) of the back portion of the margherita disk that contains fiber pigtail lengths being compensated in this region;
\item {\tt TIDSupport}: Carbon Fiber skin of part of the cylindrical support of TID; {\tt TIDSupportIn}:  Nomex  fill of the previous volume;
\item {\tt TIBTIDServiceDisk}: the transition region between {\tt TIBTIDServiceCylinder} and {\tt TIBTIDMargherita}: other than cables, Al 6mm cooling pipes and optical fibers pigtail, the mixture within takes into account the FR4 and Aluminium supports of the Margherita panels and the Carbon Fiber extensions of the TID Service Cylinder on which these supports are fixed; 
\item{\tt  TIBTIDOptoPanelBackOuter}: the outer part ($r>512$mm) of the back portion  of the Optical Panel that contains fiber pigtail lenghts being compensated in this region and ruggedized ribbon getting out;
\item{\tt  TIBTIDMargherita}; the front portion of the margherita disk that supports connectors of the power cable (see Figure~\ref{fig:margherita}); it contains FR4, Cu-laminated and gold plated support petals, Al 6mm cooling pipes, power cable (medusa) male connectors and accessories (screws, clamps);   
\item{\tt TIBTIDPowerConnectors}; the back portion of the margherita disk ($r>767$mm) containing lengths of ruggedized ribbon getting out, female power cable connectors and segments of Aluminium CAB60 cables and Copper PLCC cables;
\item{\tt TIBTIDTSTCabRadial}; the circular crown shaped region ($r>950$mm) of transition between the margherita disk and the Tracker Support Tube (TST) region that contains Cu 10 mm cooling pipes (the transition between Al 6mm pipes and larger Cu 10 mm pipes is infact done at the end of the margherita disk), power cables segments (CAB60 and PLCC) and ruggedized ribbon;
\item{\tt TIBTIDTSTCabAxial}; the 11mm thick cylindrical skin laying down on the internal surface of the TST (above TOB services) where all TIB/TID services are routed in their way out of the TST itself; a 1mm thick aluminium skin in used to fix cables and pipes by means of appropriate clips; these supporting structures are included in the mixture together with Cu 10 mm cooling pipes, power cables (CAB60 and PLCC) and ruggedized ribbons.
\end{enumerate}




\begin{figure*}[t]
\includegraphics*[width=0.99\textwidth]{fig/margherita.pdf}
\caption{A picture of the forward TIB/TID services as seen from outside before the TEC insertion; the margherita disk and the all cables, pipes and ruggedized ribbon within the internal surface of the TST are clearly visible. }
\label{fig:margherita}
\end{figure*}
%


\subsubsection{Computation of cable lengths}

The low voltages and high voltages to modules and the low voltage to
DOHM are provided by the so called {\em medusa cables} connected
between the head of the mother cable or the DOHM, respectively)
approximately at the level of the TIB flange and the multi service
cable connector on the margherita panel. The medusa cables are
actually just a bundle of 18 to 20 single insulated copper wires, thus
the name. It has not been possible to produce medusa cable of exactly
the needed length. Cables have been deployed in the TIB/TID as
sketched in Figure~\ref{fig:cilindro} and the extra length has been
compensated at the level of margherita panel. On each side of the
TIB/TID there are 304 medusa cables, each one routed by hand in its
way to the appropriate connector. In particular, on the Service
Cylinder, a medusa cable could have been routed either following the
minimal path, sc$_{\mathrm{min}}$ in length, the maximal path
sc$_{\mathrm max}$ in length, or following some intermediate
configuration depending on convenience. To evaluate the material due
to a medusa cables of total length L on the flange (for the TIB cables
only), the SC and the margherita panel respectively a simple model has
been applied on the medusa cable informations stored in the
integration database: 
\begin{itemize}
\item the medusa length on the TIB flange is computed from the
  corresponding radial path r given the radial position of the
  corresponding mother cable or DOHM connector; this is zero for TID
  medusa cables; 
\item the medusa minimal and maximal length on the service cylinder
  are computed from the corresponding  sc$_{\mathrm{min}}$  and
  sc$_{\mathrm{max}}$ given the angles $\phi_1$ and $\phi_2$ of the
  mother cable connector and the multi service cable connector
  respectively; sc$_{\mathrm{max}}$ is in case limited accordingly to
  r+sc$_{\mathrm{max}}$+r$_{\mathrm{conn}}<$L; 
\item the medusa cable length on the margherita is given by
  L-r-(sc$_{\mathrm{max}}$+sc$_{\mathrm{min}}$)/2.
\end{itemize}
The computation of the material takes into account the actual number
of medusa wires. Medusa cables for TIB L3 and L4 and for TID R3 does
split in 2 to 4 arms to serve 2 to 4 mother cables in parallel. This
split is done by means of a small PCBs and the arm length is comprised
between 15cm and 35cm. Also this detail is taken into account in
evaluating the material. Resulting quantities are given in
Table~\ref{tab:medusa}.
%
\begin{figure*}[h]
\includegraphics*[width=0.99\textwidth]{fig/cilindro.pdf}
\caption{A sketch of the Service Cylinder and the margherita panel seen from the interaction point. The possible routings of the TIB and TID medusa cables are shown.}
\label{fig:cilindro}
\end{figure*}
%

\begin{table}[h!]
  \caption{Medusa cable wire distribution per interested volumes for one half of TIB/TID.}
  \label{tab:medusa}
  \begin{center}
    \begin{tabular}{rcccc}
Volume Name  & Length [m]  & Cu [g]  & Ins [g] & Tot [g] \\ 
TIBFlange             &         898 &  1731 &   712 & 2443  \\
TIBTIDServiceCylinder &        2173 &  4189 &  1723 & 5912  \\
TIBTIDServiceDisk     &         449 &   866 &   356 & 1222  \\
TIBTIDMargherita      &        5382 & 10376 &  4267 & 14643 \\
Total	              &        8902 & 17162 &  7057 & 24220 \\
    \end{tabular}
  \end{center}
\end{table}

Lengths of optical fibers pigtail are computed in a similar manner
once the fiber pigtails lengths needed to reach the TIB flange or the
SC surface are taken into account. The model is simplified since fiber
pigtails are routed on the Service Cylinder essentially parallel to
the $z$-axis. Just a k-factor set to 1.2 is used to take into account
possible deviation from a path perfectly parallel to $z$, but the amount
of material is such that this detail is in any way critical. Optical
fibers lengths in excess have been compensated in the regions
represented in the simulation by the physical volumes
TIBTIDOptoPanelBackInner  and TIBTIDOptoPanelBackOuter. It has been
assumed that the total fiber length compensated in each volume is
proportional to its volume. Resulting quantities are given in
Table~\ref{tab:pigtail}. The fiber pigtail material is dominated by
the coating.

\begin{table}[h!]
  \caption{Fiber pigtail distribution per interested volumes for one half of TIB/TID.}
  \label{tab:pigtail}
  \begin{center}
    \begin{tabular}{rcc}
Volume Name              & Length [m]  & Tot [g] \\ 
TIB                      & 1319 & 1129\\
TID                      & 506  &  433\\
TIBFlange                & 664  &  568\\
TIBTIDServiceCylinder    & 2363 & 2022\\
TIBTIDServiceDisk        & 488  &  418\\
TIBTIDOptoPanelBackInner & 1777 & 1521\\
TIBTIDOptoPanelBackOuter & 2822 & 2415\\
Total	                 & 9939 & 6945\\
    \end{tabular}
  \end{center}
\end{table}

\begin{table}[h!]
  \caption{Uncertainties related to medusa cables and fiber pigtail estimations.}
  \label{tab:unc}
  \begin{center}
    \begin{tabular}{rcccc}
Volume Name   &            &  & $\frac{\Delta W}{W}$ & $\frac{\Delta X_0}{X_0}$ \\ 
TIBFlange   & medusa  & $\pm 5.0\%$  &  $\pm 4.0\%$ & $\mp 4.6 \%$ \\
                     & fibers       & $\pm 10.\%$  &  $\pm 0.9 \%$ & $\mp 0.4 \%$ \\
TIBTIDServiceCylinder  & medusa &  $\pm 6.1\%$  &  $\pm 2.9\%$ & $\mp 3.8 \%$ \\
                                           & fibers      & $\pm 10.\%$  &  $\pm 1.7\%$ & $\mp 0.9 \%$ \\
TIBTIDServiceDisk  & medusa     &  $\pm 6.0\%$  &  $\pm 1.6\%$ & $\mp 2.1 \%$ \\
                                    & fibers    &  $\pm 10.\%$  &  $\pm 0.9\%$ & $\mp 0.5 \%$ \\
TIBTIDOptoPanelBackInner & fibers &  $\pm 10.\%$  &  $\pm 6.9\%$ & $\mp 5.0 \%$ \\
TIBTIDOptoPanelBackOuter & fibers &  $\pm 10.\%$  &  $\pm 5.8\%$ & $\mp 4.6 \%$ \\
TIBTIDMargherita  & medusa    &  $\pm 3.0\%$  &  $\pm 1.2\%$ & $\mp 1.3 \%$ \\
    \end{tabular}
  \end{center}
\end{table}

%------------------------------------------------------------------------------
