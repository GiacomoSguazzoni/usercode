\documentclass{cmspaper}
\usepackage{booktabs}
\usepackage{graphicx}
\usepackage{rotating}
%\usepackage{verbatim}
\usepackage{fancyvrb}
\usepackage{units}


\begin{document}

\def \mrad      {{\rm \, mrad}}
\newcommand {\cm}         {\rm   cm}
\newcommand {\kg}         {\rm   kg}
\newcommand {\mm}         {\rm   mm}
\newcommand {\m}          {\rm   m}
\newcommand {\DM}         {\Delta {\mathrm M}}
%\newcommand {\Tot}           {\mathrm{T}}
%\newcommand {\tot}     {\mathrm{t}}
\newcommand {\Xo}         {{\mathrm{X}_0}}
\newcommand {\TX}         {{\mathrm{T}_0}}
\newcommand {\tX}         {{\mathrm{t}_0}}
\newcommand {\lI}         {{\lambda_\mathrm{I}}}
\newcommand {\TI}         {{\mathrm{T}_\mathrm{I}}}
\newcommand {\tI}         {{\mathrm{t}_\mathrm{I}}}
%\newcommand {\kvect}      {{\overrightarrow{k}}}
\newcommand {\kvect}      {{\vec{k}}}

\renewcommand{\labelenumi}{\alph{enumi})}
\newcommand{\fixme}{{\bf FIXME~}}




%==============================================================================
% title page for few authors

\begin{titlepage}

% select one of the following and type in the proper number:
%   \cmsnote{2005/000}
  \internalnote{2010/000}
%  \conferencereport{2005/000}
   \date{1 Brumaio 2010}

  \title{Evaluation of Material Budget related systematic uncertainties.}

  \begin{Authlist}
    Ernesto Migliore
     \Instfoot{to}{INFN Torino}
    Giacomo Sguazzoni
     \Instfoot{fi}{INFN Firenze}
  \end{Authlist}

 
  \begin{abstract}
A method to study the systematic uncertainties related to the Tracker Material Budget simulation has been developed.
The method can be applied whenever the effect of a know variation of a material budget component has to be studied.
A set of realistic material alterations to be used by physic analysis groups is proposed.
  \end{abstract} 

\end{titlepage}

\setcounter{page}{2}%JPP

\section{Introduction}

The present geometry contains the state-of-the-art of the Tracker material budget description. However there are knows issues:
\begin{itemize}
\item residual discrepancies between simulated and measured masses or $\DM$'s; 
\item unavoidable approximations with respect to the actual of the geometry of the physical volumes and of the material distribution and composition.
\end{itemize}

How to estimate the uncertainty related to these discrepancies and approximations?

The most quantitative bases of any possible way to answer are the mass discrepancies $\DM$, when available. Nevertheless, the impact of $\DM$ on the total thickness in terms of number of radiation lengths, $\TX$, or in terms of number of interaction lengths, $\TI$, does depend on how the $\DM$ is distributed among the various tracker components/materials. Once this dependence is known, it can be fed not only with $\DM$'s, but also with any other input we may consider useful to understand downstream effects on $\TX$ or $\TI$.

\section{Method}

As sketched in Fig.~\ref{fig:can}, the total thickness in terms of radiation lengths, $\TX$, has the following dependance on the $N$ materials:
\begin{equation}
\TX = \sum^N_i \rho_i \frac{\ell_i}{\Xo_i} \equiv \sum_i \tX_i.
\end{equation}
Similarly the total thickness in terms of radiation lengths, $\TI$, is
\begin{equation}
\TI = \sum^N_i \rho_i \frac{\ell_i}{\lI_i} \equiv \sum_i \tI_i.
\end{equation}
In the above relations, $i$ is the index running on the materials, $\rho$ the density, $\ell_i$ the thickness of the material $i$ and $\Xo$ and $\lI$ the radiation length and the interaction length, respectively, in $\mathrm{g}/\mathrm{cm}^2$. 

\begin{figure}[h]
\begin{center}
\includegraphics[width=0.4\textwidth]{fig/can.pdf}
\end{center}
\caption{Sketch of a simplified cylindrical volume made up of three different materials.}
\label{fig:can}
\end{figure}

The proposed method consists in changing in an appropriate and controlled way the material components by using a set of k-factors $\kvect\equiv(k_1, ..., k_i, ..., k_N)$ and then study the effects on $\TX$ and $\TI$. In particular
\begin{equation}
\TX = \sum^N_i \rho_i \frac{\ell_i}{\Xo_i} \rightarrow  \TX_\kvect = \sum^N_i k_i \rho_i \frac{\ell_i}{\Xo_i},
\end{equation}
and
\begin{equation}
\TI = \sum^N_i \rho_i \frac{\ell_i}{\lI_i} \rightarrow  \TI_\kvect = \sum^N_i k_i \rho_i \frac{\ell_i}{\lI_i}.
\end{equation}

\section{Parser}

The method described in the previous section can only be practically implemented by modifying the material densities in the DDD xml files describing the detector geometry. This is done by using a software tool written on purpose consisting of a perl xml parser. It takes the DDD material xml files in input and then, according to a {\em scenario} configuration file, changes the effective density of the material under study and of all materials that depend (directly and indirectly) on it; finally the modified xml files are written in output. These modified xml files are fully working in CMSSW.

The detailed modification procedure is the following:
\begin{enumerate}
\item the density attribute is modified according to the k-factors in the tags {\tt <ElementaryMaterial .../>} or {\tt <CompositeMaterial .../>} where the materials are defined;
\item all composite materials that depend on the modified materials are accordingly modified; if the composite material has density $\rho$ and fractions $w_j$, $j$ being the index running over the components, the parser apply the following substitutions:
\begin{equation}
w_j \to \frac{k_jw_j}{\sum kw}, \,\,\,\,\, \rho \to \rho \sum kw;
\end{equation}
\item step b) is repeated recursively with a scaling factor $1/\sum kw$ for all materials modified in the step b).
\end{enumerate}

The parser is accessible via CVS in ...

The parser can be launched by using the wrapper script {\tt xmlFatThaMat.sh}:
{\small
\begin{Verbatim}[frame=leftline]
./xmlSuperPorker.sh <scenario-file> <base-dir>
\end{Verbatim}
}
where {\tt <scenario-file>} is the scenario configuration file described below and {\tt 
<base-dir>} is the directory to which the paths of the xml files to be modified in the scenario fle are referred. It is essentially the base dir of the CMSSW work area we are using.


\subsection{The {\em scenario} configuration file}

An example of the scenario configuration file is given below; it is the scenario file designed to properly change the Aluminium weigth in the simulation; in  this case, for example, you want to skip materials that do contain Al (i.e.Alumina!) but that you want to study independently.

{\small
\begin{Verbatim}[frame=leftline]
<?xml version="1.0"?>
<scenario>
<file name="./Geometry/TrackerCommonData/data/tecmaterial.xml" />
<file name="./Geometry/TrackerCommonData/data/tibmaterial.xml" />
<file name="./Geometry/TrackerCommonData/data/tibtidcommonmaterial.xml" />
<file name="./Geometry/TrackerCommonData/data/tidmaterial.xml" />
<file name="./Geometry/TrackerCommonData/data/tobmaterial.xml" />
<file name="./Geometry/TrackerCommonData/data/trackermaterial.xml" />
<file name="./Geometry/CMSCommonData/data/materials.xml" />
<ElementaryMaterial name="Aluminium" kfactor="1.1">
<SkipMaterial name="Ceramic"/>
<SkipMaterial name="Alumina"/>
<SkipMaterial name="Steel-008"/>
<SkipMaterial name="Borosilicate_Glass"/>
<SkipMaterial name="TEC_PitchAdapter"/>
</ElementaryMaterial>
</scenario>
\end{Verbatim}
}

Each scenario file tag has its own meaning:
\begin{itemize}
\item the tags {\tt <file .../>} define the xml material files to be
  altered; the material files not included in any tag are not changed;
%- il tag <path .../> (sopra commentato) definisce un altro eventuale
%  path comune, oltre a quello definito in linea di comando (base-dir)
%  che debba precedere il path di ciascun file;
\item there are three possible tags to define the material variations:
\begin{description}
  \item[{\tt <ElementaryMaterial ... />}] to modify an elementary
    material and recursively all composite material that directly or
    indirectly depend on itl; if the attribute {\tt
      recurse} is given with the option {\tt false} ({\tt
      recurse=false"}) the recursive modification is not done and the
    modification is resctricted to the pure elementary material only;
  \item[{\tt <CompositeMaterial ... />}] to modify a composite
    material and recursively all composite material that directly or 
    indirectly depend on it; if the attribute {\tt
      recurse} is given with the option {\tt false} ({\tt
      recurse=''false"}) the recursive modification is not done and the
    modification is resctricted to the pure composite material only;
  \item[{\tt <rMaterial ... />}] to modify all composite materials
    that contain the specified {\tt rMaterial} and recursively all composite material that directly or 
    indirectly depend on it; in this case the material name must
    contain the namespace also; this tags is useful in the case a
    given material needs to be modified but the file containing the
    material definition is not included in any of the {\tt <file
      .../>} tags. 
\end{description}
For all material modification tags the amount of alteration is
defined via the attribute {\tt kfactor} ({\tt kfactor="..."}) that quantifies the
multiplicative factor of the density change.

All material modification tags also accept the following tags for
extra functionalities:
\begin{description}
  \item[{\tt <SkipMaterial ... />}] to specify one or more {\tt CompositeMaterial} 
  that has to remain unaltered;
\item[{\tt <OnlyMaterial ... />}] to esclusively specify the {\tt CompositeMaterial} 
  that has to be altered; if a tag of this kind is present
  no other modification is done;
\item[{\tt <SkipNamespace ... />}] to specify one or more namespaces,
  i.e. files, that has to remain unaltered;
\end{description}
\end{itemize}

As an output, the parser produces a modified xml file per each input
file configured in the {\tt <file ... />} tags. The new file has the
same name of the original one but the extension {\tt .xml} is changed
into {\tt .xml.fat}. Within these outout files, the modified materials
are identified by the attribute {\tt fat="true"} in the corresponding
material definition tags.

\subsection{Helper scripts}

To ease the material modification onto a typical CMSSW working area, two scripts have been written:
CMSSW work area:
\begin{description}
\item[{\tt fat.sh}] (usage: {\tt ./fat.sh <scenario-file> <base-dir>})
  make the parser modify the material files under the {\tt
    <base-dir>} according to the given {\tt <scenario-file>}; the
  original xml files are replaced by the modified ones in order to
  have a working CMSSW area with the new material files; the old ones
  are saved with the extension {\tt .xml.orig}; the replacement does
  not occur in the case original files appear to be already treated by
  the parser (i.e. the string {\tt fat="true"} is found in the input
  files); this to avoid to apply material modification in cascade by mistake;
\item[{\tt defat.sh}] (usage: {\tt ./defat.sh <base-dir>}) restores
  the original xml files by using the saved versions. 
\end{description}

\section{The Tracker Recipe}

The first simple application of the method allows the tracker
composition in the simulation to be evaluated. Geant4 tools, in fact, 
make easy to dump the total mass and the composition only in terms of
chemical elements of a given physical volume, and thus also of
the entire tracker. More interesting is the understanding of the composition of
the tracker in terms of basic material; as an example, a non exaustive list follows:\\
-- pure chemical elements, as aluminium, copper, or silicon;\\
-- simple compounds like alumina (${\rm Al}_2{\rm O}_3$) or glass;\\
-- more complicated compounds like carbon fiber, FR4 or plastics. 
or  since
\begin{equation}
M({\rm Tracker}) = \sum_i m_i = \sum_i \rho_i V_i,
\end{equation}
where $i$ is the index running on the material components and $m_i$,
$\rho_i$ and $V_i$ are, respectively, the mass, the density, and the
volume of the component $i$.

The method can be applied with the k-factor different from unity only
for the material component $j$. In this case the resulting total mass
of the entire tracker voulme is 
\begin{equation}
M_{\vec{k}=(..., 1, k_j, 1, ...)} (\mathrm{Tracker}) = M +
( k_j - 1) m_j,
\end{equation}
from which it is straigthforward to derive the weigth of the component $j$: 
\begin{equation}
m_j = \frac{M_{\vec{k}=(..., 1, k_j, 1, ...)} - M}{k_j
  - 1}.
\end{equation}
The procedure, repeated for all the material components, allows the tracker composition to be derived. The sum of the component masses estimated in this way is equal to the total Tracker mass within 1ppm. The exercise confirms that the method is correctly implemented.

The tables of components weight composition of the whole Tracker volume and the subdetector volumes are reported in Appendix~\ref{compositionAppendix}.

The sum of the component masses derived by this method is well within 1ppm equal to the total Tracker mass as directly given by Geant. This exercise is a confirmation that the method works and it is correctly implemented.

\subsection{Material classification in categories}

The $37$ basic materials listed in table~\ref{table:Tracker} can be grouped into a limited number of material categories: a material category is a class of homogenous and/or similar materials. This grouping is needed for the following of the study as explained in section~\ref{sec:minmax}. A category can be made up also of only one basic material.

The category definitions are given in table~\ref{table:cat}.

\begin{table}[h]
\begin{center}
\begin{tabular}{clll}
  & Category    & Short       &           \\ 
  & Name        & Name        & Components \\
  \hline 
  1 & CarbonFiber & CarbonFiber & CarbonFiber \\
  \hline 
  2 & Other mechanical structure & OtherMechStruct & Nomex \\
  & &             & Graphite \\
  & &            & Rohacell \\
  & &             & Peek   \\
  \hline 
  3 & Copper and copper alloys & CopperAndCuAlloys & Copper \\
  & & &              Bronze \\
  & & &                 Brass \\
  & & &           CuNi \\
  \hline 
  4 & Aluminum & Aluminium & Aluminium \\ 
  \hline 
  5 & Organic materials & OrganicMaterials & PE \\
  & & &               Kapton \\
  & & &       Acrylate \\
  & & &  Kevlar \\
  \hline 
6 & C6F14 & C6F14 & C6F14 \\
  \hline 
7 & Silicon sensitive & Silicon & Silicon \\
  \hline 
8 & Fiber Glass laminated & FiberGlassLaminated & FR4 \\
& & &                  G10 \\
  \hline 
9 & Inorganic oxides & InorganicOxides & Ceramic \\
& & &               Quartz \\
& & &    BorosilicateGlass \\
  \hline 
10 & Other metals & OtherMetals &  Steel \\
& & &             Titanium \\
 & & &              Silver \\
  & & &         Inconel600 \\
 & & &                Iron \\
 & & &           Beryllium \\
 & & &                Gold \\
  \hline 
11 & Glue and resins & GlueAndResins &  Epoxy \\
& & &        SiliconeGel \\
  \hline 
12 & Electronic components & ElectronicComponents & Solders \\
& & &                  BaTi \\
& & &         SiliconNoSens \\
& & &                Nickel \\
& & &             Resistors \\
  \hline 
13 & Air & Air & Air \\
\end{tabular}
\caption{Table of definitions of material categories.}
\label{table:cat}
\end{center}
\end{table}

The tables of weight per category for the whole Tracker volume and the subdetector volumes are reported in Appendix~\ref{categoriesAppendix}.

\section{The $\TX$ and $\TI$ dependance on materials}

The dependance of $\TX$ and $\TI$ on the single material or category contribution is the starting point of the systematic study. These dependances are also very important to understand which material or category dominates the total radiation length and which material or category is, on the other hand, negligible.

The derivation of the dependance of $\TX$ and $\TI$ on the materials or categories is similar to the method described above for the mass fractions; if $k_j$ is the factor different from unity:
\begin{equation}
\mathrm{T}_{(...,1,{k_j},1...)} = \mathrm{T} + (k_j - 1) t_j,
\end{equation}
from which it is straigthforward to derive: 
\begin{equation}
t_j = \frac{\mathrm{T}_{(...,1,k_j,1...)} - \mathrm{T}}{k_j - 1} ,
\end{equation}
where $\mathrm{T}$ can be $\TX$ or $\TI$ and $t_j$ the respective component.

\section{Derivation of the systematic uncertainties}

The systematic uncertainties on the representation of the material at the level of GEANT simulation can be due several sources, as, for example, the lack of knowledge or approximation in the shape and size of the volumes, chemical compositions and mass. Independently of the source, any effect can be represented to the extent of our needs as an appropriate relative variations $\nicefrac{\Delta \rho}{\rho}$ on the material densities resulting in a variation on the total thickness
\begin{equation}
\Delta \mathrm{T} = \sum_i \frac{\Delta \rho_i}{\rho_i} t_i
\end{equation}
where, again, $\mathrm{T}$ can be $\TX$ or $\TI$ and $t_j$ the respective component measured by means of the simple method described above.

An insight on the systematic uncertainties on the knowledge of the material can be derived by maximizing or minimizing $\TX$ and $\TI$ with respect to the possible the density variations within an allowed domain:
\begin{equation}
\mathrm{\TX} + \Delta \mathrm{\TX} = \sum_i \frac{\ell_i}{X_{0 i}} \left( \rho_i + \Delta \rho_i \right) = \sum_i \tX_i k_i,
\end{equation}
\begin{equation}
\mathrm{\TI} + \Delta \mathrm{\TI} = \sum_i \frac{\ell_i}{\lambda_{{\mathrm I} i}} \left( \rho_i + \Delta \rho_i \right) = \sum_i \tI_i k_i.
\end{equation}

The aim of the present study is to derive the set of k-factors $\{k^0_{\rm min}\}$ ({\em minimal $X_0$ scenario}) and $\{k^0_{\rm max}\}$ ({\em maximal $X_0$ scenario})  corresponding to the minimal and maximal variations $\Delta \TX_{\rm min}$ and $\Delta \TX_{\rm max}$ compatible with the systematic uncertainties. Similar entities can be defined with respect to $\lambda_{\rm I}$ in place of $X_0$: $\{k^{\rm I}_{\rm min}\}$ ({\em minimal $\lambda_{\rm I}$ scenario}) and $\{k^{\rm I}_{\rm max}\}$ ({\em maximal $\lambda_{\rm I}$ scenario}) corresponding to the minimal and maximal variations $\Delta \TI_{\rm min}$ and $\Delta \TI_{\rm max}$. The variation is defined as the average of $\Delta \TX$ or $\Delta \TI$ over the $\eta$ range comprised between $-3$ and $3$. 

The derivation of the maximal and minimal scenarios results from an extrema finding problem with respect to the variation of material densities intended to represent the effects of the systematic uncertainties. To ease the minimization and maximization procedures the material categories are used in place of the basic materials, much larger in number. The approach is soound since similar materials have been grouped on purpose into homogeneous categories.

The density variation domain results by the imposition of an appropriate set of constraints. These are listed below.
\begin{itemize}
%
% http://indico.cern.ch/getFile.py/access?contribId=6&resId=0&materialId=slides&confId=26422
\item The overall difference between measured mass and total simulated mass is kept fixed by imposing
 \begin{equation}
\Delta M = \sum_i \frac{\Delta \rho_i}{\rho_i} m_i.
\end{equation}
The Silicon Strip Tracker weight as estimated during the integration and installation is $3996^{+88}_{-132}\kg$; the Pixel detector, installed in a later time, is not included in this estimation; the corresponding weigth in the simulation is $4037\kg$. The resulting relative measured versus simulated weigth difference is $-1^{+2}_{-3}\%$ suggesting an agreement between measured and simulated weigths within the estimation error.


Similar estimation performed over single subdetector indicate possible localized larger discrepancies. For example the TIB/TID subdetector plus some ancillary pieces have been weigthed. The result is $450^{+12}_{-20}\kg$ to be compared with $427.0\kg$, the simulated mass of the corresponding volumes. The resulting measured versus simulated weigth difference is $5^{+3}_{-4}\%$. As far as the TEC subdetector is concerned an uncertainty of about $\sim 3\%$ is claimed by the experts.

As a consequence a conservative approach has been choosen in this study: it is assumed that the actual tracker mass is $5\%$ larger than the mass on the corresponding volumes in the simulation. {\bf FIXME FIXME FIXME}
\item The density variation ranges within each category are constrained depending on the charachteristic and typical use of the materials contained in the categories. A large amount of direct experience in integrating and assemblying the tracker is used in defining these numbers. They are listed in table~\ref{table:catVariations}.

Composite materials for which the chemical composition can vary substantially and used in shapes with not well defined geometry that can be difficult to represent in the simulation can vary up to $\pm 15\%$; typical examples in this typology is the carbon fiber and the organic material (i.e. plastics).

Pure materials and metals are typically well known chemically and easier to represent. The cooling fluid is an exception since its amount is estimated by the internal volume of the pipes that build up the cooling system, often with shapes very complicated. This estimation is difficult.

The category ``Glue and resins'' is assigned with a variation range as large as $\pm 20\%$ because volumes of glues, resins, silicone gel heavily used for assemblying modules and in the integration activity in general were not at all controlled with respect to the amounts used, very dependent of many parameters like temperature.

The density of the sensitive Silicon volume is kept fixed to avoid affecting the signal generated in the active volumes. Nevertheless it is worth noticing that also silicon detector thicknesses can vary up to $\sim 10\%$.
\end{itemize} 


\begin{table}[h]
\begin{center}
\begin{tabular}{clc}
  & Category    & Allowed density \\ 
  & Name        & Variation in $\%$    \\
  \hline 
  1 & Carbon Fiber               & $\pm15$ \\
  2 & Other mechanical structure & $\pm15$ \\
  3 & Copper and copper alloys   & $\pm5$ \\
  4 & Aluminum                   & $\pm5$\\ 
  5 & Organic materials          & $\pm15$ \\
  6 & C6F14                      & $\pm10$\\
  7 & Silicon sensitive          & Fixed                       \\
  8 & Fiber Glass laminated      & $\pm10$\\
  9 & Inorganic oxides           & $\pm5$ \\
 10 & Other metals               & $\pm10$\\
 11 & Glue and resins            & $\pm20$\\
 12 & Electronic components      & $\pm10$\\
 13 & Air                        & Fixed \\
\end{tabular}
\caption{Allowed density variation of the various material categories in the minimization/maximization problem.}
\label{table:catVariations}
\end{center}
\end{table}

\subsection{The extrema finding problem}



\clearpage

\appendix
\section{Tables of components weight.}
\label{compositionAppendix}

\vskip 1cm

\begin{table}[h]
\begin{center}
\begin{tabular}{clrrrrrr}
  &           & \multicolumn{2}{c}{Mass} & \multicolumn{2}{c}{Total} & \multicolumn{2}{c}{Remaining} \\ 
  & Name      & [kg]    & Fraction & [kg]  & Fraction & [kg]   & Fraction \\ 
 \hline 
 1 &          CarbonFiber & 1144.503 & 27.631\% & 1144.503 & 27.631\%  & 2997.524 & 72.369\% \\
 2 &               Copper & 595.674 & 14.381\% & 1740.176 & 42.013\%  & 2401.851 & 57.987\% \\
 3 &            Aluminium & 594.960 & 14.364\% & 2335.136 & 56.377\%  & 1806.891 & 43.623\% \\
 4 &                   PE & 354.633 & 8.562\% & 2689.769 & 64.938\%  & 1452.258 & 35.062\% \\
 5 &                C6F14 & 258.890 & 6.250\% & 2948.659 & 71.189\%  & 1193.368 & 28.811\% \\
 6 &              Silicon & 225.847 & 5.453\% & 3174.506 & 76.641\%  & 967.521 & 23.359\% \\
 7 &                Nomex & 123.331 & 2.978\% & 3297.836 & 79.619\%  & 844.190 & 20.381\% \\
 8 &                  G10 & 110.180 & 2.660\% & 3408.016 & 82.279\%  & 734.011 & 17.721\% \\
 9 &                  FR4 & 103.238 & 2.492\% & 3511.254 & 84.771\%  & 630.773 & 15.229\% \\
10 &              Ceramic &  91.062 & 2.198\% & 3602.316 & 86.970\%  & 539.711 & 13.030\% \\
11 &               Kapton &  82.757 & 1.998\% & 3685.073 & 88.968\%  & 456.954 & 11.032\% \\
12 &                Steel &  81.521 & 1.968\% & 3766.595 & 90.936\%  & 375.432 & 9.064\% \\
13 &                Epoxy &  57.053 & 1.377\% & 3823.647 & 92.313\%  & 318.380 & 7.687\% \\
14 &             Graphite &  50.237 & 1.213\% & 3873.884 & 93.526\%  & 268.143 & 6.474\% \\
15 &               Quartz &  39.025 & 0.942\% & 3912.909 & 94.468\%  & 229.118 & 5.532\% \\
16 &             Titanium &  31.707 & 0.765\% & 3944.615 & 95.234\%  & 197.412 & 4.766\% \\
17 &             Acrylate &  29.604 & 0.715\% & 3974.220 & 95.949\%  & 167.807 & 4.051\% \\
18 &    BorosilicateGlass &  23.134 & 0.559\% & 3997.354 & 96.507\%  & 144.673 & 3.493\% \\
19 &               Bronze &  19.910 & 0.481\% & 4017.264 & 96.988\%  & 124.763 & 3.012\% \\
20 &          SiliconeGel &  18.246 & 0.440\% & 4035.510 & 97.428\%  & 106.517 & 2.572\% \\
21 &               Silver &  16.053 & 0.388\% & 4051.562 & 97.816\%  &  90.465 & 2.184\% \\
22 &                Brass &  14.932 & 0.361\% & 4066.495 & 98.176\%  &  75.532 & 1.824\% \\
23 &             Rohacell &  14.539 & 0.351\% & 4081.034 & 98.527\%  &  60.993 & 1.473\% \\
24 &                 CuNi &  13.941 & 0.337\% & 4094.975 & 98.864\%  &  47.052 & 1.136\% \\
25 &              Solders &   9.868 & 0.238\% & 4104.842 & 99.102\%  &  37.184 & 0.898\% \\
26 &                 BaTi &   9.837 & 0.237\% & 4114.680 & 99.340\%  &  27.347 & 0.660\% \\
27 &           Inconel600 &   7.426 & 0.179\% & 4122.105 & 99.519\%  &  19.922 & 0.481\% \\
28 &               Kevlar &   5.119 & 0.124\% & 4127.224 & 99.643\%  &  14.803 & 0.357\% \\
29 &        SiliconNoSens &   4.161 & 0.100\% & 4131.385 & 99.743\%  &  10.642 & 0.257\% \\
30 &                 Iron &   3.919 & 0.095\% & 4135.305 & 99.838\%  &   6.722 & 0.162\% \\
31 &                 Peek &   3.572 & 0.086\% & 4138.877 & 99.924\%  &   3.150 & 0.076\% \\
32 &               Nickel &   1.675 & 0.040\% & 4140.552 & 99.964\%  &   1.475 & 0.036\% \\
33 &            Beryllium &   0.760 & 0.018\% & 4141.311 & 99.983\%  &   0.716 & 0.017\% \\
34 &                 Gold &   0.525 & 0.013\% & 4141.836 & 99.995\%  &   0.191 & 0.005\% \\
35 &            Resistors &   0.172 & 0.004\% & 4142.008 & 100.000\%  &   0.019 & 0.000\% \\
36 &                  Air &   0.019 & 0.000\% & 4142.027 & 100.000\%  &   0.000 & 0.000\% \\
 \hline 
  & Total & 4142.027 & 100.000\% & & & & \\ 

\end{tabular}
\caption{Weight of the basic components in the Tracker volume.}
\label{table:Tracker}
\end{center}
\end{table}

\begin{table}[h]
\begin{center}
\begin{tabular}{clrrrrrr}
  &           & \multicolumn{2}{c}{Weight} & \multicolumn{2}{c}{Total} & \multicolumn{2}{c}{Remaining} \\ 
  & Name      & [kg]    & Fraction & [kg]  & Fraction & [kg]   & Fraction \\ 
 \hline 
 1 &          CarbonFiber &  13.618 & 17.770\% &  13.618 & 17.770\%  &  63.016 & 82.230\% \\
 2 &               Copper &  10.621 & 13.859\% &  24.239 & 31.630\%  &  52.395 & 68.370\% \\
 3 &                C6F14 &   8.716 & 11.374\% &  32.955 & 43.003\%  &  43.679 & 56.997\% \\
 4 &            Aluminium &   8.588 & 11.207\% &  41.543 & 54.210\%  &  35.091 & 45.790\% \\
 5 &                   PE &   7.944 & 10.366\% &  49.487 & 64.576\%  &  27.147 & 35.424\% \\
 6 &                  FR4 &   7.105 & 9.272\% &  56.593 & 73.847\%  &  20.042 & 26.153\% \\
 7 &                Epoxy &   6.124 & 7.991\% &  62.717 & 81.839\%  &  13.918 & 18.161\% \\
 8 &                Steel &   5.505 & 7.184\% &  68.222 & 89.023\%  &   8.413 & 10.977\% \\
 9 &             Acrylate &   4.558 & 5.948\% &  72.780 & 94.971\%  &   3.854 & 5.029\% \\
10 &              Silicon &   0.995 & 1.298\% &  73.775 & 96.269\%  &   2.860 & 3.731\% \\
11 &                Nomex &   0.720 & 0.939\% &  74.495 & 97.208\%  &   2.140 & 2.792\% \\
12 &               Kapton &   0.420 & 0.548\% &  74.914 & 97.755\%  &   1.720 & 2.245\% \\
13 &              Ceramic &   0.402 & 0.525\% &  75.317 & 98.280\%  &   1.318 & 1.720\% \\
14 &          SiliconeGel &   0.379 & 0.494\% &  75.695 & 98.774\%  &   0.939 & 1.226\% \\
15 &               Kevlar &   0.278 & 0.363\% &  75.973 & 99.137\%  &   0.661 & 0.863\% \\
16 &              Solders &   0.197 & 0.257\% &  76.170 & 99.394\%  &   0.464 & 0.606\% \\
17 &                Brass &   0.184 & 0.240\% &  76.354 & 99.634\%  &   0.280 & 0.366\% \\
18 &                  G10 &   0.134 & 0.175\% &  76.488 & 99.809\%  &   0.146 & 0.191\% \\
19 &                 BaTi &   0.117 & 0.152\% &  76.605 & 99.961\%  &   0.030 & 0.039\% \\
20 &                 Gold &   0.018 & 0.023\% &  76.623 & 99.984\%  &   0.012 & 0.016\% \\
21 &               Nickel &   0.009 & 0.011\% &  76.631 & 99.996\%  &   0.003 & 0.004\% \\
22 &                  Air &   0.003 & 0.004\% &  76.635 & 100.000\%  &   0.000 & 0.000\% \\
 \hline 
  & Total &   76.635 & 100.000\% & & & & \\ 

\end{tabular}
\caption{Weight of the basic components in the Pixel Barrel volume.}
\end{center}
\end{table}

\begin{table}[h]
\begin{center}
\begin{tabular}{clrrrrrr}
\input{recipes/PixelForwardZPlus}
\end{tabular}
\caption{Weight of the basic components in the Pixel Forward half located at $z>0$; the Pixel Forward half located at $z<0$ is identical.}
\end{center}
\end{table}

\begin{table}[h]
\begin{center}
\begin{tabular}{clrrrrrr}
  &           & \multicolumn{2}{c}{Mass} & \multicolumn{2}{c}{Total} & \multicolumn{2}{c}{Remaining} \\ 
  & Name      & [kg]    & Fraction & [kg]  & Fraction & [kg]   & Fraction \\ 
 \hline 
 1 &          CarbonFiber &  62.121 & 29.219\% &  62.121 & 29.219\%  & 150.485 & 70.781\% \\
 2 &            Aluminium &  32.724 & 15.392\% &  94.845 & 44.611\%  & 117.760 & 55.389\% \\
 3 &               Copper &  21.914 & 10.307\% & 116.759 & 54.918\%  &  95.846 & 45.082\% \\
 4 &              Silicon &  15.297 & 7.195\% & 132.056 & 62.113\%  &  80.550 & 37.887\% \\
 5 &                C6F14 &  15.210 & 7.154\% & 147.266 & 69.267\%  &  65.340 & 30.733\% \\
 6 &               Kapton &  14.339 & 6.745\% & 161.605 & 76.011\%  &  51.001 & 23.989\% \\
 7 &                   PE &  10.258 & 4.825\% & 171.863 & 80.837\%  &  40.743 & 19.163\% \\
 8 &              Ceramic &   8.885 & 4.179\% & 180.748 & 85.016\%  &  31.858 & 14.984\% \\
 9 &                  FR4 &   8.516 & 4.005\% & 189.264 & 89.021\%  &  23.342 & 10.979\% \\
10 &               Bronze &   5.199 & 2.445\% & 194.462 & 91.466\%  &  18.143 & 8.534\% \\
11 &                  G10 &   4.743 & 2.231\% & 199.205 & 93.697\%  &  13.400 & 6.303\% \\
12 &                Epoxy &   4.399 & 2.069\% & 203.604 & 95.766\%  &   9.002 & 4.234\% \\
13 &          SiliconeGel &   2.361 & 1.110\% & 205.965 & 96.876\%  &   6.641 & 3.124\% \\
14 &    BorosilicateGlass &   2.183 & 1.027\% & 208.148 & 97.903\%  &   4.458 & 2.097\% \\
15 &             Acrylate &   1.471 & 0.692\% & 209.618 & 98.595\%  &   2.987 & 1.405\% \\
16 &                 BaTi &   1.104 & 0.519\% & 210.722 & 99.114\%  &   1.883 & 0.886\% \\
17 &        SiliconNoSens &   1.050 & 0.494\% & 211.772 & 99.608\%  &   0.833 & 0.392\% \\
18 &                 Iron &   0.346 & 0.163\% & 212.119 & 99.771\%  &   0.487 & 0.229\% \\
19 &              Solders &   0.252 & 0.118\% & 212.370 & 99.889\%  &   0.235 & 0.111\% \\
20 &               Quartz &   0.101 & 0.047\% & 212.471 & 99.937\%  &   0.135 & 0.063\% \\
21 &                 Gold &   0.065 & 0.030\% & 212.536 & 99.967\%  &   0.070 & 0.033\% \\
22 &                Brass &   0.040 & 0.019\% & 212.576 & 99.986\%  &   0.030 & 0.014\% \\
23 &            Resistors &   0.030 & 0.014\% & 212.606 & 100.000\%  &   0.000 & 0.000\% \\
 \hline 
  & Total &  212.606 & 100.000\% & & & & \\ 

\end{tabular}
\caption{Weight of the basic components in the TIB volume.}
\end{center}
\end{table}

\begin{table}[h]
\begin{center}
\begin{tabular}{clrrrrrr}
\input{recipes/TIDF}
\end{tabular}
\caption{Weight of the basic components in the forward TID volume; backward TID volume is identical.}
\end{center}
\end{table}

\begin{table}[h]
\begin{center}
\begin{tabular}{clrrrrrr}
\input{recipes/TIBTIDServicesF}
\end{tabular}
\caption{Weight of the basic components in the forward TIB and TID service volume; the backward replica is identical.}
\end{center}
\end{table}

\begin{table}[h]
\begin{center}
\begin{tabular}{clrrrrrr}
  &           & \multicolumn{2}{c}{Weight} & \multicolumn{2}{c}{Total} & \multicolumn{2}{c}{Remaining} \\ 
  & Name      & [kg]    & Fraction & [kg]  & Fraction & [kg]   & Fraction \\ 
 \hline 
 1 &          CarbonFiber & 263.406 & 25.337\% & 263.406 & 25.337\%  & 776.184 & 74.663\% \\
 2 &               Copper & 124.997 & 12.024\% & 388.403 & 37.361\%  & 651.187 & 62.639\% \\
 3 &            Aluminium & 121.048 & 11.644\% & 509.451 & 49.005\%  & 530.139 & 50.995\% \\
 4 &              Silicon & 110.498 & 10.629\% & 619.949 & 59.634\%  & 419.641 & 40.366\% \\
 5 &                  G10 &  99.293 & 9.551\% & 719.242 & 69.185\%  & 320.347 & 30.815\% \\
 6 &                C6F14 &  86.370 & 8.308\% & 805.612 & 77.493\%  & 233.977 & 22.507\% \\
 7 &                   PE &  75.903 & 7.301\% & 881.515 & 84.795\%  & 158.074 & 15.205\% \\
 8 &              Ceramic &  36.157 & 3.478\% & 917.671 & 88.272\%  & 121.918 & 11.728\% \\
 9 &               Kapton &  21.757 & 2.093\% & 939.428 & 90.365\%  & 100.161 & 9.635\% \\
10 &                Epoxy &  20.287 & 1.951\% & 959.715 & 92.317\%  &  79.874 & 7.683\% \\
11 &                Nomex &  14.346 & 1.380\% & 974.061 & 93.697\%  &  65.528 & 6.303\% \\
12 &                 CuNi &  13.941 & 1.341\% & 988.002 & 95.038\%  &  51.587 & 4.962\% \\
13 &          SiliconeGel &  13.320 & 1.281\% & 1001.323 & 96.319\%  &  38.267 & 3.681\% \\
14 &    BorosilicateGlass &   8.799 & 0.846\% & 1010.122 & 97.165\%  &  29.468 & 2.835\% \\
15 &             Acrylate &   8.130 & 0.782\% & 1018.252 & 97.947\%  &  21.338 & 2.053\% \\
16 &                 BaTi &   7.811 & 0.751\% & 1026.063 & 98.699\%  &  13.527 & 1.301\% \\
17 &               Silver &   4.914 & 0.473\% & 1030.976 & 99.172\%  &   8.613 & 0.828\% \\
18 &               Kevlar &   2.603 & 0.250\% & 1033.580 & 99.422\%  &   6.010 & 0.578\% \\
19 &                Steel &   1.462 & 0.141\% & 1035.042 & 99.563\%  &   4.547 & 0.437\% \\
20 &               Quartz &   1.231 & 0.118\% & 1036.274 & 99.681\%  &   3.316 & 0.319\% \\
21 &        SiliconNoSens &   1.137 & 0.109\% & 1037.411 & 99.790\%  &   2.178 & 0.210\% \\
22 &              Solders &   1.100 & 0.106\% & 1038.511 & 99.896\%  &   1.078 & 0.104\% \\
23 &                Brass &   0.750 & 0.072\% & 1039.261 & 99.968\%  &   0.328 & 0.032\% \\
24 &                 Iron &   0.273 & 0.026\% & 1039.534 & 99.995\%  &   0.055 & 0.005\% \\
25 &            Resistors &   0.055 & 0.005\% & 1039.589 & 100.000\%  &   0.000 & 0.000\% \\
 \hline 
  & Total & 1039.589 & 100.000\% & & & & \\ 

\end{tabular}
\caption{Weight of the basic components in the TOB volume.}
\end{center}
\end{table}

\begin{table}[h]
\begin{center}
\begin{tabular}{clrrrrrr}
  &           & \multicolumn{2}{c}{Weight} & \multicolumn{2}{c}{Total} & \multicolumn{2}{c}{Remaining} \\ 
  & Name      & [kg]    & Fraction & [kg]  & Fraction & [kg]   & Fraction \\ 
 \hline 
 1 &          CarbonFiber & 194.292 & 29.105\% & 194.292 & 29.105\%  & 473.272 & 70.895\% \\
 2 &                   PE &  64.880 & 9.719\% & 259.171 & 38.823\%  & 408.392 & 61.177\% \\
 3 &            Aluminium &  63.340 & 9.488\% & 322.511 & 48.312\%  & 345.052 & 51.688\% \\
 4 &               Copper &  49.541 & 7.421\% & 372.053 & 55.733\%  & 295.511 & 44.267\% \\
 5 &              Silicon &  46.466 & 6.960\% & 418.518 & 62.693\%  & 249.045 & 37.307\% \\
 6 &                Nomex &  35.256 & 5.281\% & 453.774 & 67.975\%  & 213.789 & 32.025\% \\
 7 &                C6F14 &  32.197 & 4.823\% & 485.971 & 72.798\%  & 181.592 & 27.202\% \\
 8 &                Steel &  30.539 & 4.575\% & 516.510 & 77.373\%  & 151.053 & 22.627\% \\
 9 &                  FR4 &  28.236 & 4.230\% & 544.746 & 81.602\%  & 122.817 & 18.398\% \\
10 &             Graphite &  25.118 & 3.763\% & 569.865 & 85.365\%  &  97.698 & 14.635\% \\
11 &              Ceramic &  19.994 & 2.995\% & 589.859 & 88.360\%  &  77.704 & 11.640\% \\
12 &               Quartz &  17.876 & 2.678\% & 607.735 & 91.038\%  &  59.829 & 8.962\% \\
13 &             Titanium &  15.733 & 2.357\% & 623.468 & 93.395\%  &  44.095 & 6.605\% \\
14 &                Epoxy &  11.426 & 1.712\% & 634.894 & 95.106\%  &  32.670 & 4.894\% \\
15 &               Kapton &  10.469 & 1.568\% & 645.362 & 96.674\%  &  22.201 & 3.326\% \\
16 &    BorosilicateGlass &   5.433 & 0.814\% & 650.796 & 97.488\%  &  16.767 & 2.512\% \\
17 &                Brass &   3.696 & 0.554\% & 654.492 & 98.042\%  &  13.072 & 1.958\% \\
18 &               Silver &   2.659 & 0.398\% & 657.151 & 98.440\%  &  10.412 & 1.560\% \\
19 &              Solders &   2.600 & 0.389\% & 659.751 & 98.830\%  &   7.812 & 1.170\% \\
20 &             Acrylate &   2.526 & 0.378\% & 662.277 & 99.208\%  &   5.287 & 0.792\% \\
21 &                  G10 &   2.139 & 0.320\% & 664.415 & 99.528\%  &   3.148 & 0.472\% \\
22 &                 Peek &   1.786 & 0.268\% & 666.201 & 99.796\%  &   1.362 & 0.204\% \\
23 &        SiliconNoSens &   0.805 & 0.121\% & 667.006 & 99.917\%  &   0.557 & 0.083\% \\
24 &                 BaTi &   0.222 & 0.033\% & 667.229 & 99.950\%  &   0.334 & 0.050\% \\
25 &                 Iron &   0.173 & 0.026\% & 667.402 & 99.976\%  &   0.162 & 0.024\% \\
26 &               Nickel &   0.109 & 0.016\% & 667.511 & 99.992\%  &   0.053 & 0.008\% \\
27 &            Resistors &   0.039 & 0.006\% & 667.549 & 99.998\%  &   0.014 & 0.002\% \\
28 &                  Air &   0.008 & 0.001\% & 667.557 & 99.999\%  &   0.006 & 0.001\% \\
29 &                 Gold &   0.006 & 0.001\% & 667.563 & 100.000\%  &   0.000 & 0.000\% \\
 \hline 
  & Total &  667.563 & 100.000\% & & & & \\ 

\end{tabular}
\caption{Weight of the basic components in one of the two TEC volume replicas.}
\end{center}
\end{table}

\begin{table}[h]
\begin{center}
\begin{tabular}{clrrrrrr}
  &           & \multicolumn{2}{c}{Weight} & \multicolumn{2}{c}{Total} & \multicolumn{2}{c}{Remaining} \\ 
  & Name      & [kg]    & Fraction & [kg]  & Fraction & [kg]   & Fraction \\ 
 \hline 
 1 &               Copper & 110.151 & 54.453\% & 110.151 & 54.453\%  &  92.134 & 45.547\% \\
 2 &          CarbonFiber &  34.079 & 16.847\% & 144.229 & 71.300\%  &  58.056 & 28.700\% \\
 3 &            Aluminium &  22.036 & 10.893\% & 166.265 & 82.193\%  &  36.020 & 17.807\% \\
 4 &                   PE &  21.289 & 10.524\% & 187.554 & 92.718\%  &  14.731 & 7.282\% \\
 5 &                C6F14 &   5.995 & 2.963\% & 193.548 & 95.681\%  &   8.737 & 4.319\% \\
 6 &               Kapton &   4.262 & 2.107\% & 197.810 & 97.788\%  &   4.475 & 2.212\% \\
 7 &                Steel &   2.152 & 1.064\% & 199.962 & 98.852\%  &   2.323 & 1.148\% \\
 8 &               Nickel &   0.707 & 0.349\% & 200.668 & 99.201\%  &   1.616 & 0.799\% \\
 9 &                Epoxy &   0.448 & 0.221\% & 201.116 & 99.422\%  &   1.169 & 0.578\% \\
10 &               Silver &   0.441 & 0.218\% & 201.557 & 99.640\%  &   0.727 & 0.360\% \\
11 &                Nomex &   0.306 & 0.151\% & 201.863 & 99.791\%  &   0.422 & 0.209\% \\
12 &               Quartz &   0.260 & 0.128\% & 202.123 & 99.920\%  &   0.162 & 0.080\% \\
13 &             Titanium &   0.120 & 0.059\% & 202.243 & 99.979\%  &   0.042 & 0.021\% \\
14 &                  FR4 &   0.024 & 0.012\% & 202.267 & 99.991\%  &   0.018 & 0.009\% \\
15 &        SiliconNoSens &   0.018 & 0.009\% & 202.285 & 100.000\%  &   0.000 & 0.000\% \\
 \hline 
  & Total &  202.285 & 100.000\% & & & & \\ 

\end{tabular}
\caption{Weight of the basic components in one of the two Tracker Bulkhead volume replicas.}
\end{center}
\end{table}

\begin{table}[h]
\begin{center}
\begin{tabular}{clrrrrrr}
  &           & \multicolumn{2}{c}{Mass} & \multicolumn{2}{c}{Total} & \multicolumn{2}{c}{Remaining} \\ 
  & Name      & [kg]    & Fraction & [kg]  & Fraction & [kg]   & Fraction \\ 
 \hline 
 1 &          CarbonFiber & 293.221 & 54.572\% & 293.221 & 54.572\%  & 244.094 & 45.428\% \\
 2 &            Aluminium & 146.301 & 27.228\% & 439.522 & 81.800\%  &  97.792 & 18.200\% \\
 3 &                Nomex &  33.834 & 6.297\% & 473.356 & 88.097\%  &  63.959 & 11.903\% \\
 4 &                C6F14 &  28.854 & 5.370\% & 502.209 & 93.467\%  &  35.105 & 6.533\% \\
 5 &             Rohacell &  14.539 & 2.706\% & 516.749 & 96.172\%  &  20.566 & 3.828\% \\
 6 &                Steel &   8.230 & 1.532\% & 524.979 & 97.704\%  &  12.335 & 2.296\% \\
 7 &           Inconel600 &   7.426 & 1.382\% & 532.405 & 99.086\%  &   4.910 & 0.914\% \\
 8 &               Kapton &   4.910 & 0.914\% & 537.314 & 100.000\%  &   0.000 & 0.000\% \\
 \hline 
  & Total &  537.314 & 100.000\% & & & & \\ 

\end{tabular}
\caption{Weight of the basic components of the Tracker Outer Cylinder volume.}
\end{center}
\end{table}

\clearpage

\section{Tables of material category weight per subdetector}
\label{categoriesAppendix}

\vskip 1cm


\begin{table}[h]
\begin{center}
\begin{tabular}{clrrrrrr}
  &           & \multicolumn{2}{c}{Mass} & \multicolumn{2}{c}{Total} & \multicolumn{2}{c}{Remaining} \\ 
  & Name      & [kg]    & Fraction & [kg]  & Fraction & [kg]   & Fraction \\ 
 \hline 
 1 &          CarbonFiber & 1144.503 & 27.631\% & 1144.503 & 27.631\%  & 2997.524 & 72.369\% \\
 2 &    CopperAndCuAlloys & 644.458 & 15.559\% & 1788.960 & 43.190\%  & 2353.067 & 56.810\% \\
 3 &            Aluminium & 594.960 & 14.364\% & 2383.920 & 57.554\%  & 1758.107 & 42.446\% \\
 4 &     OrganicMaterials & 472.113 & 11.398\% & 2856.033 & 68.953\%  & 1285.994 & 31.047\% \\
 5 &                C6F14 & 258.890 & 6.250\% & 3114.923 & 75.203\%  & 1027.104 & 24.797\% \\
 6 &              SiliconSensitive & 225.847 & 5.453\% & 3340.770 & 80.655\%  & 801.257 & 19.345\% \\
 7 &  FiberGlassLaminated & 213.418 & 5.152\% & 3554.188 & 85.808\%  & 587.839 & 14.192\% \\
 8 &      OtherMechStruct & 191.679 & 4.628\% & 3745.866 & 90.436\%  & 396.161 & 9.564\% \\
 9 &      InorganicOxides & 153.221 & 3.699\% & 3899.087 & 94.135\%  & 242.940 & 5.865\% \\
10 &          OtherMetals & 141.909 & 3.426\% & 4040.997 & 97.561\%  & 101.030 & 2.439\% \\
11 &        GlueAndResins &  75.298 & 1.818\% & 4116.295 & 99.379\%  &  25.732 & 0.621\% \\
12 & ElectronicComponents &  25.713 & 0.621\% & 4142.008 & 100.000\%  &   0.019 & 0.000\% \\
13 &                  Air &   0.019 & 0.000\% & 4142.027 & 100.000\%  &   0.000 & 0.000\% \\
 \hline 
  & Total & 4142.027 & 100.000\% & & & & \\ 

\end{tabular}
\caption{Weight per material category in the Tracker volume.}
\end{center}
\end{table}

\begin{table}[h]
\begin{center}
\begin{tabular}{clrrrrrr}
  &           & \multicolumn{2}{c}{Mass} & \multicolumn{2}{c}{Total} & \multicolumn{2}{c}{Remaining} \\ 
  & Name      & [kg]    & Fraction & [kg]  & Fraction & [kg]   & Fraction \\ 
 \hline 
 1 &          CarbonFiber &  13.618 & 17.770\% &  13.618 & 17.770\%  &  63.016 & 82.230\% \\
 2 &     OrganicMaterials &  13.200 & 17.225\% &  26.818 & 34.995\%  &  49.816 & 65.005\% \\
 3 &    CopperAndCuAlloys &  10.805 & 14.099\% &  37.623 & 49.094\%  &  39.011 & 50.906\% \\
 4 &                C6F14 &   8.716 & 11.374\% &  46.339 & 60.468\%  &  30.295 & 39.532\% \\
 5 &            Aluminium &   8.588 & 11.207\% &  54.928 & 71.675\%  &  21.707 & 28.325\% \\
 6 &  FiberGlassLaminated &   7.239 & 9.447\% &  62.167 & 81.121\%  &  14.468 & 18.879\% \\
 7 &        GlueAndResins &   6.503 & 8.485\% &  68.670 & 89.607\%  &   7.965 & 10.393\% \\
 8 &          OtherMetals &   5.523 & 7.207\% &  74.192 & 96.813\%  &   2.442 & 3.187\% \\
 9 &              SiliconSensitive &   0.995 & 1.298\% &  75.187 & 98.111\%  &   1.447 & 1.889\% \\
10 &      OtherMechStruct &   0.720 & 0.939\% &  75.907 & 99.050\%  &   0.728 & 0.950\% \\
11 &      InorganicOxides &   0.402 & 0.525\% &  76.309 & 99.575\%  &   0.325 & 0.425\% \\
12 & ElectronicComponents &   0.322 & 0.421\% &  76.631 & 99.996\%  &   0.003 & 0.004\% \\
13 &                  Air &   0.003 & 0.004\% &  76.635 & 100.000\%  &   0.000 & 0.000\% \\
 \hline 
  & Total &   76.635 & 100.000\% & & & & \\ 

\end{tabular}
\caption{Weight per material category in the Pixel Barrel volume.}
\end{center}
\end{table}

\begin{table}[h]
\begin{center}
\begin{tabular}{clrrrrrr}
  &           & \multicolumn{2}{c}{Weight} & \multicolumn{2}{c}{Total} & \multicolumn{2}{c}{Remaining} \\ 
  & Name      & [kg]    & Fraction & [kg]  & Fraction & [kg]   & Fraction \\ 
 \hline 
 1 &          CarbonFiber &   3.633 & 25.995\% &   3.633 & 25.995\%  &  10.344 & 74.005\% \\
 2 &    CopperAndCuAlloys &   3.085 & 22.075\% &   6.719 & 48.069\%  &   7.259 & 51.931\% \\
 3 &            Aluminium &   2.572 & 18.400\% &   9.291 & 66.469\%  &   4.687 & 33.531\% \\
 4 &                C6F14 &   1.532 & 10.958\% &  10.822 & 77.427\%  &   3.155 & 22.573\% \\
 5 &     OrganicMaterials &   1.358 & 9.717\% &  12.181 & 87.144\%  &   1.797 & 12.856\% \\
 6 &  FiberGlassLaminated &   0.781 & 5.587\% &  12.961 & 92.731\%  &   1.016 & 7.269\% \\
 7 &              Silicon &   0.393 & 2.811\% &  13.354 & 95.542\%  &   0.623 & 4.458\% \\
 8 &          OtherMetals &   0.380 & 2.718\% &  13.734 & 98.260\%  &   0.243 & 1.740\% \\
 9 &        GlueAndResins &   0.108 & 0.772\% &  13.842 & 99.031\%  &   0.135 & 0.969\% \\
10 &      InorganicOxides &   0.076 & 0.540\% &  13.918 & 99.572\%  &   0.060 & 0.428\% \\
11 & ElectronicComponents &   0.060 & 0.428\% &  13.977 & 100.000\%  &   0.000 & 0.000\% \\
 \hline 
  & Total &   13.977 & 100.000\% & & & & \\ 

\end{tabular}
\caption{Weight per material category in the Pixel Forward half located at $z>0$; the Pixel Forward half located at $z<0$ is identical.}
\end{center}
\end{table}

\begin{table}[h]
\begin{center}
\begin{tabular}{clrrrrrr}
  &           & \multicolumn{2}{c}{Weight} & \multicolumn{2}{c}{Total} & \multicolumn{2}{c}{Remaining} \\ 
  & Name      & [kg]    & Fraction & [kg]  & Fraction & [kg]   & Fraction \\ 
 \hline 
 1 &          CarbonFiber &  62.121 & 29.219\% &  62.121 & 29.219\%  & 150.485 & 70.781\% \\
 2 &            Aluminium &  32.724 & 15.392\% &  94.845 & 44.611\%  & 117.760 & 55.389\% \\
 3 &    CopperAndCuAlloys &  27.153 & 12.772\% & 121.998 & 57.382\%  &  90.607 & 42.618\% \\
 4 &     OrganicMaterials &  26.068 & 12.261\% & 148.067 & 69.644\%  &  64.539 & 30.356\% \\
 5 &              Silicon &  15.297 & 7.195\% & 163.363 & 76.839\%  &  49.243 & 23.161\% \\
 6 &                C6F14 &  15.210 & 7.154\% & 178.573 & 83.992\%  &  34.033 & 16.008\% \\
 7 &  FiberGlassLaminated &  13.259 & 6.236\% & 191.831 & 90.229\%  &  20.774 & 9.771\% \\
 8 &      InorganicOxides &  11.169 & 5.253\% & 203.000 & 95.482\%  &   9.605 & 4.518\% \\
 9 &        GlueAndResins &   6.759 & 3.179\% & 209.759 & 98.661\%  &   2.846 & 1.339\% \\
10 & ElectronicComponents &   2.435 & 1.145\% & 212.195 & 99.807\%  &   0.411 & 0.193\% \\
11 &          OtherMetals &   0.411 & 0.193\% & 212.606 & 100.000\%  &   0.000 & 0.000\% \\
 \hline 
  & Total &  212.606 & 100.000\% & & & & \\ 

\end{tabular}
\caption{Weight per material category in the TIB volume.}
\end{center}
\end{table}

\begin{table}[h]
\begin{center}
\begin{tabular}{clrrrrrr}
\input{recipes/TIDF.cats}
\end{tabular}
\caption{Weight per material category in the forward TID volume; backward TID volume is identical.}
\end{center}
\end{table}

\begin{table}[h]
\begin{center}
\begin{tabular}{clrrrrrr}
\input{recipes/TIBTIDServicesF.cats}
\end{tabular}
\caption{Weight per material category in the forward TIB and TID service volume; the backward replica is identical.}
\end{center}
\end{table}

\begin{table}[h]
\begin{center}
\begin{tabular}{clrrrrrr}
  &           & \multicolumn{2}{c}{Weight} & \multicolumn{2}{c}{Total} & \multicolumn{2}{c}{Remaining} \\ 
  & Name      & [kg]    & Fraction & [kg]  & Fraction & [kg]   & Fraction \\ 
 \hline 
 1 &          CarbonFiber & 263.406 & 25.337\% & 263.406 & 25.337\%  & 776.184 & 74.663\% \\
 2 &    CopperAndCuAlloys & 139.688 & 13.437\% & 403.094 & 38.774\%  & 636.496 & 61.226\% \\
 3 &            Aluminium & 121.048 & 11.644\% & 524.142 & 50.418\%  & 515.448 & 49.582\% \\
 4 &              Silicon & 110.498 & 10.629\% & 634.640 & 61.047\%  & 404.950 & 38.953\% \\
 5 &     OrganicMaterials & 108.393 & 10.427\% & 743.033 & 71.474\%  & 296.557 & 28.526\% \\
 6 &  FiberGlassLaminated &  99.293 & 9.551\% & 842.326 & 81.025\%  & 197.263 & 18.975\% \\
 7 &                C6F14 &  86.370 & 8.308\% & 928.696 & 89.333\%  & 110.893 & 10.667\% \\
 8 &      InorganicOxides &  46.187 & 4.443\% & 974.883 & 93.776\%  &  64.706 & 6.224\% \\
 9 &        GlueAndResins &  33.607 & 3.233\% & 1008.490 & 97.009\%  &  31.099 & 2.991\% \\
10 &      OtherMechStruct &  14.346 & 1.380\% & 1022.837 & 98.389\%  &  16.753 & 1.611\% \\
11 & ElectronicComponents &  10.103 & 0.972\% & 1032.940 & 99.360\%  &   6.650 & 0.640\% \\
12 &          OtherMetals &   6.650 & 0.640\% & 1039.589 & 100.000\%  &   0.000 & 0.000\% \\
 \hline 
  & Total & 1039.589 & 100.000\% & & & & \\ 

\end{tabular}
\caption{Weight per material category in the TOB volume.}
\end{center}
\end{table}

\begin{table}[h]
\begin{center}
\begin{tabular}{clrrrrrr}
  &           & \multicolumn{2}{c}{Mass} & \multicolumn{2}{c}{Total} & \multicolumn{2}{c}{Remaining} \\ 
  & Name      & [kg]    & Fraction & [kg]  & Fraction & [kg]   & Fraction \\ 
 \hline 
 1 &          CarbonFiber & 194.292 & 29.105\% & 194.292 & 29.105\%  & 473.272 & 70.895\% \\
 2 &     OrganicMaterials &  77.874 & 11.665\% & 272.166 & 40.770\%  & 395.398 & 59.230\% \\
 3 &            Aluminium &  63.340 & 9.488\% & 335.506 & 50.258\%  & 332.057 & 49.742\% \\
 4 &      OtherMechStruct &  62.160 & 9.312\% & 397.666 & 59.570\%  & 269.897 & 40.430\% \\
 5 &    CopperAndCuAlloys &  53.237 & 7.975\% & 450.904 & 67.545\%  & 216.660 & 32.455\% \\
 6 &          OtherMetals &  49.111 & 7.357\% & 500.015 & 74.901\%  & 167.548 & 25.099\% \\
 7 &              SiliconSensitive &  46.466 & 6.960\% & 546.480 & 81.862\%  & 121.083 & 18.138\% \\
 8 &      InorganicOxides &  43.303 & 6.487\% & 589.784 & 88.349\%  &  77.780 & 11.651\% \\
 9 &                C6F14 &  32.197 & 4.823\% & 621.980 & 93.172\%  &  45.583 & 6.828\% \\
10 &  FiberGlassLaminated &  30.375 & 4.550\% & 652.355 & 97.722\%  &  15.208 & 2.278\% \\
11 &        GlueAndResins &  11.426 & 1.712\% & 663.781 & 99.433\%  &   3.782 & 0.567\% \\
12 & ElectronicComponents &   3.775 & 0.565\% & 667.555 & 99.999\%  &   0.008 & 0.001\% \\
13 &                  Air &   0.008 & 0.001\% & 667.563 & 100.000\%  &   0.000 & 0.000\% \\
 \hline 
  & Total &  667.563 & 100.000\% & & & & \\ 

\end{tabular}
\caption{Weight per material category in one of the two TEC volume replicas.}
\end{center}
\end{table}

\begin{table}[h]
\begin{center}
\begin{tabular}{clrrrrrr}
  &           & \multicolumn{2}{c}{Mass} & \multicolumn{2}{c}{Total} & \multicolumn{2}{c}{Remaining} \\ 
  & Name      & [kg]    & Fraction & [kg]  & Fraction & [kg]   & Fraction \\ 
 \hline 
 1 &    CopperAndCuAlloys & 110.151 & 54.453\% & 110.151 & 54.453\%  &  92.134 & 45.547\% \\
 2 &          CarbonFiber &  34.079 & 16.847\% & 144.229 & 71.300\%  &  58.056 & 28.700\% \\
 3 &     OrganicMaterials &  25.551 & 12.631\% & 169.780 & 83.931\%  &  32.505 & 16.069\% \\
 4 &            Aluminium &  22.036 & 10.893\% & 191.816 & 94.825\%  &  10.469 & 5.175\% \\
 5 &                C6F14 &   5.995 & 2.963\% & 197.810 & 97.788\%  &   4.475 & 2.212\% \\
 6 &          OtherMetals &   2.713 & 1.341\% & 200.523 & 99.129\%  &   1.762 & 0.871\% \\
 7 & ElectronicComponents &   0.725 & 0.358\% & 201.248 & 99.487\%  &   1.037 & 0.513\% \\
 8 &        GlueAndResins &   0.448 & 0.221\% & 201.695 & 99.709\%  &   0.589 & 0.291\% \\
 9 &      OtherMechStruct &   0.306 & 0.151\% & 202.001 & 99.860\%  &   0.284 & 0.140\% \\
10 &      InorganicOxides &   0.260 & 0.128\% & 202.261 & 99.988\%  &   0.024 & 0.012\% \\
11 &  FiberGlassLaminated &   0.024 & 0.012\% & 202.285 & 100.000\%  &   0.000 & 0.000\% \\
 \hline 
  & Total &  202.285 & 100.000\% & & & & \\ 

\end{tabular}
\caption{Weight per material category in one of the two Tracker Bulkhead volume replicas.}
\end{center}
\end{table}

\begin{table}[h]
\begin{center}
\begin{tabular}{clrrrrrr}
  &           & \multicolumn{2}{c}{Mass} & \multicolumn{2}{c}{Total} & \multicolumn{2}{c}{Remaining} \\ 
  & Name      & [kg]    & Fraction & [kg]  & Fraction & [kg]   & Fraction \\ 
 \hline 
 1 &          CarbonFiber & 293.221 & 54.572\% & 293.221 & 54.572\%  & 244.094 & 45.428\% \\
 2 &            Aluminium & 146.301 & 27.228\% & 439.522 & 81.800\%  &  97.792 & 18.200\% \\
 3 &      OtherMechStruct &  48.373 & 9.003\% & 487.895 & 90.803\%  &  49.419 & 9.197\% \\
 4 &                C6F14 &  28.854 & 5.370\% & 516.749 & 96.172\%  &  20.566 & 3.828\% \\
 5 &          OtherMetals &  15.656 & 2.914\% & 532.405 & 99.086\%  &   4.910 & 0.914\% \\
 6 &     OrganicMaterials &   4.910 & 0.914\% & 537.314 & 100.000\%  &   0.000 & 0.000\% \\
 \hline 
  & Total &  537.314 & 100.000\% & & & & \\ 

\end{tabular}
\caption{Weight per material category of the Tracker Outer Cylinder volume.}
\end{center}
\end{table}

\end{document}

% LocalWords:  TODO
