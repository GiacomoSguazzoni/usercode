\section{A glimpse into the future}
\label{glimpse}

The challenge for the CMS reconstruction cannot be considered over
with the deployment of the software for 2012 data taking, currently
ongoing. After the first long shutdown, foreseen for almost two years
across 2013 and 2014, LHC will increase center-of-mass energy
and instantaneous luminosity as well. This will require a major
reengineering of the entire reconstruction software and of the
tracking as well.

Two major areas of improvements are being outlined: implementation of
tracking techniques never used in CMS up to now (like Hough transforms
tracking); exploitation at any possible level of parallelization
techniques. The latter, in particular, turns out to be necessary to
better profit of the actual trend of increase of the computing power
that is realized by an increase of the number of cores in the same
monolithic CPU.

Parallelization can be implemented in several ways. At the level of
the framework by allowing different modules to run in parellel taking
appropriately into account all dependencies; this would be almost
transparent for the final user and developer, i.e. it would require no
to minor changes to user and reconstruction module code. Nevertheless
this is not optimal as some modules of CMSSW need much longer time
to run with respect to the others; track reconstruction is the
typical example. In this case it is worthwhile to implement
parallelization at the module and algorithm level. This requires code
modifications but is also much more effective. Prototype
implementations are already being studied and are very
promising~\cite{parallel}.




