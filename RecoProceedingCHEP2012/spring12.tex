\section{Spring 2012 campaign}
\label{spring12}

The modifications put in place in the second phase of the
improvement campaign have been developed on top of CMSSW version 4.4.x
and implemented in the CMSSW version 5.2.x. Again a group of
improvements are based on better coding and technological
improvements and do not change the physics outcome. More in detail,
these modifications are described below.

\begin{description}
\item[Change of compiler version.] The implementation of CMSSW
  5.2.x has been accompanied by the switch from \verb-gcc- 4.3.4
  to \verb-gcc- 4.6.2 to produce binaries. This latter compiler
  version allows for faster code to be generated also thanks to some
  compiler specific optimizations. The net gain is up
  to 10\% as shown in figure~\ref{fig:gcc} where the reduction in CPU
  time is shown as a function of the number of pile-up vertices for
  simulated QCD events.
\begin{figure}[b]
%\begin{center}
%\includegraphics*[width=0.44\textwidth]{fig/sketch.pdf}
%\hskip 2mm
\includegraphics*[width=0.5\textwidth]{figs/gcc.png}\hspace{0.02\textwidth}%
\begin{minipage}[b]{0.38\textwidth}\caption{\label{fig:gcc}Relative
    CPU time reduction to be ascribed to the introduction of
    gcc 4.6.2 as a function of the number of PU vertices for
    simulated QCD events.} 
\end{minipage}
%\vskip -5mm
%\end{center}
\end{figure}
  Other features that came along the new compiler version are the
  C++11 standard support and autovectorization flags on by default.
\item[JEMalloc.] The concurrent malloc implementation
  JEMalloc, highly performant and able to better redeem memory, has
  been implemented in place of the standard malloc.
%\item[autovectorization]
\item[Switch to improved ROOT version.] The ROOT package version has
  been changed from 5.27 to 5.32 that features several improvements,
  especially in I/O with less memory required. 
\item[Several design modifications to improve speed and memory
  consumption.] The code has been again carefully reviewed and many
  improvements have been implemented. Several of those are related to
  track reconstruction classes. For example, the devirtualization of
  the BasicTrajectoryState class (an ancillary class for track
  reconstruction) resulted into a 10\% gain in speed and in some
  $100$MB of resident set size (RSS) saved per event. Similarly the
  stereo hit class (the class that stores the double sided module hits) has been
  considerably slimmed down (a factor three in size) with a net decrease of
  RSS memory from 50MB to 150MB, depending on the event occupancy.
\end{description}

Another set of modifications directly affects the outcome on physics
output. These are described in the following.
\begin{description}
\item[Offline vertexing.] The offline reconstruction of primary
  vertices is based on a deterministic annealing algorithm to find the
  $z$ coordinate of the vertices. Major improvements have been
  deployed for CMSSW 5.2.x: loops have been autovectorized (thanks to
  the introduction of the new compiler) but, to further profit of
  autovectorization capabilities, the exponential functions heavily
  used in the algorithm have been replaced with a fast,
  autovectorizable inlined double precision version. Eventually the
  deterministic annealing algorithm has been further made more
  efficient by optimizing some configuration parameters
  with essentially no change in physics performances. The net increase
  in CPU time amounts to a factor 3 for large PU events.
\item[Cluster shape based seed filtering] The large CPU time needed by
  the track reconstruction is to be ascribed to the huge number of
  seeds due to hit combinatorics; in fact a propagation has to be attempted for
  each of them. A way to keep this number under control is to
  implement filters able to reject fake seeds. One of the most
  effective is based on the cluster shape. For example a track
  impinging a sensor with a large angle will generate a cluster
  wider than a track with normal incidence. This can be used to
  evaluate seed compatibility with the track hypothesis. Such a filter
  was used only in steps \#0 and \#1 in CMSSW 4.4.x (see
  table~\ref{table:44xIterativeSteps}); for CMSSW 5.2.x it has been
  extended also to steps \#2, \#4 and \#5 (see
  table~\ref{table:52xIterativeSteps}) with a substantial CPU time
  benefit. For example, the step \#2, 
  particularly prone to combinatorics since seeds are made up of hit
  pairs, sees a CPU time reduction of a factor 2.7. Overall the
  improvement in CPU time is of a factor 1.5. As a side effect of the
  filter also the fake rate is reduced by $\sim20\%$.
\item[Iterative tracking.] After all the modifications
  described above, also the iterative tracking has been further optimized
  for CMSSW 5.2.x. Nevertheless the differences, summarized in
  table~\ref{table:52xIterativeSteps}, are tiny, which demonstrates
  that upstream improvements are already almost sufficient to make
  CMSSW compliant with requirements. There is no need to modify deeply
  the iterative tracking, i.e. to reduce combinatorics and to match
  performance target by increasing effective $P_T$ thresholds and/or
  by reducing efficiency for displaced tracks. A relevant change
  introduced as part of the optimization consist of the upgrade of the
  final track cleaning and selection criteria. Eventually the
  efficiency for prompts tracks with $P_T$ larger than $0.9\GeVc$ is
  not affected but the fake rate is reduced by about $\sim35\%$.
\begin{table}[t]
  \caption{  \label{table:52xIterativeSteps} Relevant parameters of
    the seven tracking iterative steps in 
    CMSSW 5.2.x, after the second phase of the improvement campaign in
    2012; in bold the parameters changed with respect to the 
    corresponding steps of CMSSW 4.4.x in
    table~\ref{table:44xIterativeSteps}; see
    table~\ref{table:42xIterativeSteps} 
    caption for symbol definitions.}
  \centering
  \begin{tabular}{llllll}
\br
  \#step & seed type & seed subdetectors & $P^{\rm min}_T$ [$\GeVc$] &
  $d_0$ cut & $z_0$ cut \\ \mr  
  0 & triplet & pixel & 0.6 & {\bf 0.02}$\cm$ & 4.0$\sigma$ \\
  1 & triplet & pixel & 0.2 & {\bf 0.02}$\cm$ & 4.0$\sigma$ \\
  2 & pair    & pixel & 0.6 & {\bf 0.015}$\cm$ & 0.09$\cm$ \\
  3 & triplet & pixel & {\bf 0.3} & {\bf 1.5}$\cm$ & {\bf 2.5}$\sigma$ \\
  4 & triplet & pixel/TIB/TID/TEC & {\bf 0.5-0.6} & {\bf 1.5}$\cm$ & 10.0$\cm$ \\
  5 & pair    & TIB/TID/TEC       & 0.6 & 2.0$\cm$ & 10.0$\cm$ \\
  6 & pair    & TOB/TEC           & 0.6       & 2.0$\cm$ & 30.0$\cm$ \\
\br
\end{tabular}
\end{table}
\end{description}
\begin{figure}[b]
%\begin{center}
%\includegraphics*[width=0.44\textwidth]{fig/sketch.pdf}
%\hskip 2mm
\includegraphics*[width=0.58\textwidth]{figs/spring12.png}\hspace{0.02\textwidth}%
\begin{minipage}[b]{0.35\textwidth}\caption{\label{fig:spring12}
 RSS memory as a
function of running time in CMSSW 4.4.x and CMSSW 5.2.x for a
reconstruction job of 100 real data events from the 2011 special
run with high PU.}
\end{minipage}
%\vskip -5mm
%\end{center}
\end{figure}

The overall result obtained with the ``spring 2012'' campaign
improvements implemented in CMSSW 5.2.x is shown in
figure~\ref{fig:spring12} where the dependence of RSS memory as a
function of running time is plotted in CMSSW 4.4.x and CMSSW 5.2.x for a
reconstruction job of 100 real data events from the 2011 special
run with high PU. The substantial reduction both in memory load as
well as in total running time is clearly evident.

The CMSSW 5.2.x releases have been
fully validated and have been accepted for production since changes in
performaces are minor with respect to physics outcome.  






