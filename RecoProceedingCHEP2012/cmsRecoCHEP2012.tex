\documentclass[a4paper]{jpconf}
\usepackage{graphicx}


\usepackage{slashed}
\usepackage{subfigure}
%\usepackage{rotating}
\usepackage{multirow}
\usepackage{amsmath}
%\usepackage{units}
\setkeys{Gin}{width=\linewidth,totalheight=\textheight,keepaspectratio}
\graphicspath{{fig/}}

%
% My Macros
%
\usepackage{graphicx}
%\usepackage{drftcite}
\usepackage{pstricks}
\usepackage[figuresright]{rotating}
%
%
% Macro declarations
%
%---- SLASH
\def\slasha#1{#1\hskip-0.65em /}  %slasha per caratteri piccoli
\def\slashb#1{#1\hskip-1.3em /}   %slashb per quelli grandi
\def\slashc#1{#1\hskip-.4em /}
%
%---- UNITA` DI MISURA
\def \pb        {{\rm \, pb}}
\def \fb        {{\rm \, fb}}
\def \ipb       {{\rm \, pb^{-1}}}
\def \ifb       {{\rm \, fb^{-1}}}
\def \eV        {{\rm \,  eV}}
\def \keV       {{\rm \, keV}}
\def \MeV       {{\rm \, MeV}}
\def \GeV       {{\rm \, GeV}}
\def \TeV       {{\rm \, TeV}}
\def \TeVc      {\TeV/c}
\def \TeVcc     {\TeV/c^2}
\def \GeVc      {\GeV/c}
\def \GeVcc     {\GeV/c^2}
\def \MeVc      {\MeV/c}
\def \MeVcc     {\MeV/c^2}
%
%---- SIMBOLI
\def\ga{\mathrel{\raise.3ex\hbox{$>$\kern-.75em\lower1ex\hbox{$\sim$}}}}
\def\la{\mathrel{\raise.3ex\hbox{$<$\kern-.75em\lower1ex\hbox{$\sim$}}}}
\newcommand {\lesssim}
     {\,\raisebox{-0.6ex}{$\stackrel{\textstyle<}{\textstyle\sim}$}\,}
\newcommand {\gtrsim}
     {\,\raisebox{-0.6ex}{$\stackrel{\textstyle>}{\textstyle\sim}$}\,}
\newcommand{\ckm}{$\checkmark$}
%
%---- MISCELLANEA
%\newcommand {\slashed}[1] { \mbox{\rlap{\hbox{/}} #1 }}
\newcommand {\onehalf}    {\raisebox{0.1ex}{${\frac{1}{2}}$}}
\newcommand {\fivethirds} {\raisebox{0.1ex}{${\frac{5}{3}}$}}
\newcommand {\OR}         {{\tt OR}\,}
\newcommand {\BR}         {{\rm BR}\,}
\newcommand {\rts}        {\sqrt{s}}
\newcommand {\lumi}       {\mathcal{L}}
\newcommand {\Lumi}       {\int\lumi\mathrm{d}t}
\newcommand {\gradi}    {^\circ}
\newcommand {\de}         {\partial}
\newcommand {\um}         {\, \mu \rm m}
\newcommand {\nm}         {\rm \, nm}
\newcommand {\us}         {\, \mu \rm s}
\newcommand {\cm}         {\rm \, cm}
\newcommand {\mm}         {\rm \, mm}
\newcommand {\m}          {\rm \, m}
\newcommand {\km}         {\rm \, km}
\newcommand {\V}          {\rm \, V}
\newcommand {\T}          {\rm \, T}
\newcommand {\kV}         {\rm \, kV}
\newcommand {\kVm}        {\rm \, kV\! / \! m} 
\newcommand {\MVm}        {\rm \, MV\! / \! m} 
\newcommand {\ns}         {\rm \, ns} 
\newcommand {\ps}         {\rm \, ps} 
%
%---- THEORY groups & AOB
\newcommand {\gws}        {\mathrm{SU(2)_L \otimes U(1)_Y}}
\newcommand {\sul}        {\mathrm{SU(2)_L}}
\newcommand {\suc}        {\mathrm{SU(3)_C}}
\newcommand {\ul}         {\mathrm{U(1)_Y}}
\newcommand {\uem}        {\mathrm{U(1)_{em}}}
\newcommand {\sigmabar}   {\overline{\sigma}}
\newcommand {\gmunu}      {g^{\mu \nu}}
\newcommand {\munu}       {{\mu \nu}}
\newcommand {\obra}       {\langle 0 |}
\newcommand {\oket}       {| 0 \rangle}
%
%---- THEORY lepton fields
\newcommand {\LL}         {L^{\alpha}_{\mathrm L}}
\newcommand {\LLd}        {L^{\dagger \alpha}_{\mathrm L}}
\newcommand {\lL}         {\ell^{\alpha}_{\mathrm L}}
\newcommand {\lLd}        {\ell^{\dagger \alpha}_{\mathrm L}}
\newcommand {\ld}         {\ell^{\dagger \alpha}}
\newcommand {\lb}         {\overline{\ell}^{\alpha}}
\newcommand {\lR}         {\ell^{\alpha}_{\mathrm R}}
\newcommand {\lRd}        {\ell^{\dagger \alpha}_{\mathrm R}}
\newcommand {\nuL}        {\nu^{\alpha}_{\mathrm L}}
\newcommand {\nuLb}       {\overline{\nu}^{\alpha}_{\mathrm L}}
\newcommand {\nub}        {\overline{\nu}^{\alpha}}
\newcommand {\lept}       {\ell^\alpha}
\newcommand {\neut}       {\nu^{\alpha}}
\newcommand {\nuLd}       {\nu^{\dagger \alpha}_{\mathrm L}}
\newcommand {\Phid}       {\Phi^\dagger}
%
%---- THEORY quark fields
\newcommand {\up}         {u^{\alpha}}
\newcommand {\ub}         {\overline{u}^{\alpha}}
\newcommand {\down}       {d^{\alpha}}
\newcommand {\db}         {\overline{d}^{\alpha}}
\newcommand {\QL}         {Q^{\alpha}_{\mathrm L}}
\newcommand {\QLd}        {Q^{\dagger \alpha}_{\mathrm L}}
\newcommand {\UL}         {U^{\alpha}_{\mathrm L}}
\newcommand {\ULd}        {U^{\dagger \alpha}_{\mathrm L}}
\newcommand {\UR}         {U^{\alpha}_{\mathrm R}}
\newcommand {\URd}        {U^{\dagger \alpha}_{\mathrm R}}
\newcommand {\DL}         {D^{\alpha}_{\mathrm L}}
\newcommand {\DLd}        {D^{\dagger \alpha}_{\mathrm L}}
\newcommand {\DR}         {D^{\alpha}_{\mathrm R}}
\newcommand {\DRd}        {D^{\dagger \alpha}_{\mathrm R}}
\newcommand {\bfell}      {\ell\kern-0.4em
                           \ell\kern-0.4em
                           \ell\kern-0.4em
                           \ell }
\newcommand {\obfell}     {\overline{\ell}\kern-0.4em
                           \overline{\ell}\kern-0.4em
                           \overline{\ell}\kern-0.4em
                           \overline{\ell}}
\newcommand {\bfH}      {\, {\cal H}\kern-0.5em \kern-0.4em
                           {\cal H}\kern-0.5em \kern-0.4em
                           {\cal H}\kern0.1em }
\newcommand {\obfH}     {\, \overline{\cal H}\kern-0.5em \kern-0.4em 
                           \overline{\cal H}\kern-0.5em \kern-0.4em 
                           \overline{\cal H}\kern0.1em }
%
%---- PARTICELLE
\def \b             {{\mathrm b}}
\def \t             {{\mathrm t}}
\def \charm         {{\mathrm c}}
\def \d             {{\mathrm d}}
\def \u             {{\mathrm u}}
\def \e             {{\mathrm e}}
\def \q             {{\mathrm q}}
\def \g             {{\mathrm g}}
\def \p             {{\mathrm p}}
\def \s             {{\mathrm s}}
\def \n             {{\mathrm n}}
\def \h             {{\mathrm h}}
\def \l             {\ell} 
\def \f             {{\mathrm f}} 
%\def \f             {{f}} 
\def \A             {{\mathrm A}}
\def \B             {{\mathrm B}}
\def \D             {{\mathrm D}}
\def \K             {{\mathrm K}}
\def \X             {{\mathrm X}}
\def \Y             {{\mathrm Y}}
\def \W             {{\mathrm W}}
\def \H             {{\mathrm H}}
\def \Z             {{\mathrm Z}}
\def \S             {{\mathrm S}}
\def \N             {{\mathrm N}}
\def \L             {{\mathrm L}}
\def \R             {{\mathrm R}}
\def \P             {{\mathrm P}}
\def \G             {{\mathrm G}}
%
%---- Higgs
\newcommand {\ho}         {{\h^0}}
\newcommand {\Ho}         {{\H^0}}
\newcommand {\Ao}         {{\A^0}}
\newcommand {\Hpm}        {{\H^\pm}}
\newcommand {\clsb}       {{\mathrm CL_{\rm s+b}}}
\newcommand {\clb}        {{\mathrm CL_{\rm b}}}
%
%---- SUSY
\newcommand {\dm}         {\Delta m}
\newcommand {\dM}         {\Delta M}
\newcommand {\ldm}        {\mbox{``low $\dm$''}}
\newcommand {\hdm}        {\mbox{``high $\dm$''}}
\newcommand {\nnc}        {{\overline{\mathrm N}_{95}}}
\newcommand {\snc}        {{\overline{\sigma}_{95}}}
\newcommand {\susy}       {{supersymmetry}}
\newcommand {\susyc}      {{supersymmetric}}
\newcommand {\aj}         {\mbox{\sf AJ}}
\newcommand {\ajl}        {\mbox{\sf AJL}}
\newcommand {\llh}        {\mbox{\sf LLH}}
%
%---- SPARTICELLE
\newcommand {\rpc}     {{\rm RPC}}
\newcommand {\rpv}     {{\rm RPV}}
\newcommand {\sfe}     {{\tilde{\f}}}
\newcommand {\sfL}     {{\tilde{\f}_{\mathrm L}}}
\newcommand {\sfR}     {{\tilde{\f}_{\mathrm R}}}
\newcommand {\sfone}   {{\tilde{\f}_{1}}}
\newcommand {\sftwo}   {{\tilde{\f}_{2}}}
\newcommand {\sneu}    {{\tilde{\nu}}}
\newcommand {\wino}    {{\mathrm{\widetilde{W}}}}
\newcommand {\bino}    {{\mathrm{\widetilde{B}}}}
\newcommand {\se}      {{\mathrm{\tilde{e}}}}
\newcommand {\seR}     {{\mathrm{\tilde{e}_{R}}}}
\newcommand {\seL}     {{\mathrm{\tilde{e}_{L}}}}
\newcommand {\st}      {{\mathrm{\tilde{\tau}}}}
\newcommand {\stR}     {{\mathrm{\tilde{\tau}_{R}}}}
\newcommand {\stL}     {{\mathrm{\tilde{\tau}_{L}}}}
\newcommand {\stone}   {{\mathrm{\tilde{\tau}_{1}}}}
\newcommand {\sttwo}   {{\mathrm{\tilde{\tau}_{2}}}}
\newcommand {\sm}      {{\mathrm{\tilde{\mu}}}}
\newcommand {\smR}     {{\mathrm{\tilde{\mu}_{R}}}}
\newcommand {\smL}     {{\mathrm{\tilde{\mu}_{L}}}}
\newcommand {\Sup}     {{\mathrm{\tilde{u}}}}
\newcommand {\suR}     {{\mathrm{\tilde{u}_{R}}}}
\newcommand {\suL}     {{\mathrm{\tilde{u}_{L}}}}
\newcommand {\sdo}     {{\mathrm{\tilde{d}}}}
\newcommand {\sdR}     {{\mathrm{\tilde{d}_{R}}}}
\newcommand {\sdL}     {{\mathrm{\tilde{d}_{L}}}}
\newcommand {\sch}     {{\mathrm{\tilde{c}}}}
\newcommand {\scR}     {{\mathrm{\tilde{c}_{R}}}}
\newcommand {\scL}     {{\mathrm{\tilde{c}_{L}}}}
\newcommand {\sst}     {{\mathrm{\tilde{s}}}}
\newcommand {\ssR}     {{\mathrm{\tilde{s}_{R}}}}
\newcommand {\ssL}     {{\mathrm{\tilde{s}_{L}}}}
\newcommand {\stopR}   {{\tilde{\mathrm{t}}_{R}}}
\newcommand {\stopL}   {{\tilde{\mathrm{t}}_{L}}}
\newcommand {\stopone} {{\tilde{\mathrm{t}}_{1}}}
\newcommand {\stoptwo} {{\mathrm{\tilde{t}_{2}}}}
\newcommand {\sto}     {{\tilde{\mathrm{t}}}}
\newcommand {\SQ}      {{\mathrm{\widetilde{Q}}}}
\newcommand {\STO}     {{\mathrm{\widetilde{T}}}}
\newcommand {\glu}     {{\mathrm{\tilde{g}}}}
\newcommand {\sbotR}   {{\mathrm{\tilde{b}_{R}}}}
\newcommand {\sbotL}   {{\mathrm{\tilde{b}_{L}}}}
\newcommand {\sbotone} {{\mathrm{\tilde{b}_{1}}}}
\newcommand {\sbottwo} {{\mathrm{\tilde{b}_{2}}}}
\newcommand {\sbot}    {{\tilde{\mathrm{b}}}}
\newcommand {\squa}    {{\tilde{\mathrm{q}}}}
\newcommand {\squal}   {{\tilde{\mathrm{q}}_{\rm L}}}
\newcommand {\squar}   {{\tilde{\mathrm{q}}_{\rm R}}}
\newcommand {\sqL}     {{\tilde{\mathrm{q}}_{\rm L}}}
\newcommand {\sqR}     {{\tilde{\mathrm{q}}_{\rm R}}}
\newcommand {\snu}     {{\tilde{\nu}}}
\newcommand {\snue}    {{\tilde{\nu}_{\mathrm e}}}
\newcommand {\snum}    {{\tilde{\nu}_{\mu}}}
\newcommand {\snut}    {{\tilde{\nu}_{\tau}}}
\newcommand {\neu}     {{\chi}}
\newcommand {\chap}    {{\chi^+}}
\newcommand {\cham}    {{\chi^-}}
\newcommand {\chapm}   {{\chi^\pm}}

%
%---- SUSY PARAMETRI
\newcommand {\thstop} {\mathrm{\theta_{\tilde{t}}}}
\newcommand {\thsbot} {\mathrm{\theta_{\tilde{b}}}}
\newcommand {\thsqua} {\mathrm{\theta_{\tilde{q}}}}
\newcommand {\Mcha}{M_{\chi^\pm}}
\newcommand {\Mchi}{M_\chi}
\newcommand {\Msnu}{M_{\tilde{\nu}}}
\newcommand {\tanb}{\tan\beta}
%
%---- ABBREVIAZIONI

%
%---- PROCESSI FISICI
\newcommand {\rb}    {{\rm R_{\b}}}
\newcommand {\qq}    {{\q \overline{\q}}}
\newcommand {\bb}    {{\b \overline{\b}}}
\newcommand {\cc}    {{\charm \overline{\charm}}}
\newcommand {\ff}    {{\f \overline{\f}}}
\newcommand {\el}    {{\e ^+}}
\newcommand {\po}    {{\e ^-}}
\newcommand {\ee}    {{\e ^+ \e ^-}}
\newcommand {\fbody} {{\sto \to \b \chi {\rm f \bar{f}'}}}
\newcommand {\gaga}  {\gamma\gamma}
\newcommand {\ggqq}  {\gamma\gamma \rightarrow \q\overline{\q}}
\newcommand {\ggtt}  {\gamma\gamma \rightarrow \tau^{+}\tau^{-}}
\newcommand {\qqg}   {\q\overline{\q}\gamma}
\newcommand {\ttg}   {\tau^{+}\tau^{-}\gamma}
\newcommand {\wenu}  {{\rm We\nu_\e}}
\newcommand {\gsZ}   {\gamma^\star\mathrm{Z}}
\newcommand {\ggh}   {\gamma\gamma\rightarrow{\mathrm{hadrons}}}
\newcommand {\ZZg}   {\mathrm ZZ^{*}/\gamma^{*}}
\newcommand {\ZZ}    {{\mathrm ZZ}}
%
%---- VARIABILI
\newcommand {\zo}      {{z_0}}
\newcommand {\ip}      {{d_0}}
%\newcommand {\thr}     {{T_{\rm thrust}}}
\newcommand {\thr}     {{{\rm thrust}}}
\newcommand {\athr}    {{\hat{\rm a}_{\rm thrust}}}
\newcommand {\ththr}   {{\theta_{\rm thrust}}}
\newcommand {\acol}    {{\Phi_{\rm acol}}}
\newcommand {\acop}    {{\Phi_{\rm acop}}}
\newcommand {\acopt}   {{\Phi_{\rm acop_T}}}
\newcommand {\thpoint} {\theta_{\rm point}}
\newcommand {\thscat}  {\theta_{\rm scat}}
\newcommand {\etwelve} {E_{12\gradi}}
\newcommand {\ethirty} {E_{30\gradi}}
\newcommand {\eiso}[1] {E^{\, \triangleleft 30\gradi}_{#1}}
\newcommand {\phimiss} {{\phi_{\vec{p}_{\rm miss}}}}
\newcommand {\ewedge}  {E(\phi_{\vec{p}_{\rm miss}}\pm 15\gradi)}
%\newcommand {\ewedge}  {{E_{\rm w}}}
\newcommand {\evis}    {E_{\rm vis}}
\newcommand {\etot}    {E_{\rm vis}}
\newcommand {\emis}    {E_{\rm miss}}
\newcommand {\mvis}    {M_{\rm vis}}
\newcommand {\mtot}    {M_{\rm vis}}
\newcommand {\mmis}    {M_{\rm miss}}
\newcommand {\mhad}    {M^{\rm ex \, \ell_1}_{\rm vis}}
\newcommand {\mhadtwo} {M^{\rm ex \, \ell_1\ell_2}_{\rm vis}}
\newcommand {\ehad}    {E^{\rm NH}_{\rm vis}}
\newcommand {\epho}    {E^{\gamma}_{\rm vis}}
\newcommand {\echa}    {E^{\rm ch}_{\rm vis}}
\newcommand {\nch}     {{N_{\rm ch}}}
\newcommand {\elept}   {E_{\rm lept}}
\newcommand {\elepone} {E_{\ell _1}}
\newcommand {\eleptwo} {E_{\ell _2}}
\newcommand {\pvis}    {{\vec{p}_{\rm vis}}}
\newcommand {\pmis}    {{\vec{p}_{\rm miss}}}
\newcommand {\thmiss}  {{\theta_{\pmis}}}
\newcommand {\pt}      {{p_{\rm t}}}
\newcommand {\ptch}    {{p_{\rm t}^{\rm ch}}}
\newcommand {\pch}    {{p^{\rm ch}}}
\newcommand {\pz}      {{p_z}}
\newcommand {\ptnoNH}  {{p_{\rm t}^{\rm ex \, NH}}}
\newcommand {\puds}    {{P_{\rm uds}}}
%
\newcommand {\pmiss}   {{P\!\!\!\,\!/ }}
\newcommand {\emiss}   {{E\!\!\!\,\!/ }}
%
%
% no more of Christian's random capitalization!
% more of mine
\newcommand{\brchal}{\cal{B}($\PCha \rightarrow \ell\nu\PChi\ $)}
\newcommand{\M}{M_{2}}
\newcommand{\Mp}{M_{2}}
\newcommand{\sigbg}{\sigma_{\mathrm{bg}}}
\newcommand{\ww}   {\mathrm {WW}}
\newcommand{\zz}   {\mathrm Z\gamma^{*}}
\newcommand{\ewnu} {\mathrm{eW}\nu}
\newcommand{\eez}  {\mathrm {eeZ}}
\newcommand{\gagall}{{\gamma\gamma\rightarrow \ell\ell }}
\newcommand{\Pstaup}{{\widetilde{\tau}_{1}}}
\newcommand{\Pstaul}{{\widetilde{\tau}_{L}}}
\newcommand{\Pstaur}{{\widetilde{\tau}_{R}}}
\newcommand{\mzero}{m_{0}}
\newcommand{\msnu}{M_{\tilde{\nu}}}
\newcommand{\mcha}{M_{\chi^{\pm}}}
\newcommand{\mchi}{M_{\chi}}
\newcommand{\mstau}{M_{{\widetilde{\tau}_{1}}}}
\newcommand{\atau}{A_{\tau}}
\newcommand{\chsnu}{\PCha \rightarrow \ell \tilde{\nu}}
\newcommand{\chstau}{\PCha \rightarrow \tilde{\tau}_{1}\nu}
\newcommand{\chlep}{\PCha \rightarrow \ell\nu\chi}
\newcommand{\Tcsq}{\mathrm{TeV}/c^2}
% new for thesis
\newcommand{\nobs}{N_{\mathrm{obs}}}
\newcommand{\nlim}{N_{\mathrm{lim}}}
\newcommand{\Brl}{\cal{B}_{\ell}}
\newcommand{\leff} {\mathcal{L}_{\mathrm{eff}}}
\newcommand{\dedx}{{\mathrm{d}}E/{\mathrm{d}}x}
\newcommand{\chtau}{\PCha \rightarrow \tau\nu\chi}
\newcommand{\ssqtw}{\sin^{2}\theta_{\mathrm W}}
%\newcommand{\PSql}{\tilde{\mathrm q}_L}
%\newcommand{\PSqr}{\tilde{\mathrm q}_R}
%\newcommand{\PSq1}{\tilde{\mathrm q}_1}
%\newcommand{\PSq2}{\tilde{\mathrm q}_2}
%\newcommand{\ww}{{\mathrm WW}}
%\newcommand{\zz}{{\mathrm Z\gamma^{*}}}
%\newcommand{\eez}{{\mathrm eeZ}}
\newcommand{\nnz}{{\mathrm \nu\bar{\nu}Z}}
% added by bill
\def \ggll    {\gamma\gamma \rightarrow \ell^{+}{\ell}^{-}}
\def \tautau  {\mathrm \tau^{+}\tau^{-}}
\def \ffg  {f\bar{f}(\gamma)}
\def \lll   {\ell^{+}{\ell}^{-}}
\def \ww   {\mathrm WW}
\def \zz   {\mathrm Z\gamma^{*}}
\def \znn  {\mathrm Z\nu\nu}
\def \zee  {\mathrm Zee}
\def \rts  {\sqrt{s}}
\def \mstop {m_{\tilde{\mathrm{t}}}}
\def \msnu  {m_{\tilde{\nu}}}
\def \elow   {E_{12^{\circ}}}
\def \gev    { \, \mathrm{GeV}/\it{c}^{\mathrm{2}}}
\def \gvm    { \, \mathrm{GeV}/\it{c}}
\def \mx     {M_{\mathrm{eff}}} 
\newcommand{\neutr}{\chi}
%end fabio



%dalla mia pretesi

%\def \X             {\mathrm X} 
%\def \V             {\mathrm V} 
\def \Zcc           {\Z \to \charm \bar{\charm} }
\def \Zbb           {\Z \to \b \bar{\b} }
\def \decDS         {\D^{*+} \to \D^0 \pi^+}
\def \decsDS        {\D^{*+} \to \D^0 \pi^+_s}
\def \deckp         {\D^{0} \to \K^- \pi^+}
\def \deckppp       {\D^{0} \to \K^- \pi^+ \pi^+ \pi^-}
\def \deckpp        {\D^{0} \to \K^- \pi^+ \pi^0}
\def \deckpS        {\D^{0} \to \K^- \pi^+ (\pi^0)}
\def \decskp        {\D^{*+} \to \pi^{+}_{s} \K^- \pi^+}
\def \decskppp      {\D^{*+} \to \pi^{+}_{s} \K^- \pi^+ \pi^+ \pi^-}
\def \decskpp       {\D^{*+} \to \pi^{+}_{s} \K^- \pi^+ \pi^0}
\def \decskpS       {\D^{*+} \to \pi^{+}_{s} \K^- \pi^+ (\pi^0)}
\def \epsc          {\varepsilon_{\charm}}
\def \epsb          {\varepsilon_{\b}}
\def \pctod         {P_{\charm \to \D^*}}
\def \pbtod         {P_{\b \to \D^*}}
%\def \R             {{\mathrm R}}
\def \Gbb           {\Gamma_{\b\bar{\b}}}
\def \Gcc           {\Gamma_{\charm\bar{\charm}}}
\def \Gh            {\Gamma_{\mathrm h}}
%
% End of my macros
%

\def\centeron#1#2{{\setbox0=\hbox{#1}\setbox1=\hbox{#2}\ifdim
\wd1>\wd0\kern.5\wd1\kern-.5\wd0\fi
\copy0\kern-.5\wd0\kern-.5\wd1\copy1\ifdim\wd0>\wd1
\kern.5\wd0\kern-.5\wd1\fi}}
\def\ltap{\;\centeron{\raise.35ex\hbox{$<$}}{\lower.65ex\hbox{$\sim$}}\;}
\def\gtap{\;\centeron{\raise.35ex\hbox{$>$}}{\lower.65ex\hbox{$\sim$}}\;}
\def\gsim{\mathrel{\gtap}}
\def\lsim{\mathrel{\ltap}}

\begin{document}
\title{CMS reconstruction improvements for the tracking in large pile-up events}

\author{$^1$D~Giordano and $^2$G~Sguazzoni}

\address{$^1$CERN, Information Technology Department, Experiment Support Group, Geneva, Switzerland}
\address{$^2$INFN, Firenze, Italy}


\ead{domenico.giordano@cern.ch, giacomo.sguazzoni@cern.ch}

\begin{abstract}
The CMS tracking code is organized in several levels, known as
'iterative steps', each optimized to reconstruct a class of particle
trajectories, as the ones of particles originating from the primary
vertex or displaced tracks from particles resulting from secondary
vertices. Each iterative step consists of seeding, pattern recognition
and fitting by a kalman filter, and a final filtering and
cleaning. Each subsequent step works on hits not yet associated to a
reconstructed particle trajectory. The CMS tracking code underwent a
major upgrade needed to make the reconstruction computing load
compatible with the increasing instantaneous luminosity of LHC,
resulting in a large number of primary vertices and tracks per bunch
crossing. The iterative steps have been reorganized and optimized and
an iterative step specialized for the reconstruction of photon
conversion has been added. It is based on the innovative idea to use
an existing track to build up a custom seed in the conversion
hypothesis. For special event reconstruction applications, as the
particle flow algorithm, it is necessary to test the possible
association between a given reconstructed track and an energy deposit
in calorimeters (cluster). The implementation of a k-dimensional tree
in two dimensions allowed the combinatorics of links between tracks
and clusters to be reduced from N*N to N*log(N), where N is the number
of objects. The impact on reconstruction performances are promising
and the prospects for future applications are discussed.
\end{abstract}


\section{Introduction}
\label{introduction}

The Compact Muon Solenoid, CMS, is one of the two general-purpose
experiments installed at the Large Hadron Collider (LHC) at
CERN~\cite{cms}. The core of the CMS detector is
the superconducting solenoid, $6\m$ in diameter and $13\m$ long, that
produces a magnetic field of $3.8\T$. The solenoid contains, from
outside to inside, the calorimeter system and the silicon 
tracking system for the reconstruction of charged particles
trajectories.

CMS uses a right-handed coordinate system, with the origin at the
nominal interaction point, the $x$-axis pointing to the centre of the
LHC, the $y$-axis pointing up (perpendicular to the LHC plane), and
the $z$-axis along the anticlockwise-beam direction. The polar angle,
$\theta$, is measured from the positive $z$-axis and the azimuthal
angle, $\phi$, is measured in the $x$-$y$ plane. Pseudo-rapidity is
$\eta = -\log \tan \theta/2$.

The silicon tracking system, shown in figure~\ref{fig:tracker}, is composed of a Pixel
Silicon detector with three barrel layers at radii between $4.4\cm$
and $10.2\cm$ and two endcap disks at each end. Pixel sensors feature single pixel size
of $100\times150\um^2$ for a total of 66M channels.  
\begin{figure}[t]
%\begin{center}
%\includegraphics*[width=0.44\textwidth]{fig/sketch.pdf}
%\hskip 2mm
\includegraphics*[width=0.65\textwidth]{figs/layout_rz.pdf}\hspace{0.02\textwidth}%
\begin{minipage}[b]{0.33\textwidth}\caption{\label{fig:tracker}A simplified
    sketch of a quadrant of the $Rz$ section of the CMS Tracker (bold
    lines represent double sided module assemblies).}
\end{minipage}
%\end{center}
\end{figure}
The Silicon Strip Tracker
covers the radial range between $20\cm$ and $110\cm$ around the LHC
interaction point. The barrel region ($| z |  < 110\cm$) is split into
a Tracker Inner Barrel (TIB), made of four detector layers, and a
Tracker Outer Barrel (TOB), made of six detector layers. The TIB is
complemented by three Tracker Inner Disks 
per side (TID). The forward and backward
regions ($120\cm < |z| < 280\cm$) are covered by nine Tracker
End-Cap (TEC) disks per side
thus extending the overall acceptance to cover the region
$|\eta|<2.5$. In some of the layers and in the innermost 
rings, special double-sided modules
%(figure~\ref{fig:sst}, right panel)
are able to provide accurate three-dimensional position measurement of the charged
particle hits.  The Silicon Strip Tracker is the world's largest silicon
strip detector with a volume of approximately $23\m^3$, instrumented by about 15,000 modules
with different strip pitches ranging from 80 to $180\um$, for a total
of 198$\m^2$ of Silicon active area and about 9.6 million channels with full optical analog readout~\cite{cms}\cite{TkTDR}\cite{TkTDRadd}.
%The granularity was chosen to balance the need
%for a low occupancy, which is estimated to be a few percent at the
%largest expected LHC luminosity, and the requirement of minimising the
%power density and the amount of 
%material.

%The tracking detector features a 
%transverse momentum resolution of about 1-2\% for muons of
%$P_T\sim100\GeV$, an impact parameter resolution of about $10-20\um$
%for tracks with $P_T\sim10-20\GeV$, a reconstruction efficiency of
%tracks within jets of about 0.85-0.90 with a few percent fake rate.

%The operation of the CMS Pixel detector and of the CMS Silicon Tracker
%is described elsewhere~\cite{pxlop}~\cite{tkop}.

The CMS track reconstruction~\cite{trackreco} starts with the appropriate grouping of the hits
in the innermost layers to build up {\em seeds}. The seed is an initial track
estimate and consists of a triplet or a pair of hits, sufficient for a
basic prediction of the trajectory parameters if the primary vertex is
 also used. Starting from a given seed, pattern recognition using a Kalman Filter
is performed to build inside-out trajectories. Then each identified
track undergoes a procedure to reject possible outlier hits and is
refitted, also using a Kalman Filter. Finally, a quality selection is
performed. Reconstruction efficiency relies on several iterations (steps) of the
tracking procedure; every step, except the first, works on the
not-yet-associated hits surviving the previous step. Each step is
optimized with respect to the seeding topology and to the final
quality cuts. This recursive procedure is referred to as {\em
  iterative tracking}.

The algorithms that build up the CMS event reconstruction software, CMSSW~\cite{cmssw},
output physics objects (e.g., tracks, electrons, jets, ...) from the
raw data recorded by the detector. Ideally, all events collected by CMS are
reconstructed by the CMS prompt reconstruction system quasi real-time, soon after being
collected. The prompt reconstruction stream is of utmost
importance for a discovery experiment and prompt reconstruction data
samples have been the base of most of the CMS physics results so
far. Clearly, they are also crucial for fast and accurate feedback on detector
conditions.

The complexity and the granularity of the tracker system in
connection with the large LHC instantaneous luminosity, resulting in a
large number of interactions (pile-up, PU) per bunch crossing, make the
track reconstruction largely dominating the entire reconstruction
chain memory-wise and in terms of CPU time.

%In addition, a subset of the event reconstruction
%algorithms make up part of the high-level trigger system (HLT), which
%reduces the data recording rate from approximately 100 kHz of the Level-1
%trigger accepts down to a few hundred events/second of data written to
%tape rate. 

The LHC instantaneous luminosity is steadily growing since the
beginning of the run in 2010. In figure~\ref{fig:LHCinstLumi2011} and
figure~\ref{fig:LHCinstLumi2012} the maximum LHC instantaneous
luminosity per day delivered to CMS is shown as a function of time for 2011 and 
2012, respectively. During 2012 the LHC instantaneous luminosity is expected to
reach a value of about $7\cdot10^{33}\cm^{-2}\sec^{-1}$ that corresponds to a number
of primary vertices per bunch crossing of about 25.
\begin{figure}[t]
%\begin{center}
\includegraphics*[width=0.55\textwidth]{figs/peak_lumi_per_day_pp_2011.png}\hspace{0.0002\textwidth}%
\begin{minipage}[b]{0.33\textwidth}\caption{\label{fig:LHCinstLumi2011}Maximum
    instantaneous luminosity per day delivered to CMS for pp running at $7\TeV$ centre-of-mass energy in 2011.} 
\end{minipage}
%\vskip -5mm
%\end{center}
\end{figure}
\begin{figure}[b]
%\begin{center}
\includegraphics*[width=0.55\textwidth]{figs/peak_lumi_per_day_pp_2012.png}\hspace{0.0002\textwidth}%
\begin{minipage}[b]{0.33\textwidth}\caption{\label{fig:LHCinstLumi2012}Maximum
    instantaneous luminosity per day delivered to CMS for 
  pp running at $8\TeV$ centre-of-mass energy in 2012.} 
%\vskip -5mm
%\end{center}
\end{minipage}
\end{figure}

The other worsening factor that was not foreseen at the design level
of LHC and consequently not taken into account in writing first
versions of reconstruction software, is the bunch crossing
frequency. LHC was supposed to run with 25ns between subsequent bunch
crossings. But as a consequence of the LHC operating conditions and
beam optics, it is preferable to run with 50ns between bunch
crossings. This allows for fatter bunches in the machine and,
eventually, for a larger overall delivered luminosity. On the other
hand, this results into a larger pile-up per bunch crossing and
considerably increases the computing time per event that, being due to
combinatorics, scales faster than with purely linear dependence.
During 2011 it was soon clear that the current version of
reconstruction software was not performing well enough to withstand prompt
reconstruction for the last part of 2011 and 2012. 

This paper describes the actions that have been put in place to
improve the tracking reconstruction software, the major responsible of
the CPU load at the reconstruction level, in order to be compliant
with the expected luminosity in 2011 and 2012. This improvement
campaign took place in two phases: the first phase took place during
2011 and was completed in September 2011 (so-called {\em Fall 2011
  campaign}), the second phase started at the beginning of 2012 and
was completed in March 2012 ({\em Spring 2012 campaign}).

\section{Fall 2011}
\label{fall11}






\section{Summer 2012}
\label{summer12}






\section{A glimpse into the future}
\label{glimpse}

The challenge for the CMS reconstruction cannot be considered over
with the deployment of the software for 2012 data taking, currently
ongoing. After the first long shutdown, foreseen for almost two years
in 2013 and 2014, LHC will increase center-of-mass energy
and instantaneous luminosity as well. This will require a major
reengineering of the entire reconstruction software and of the
tracking.

Two major areas of improvements are being considered: implementation of
tracking techniques never used in CMS up to now (like Hough transforms
tracking); exploitation at any possible level of parallelization
techniques. The latter, in particular, turns out to be necessary to
better profit from the actual trend of increase of the computing power
that is realized by an increase of the number of cores in the same
monolithic CPU.

Parallelization can be implemented in several ways. At the level of
the framework by allowing different modules to run in parallel taking
appropriately into account all dependencies; this would be almost
transparent for the final user and developer, i.e. it would require
minor or no changes to user and reconstruction module code. Nevertheless
this is not optimal as some modules of CMSSW need much longer time
to run with respect to the others; track reconstruction is the
typical example. In this case it is worthwhile to implement
parallelization at the module and algorithm level. This requires code
modifications but is also much more effective. Prototype
implementations are already being studied and are very
promising~\cite{parallel}.






%\thispagestyle{empty}
\section{Conclusions}
\label{section_conclusions}

A first attempt of estimating the Tracker Material Budget using photon conversions with 2009 $\sqrt{s}=900\GeV$ CMS 
data has been completed.
Results are still qualitative since the statistics is too low. 
However, a good match with MC data is already visible in a region including the beam pipe and the two innermost pixel barrel layer.

The method here presented provides an uncalibrated measurement of the material, after correcting the number of conversions per unit volume 
for appropriate factors due to reconstruction efficiency, photon flux and geometrical effects. 

Current problems of this method are the low photon reconstruction efficiency at radii exceeding the first pixel barrel layer, the poor
conversion radius resolution and the lack of an absolute measurement scale.
The conversion reconstruction efficiency could be improved by retuning the track reconstruction parameters and 
by looking for possible constraints provided by the second photon from the pion decay 
(which may have reached the calorimeter or converted in a different part of the tracker).
The poor resolution can be improved by accounting for hits possibly shared by the two electron legs and properly take into account by means of 
unfolding techniques.
An absolute scale of the material estimation could be provided by the knowledge of the photon flux or by gauging the measurement 
counting the number of conversions in a volume of known radiation length.

Final remark is that the validity of the proposed method is not necessarily limited to converted photons and could be applied to
nuclear interactions as well.



%\newpage
\section*{References}

\begin{thebibliography}{99}
\bibitem{cms}
  Adolphi R et al. [CMS Collaboration] 2008 The CMS experiment at the CERN LHC
  {\em JINST} {\bf 3} S08004 
\bibitem{TkTDR} CMS Collaboration 1998 The Tracker System Project Technical Design Report
  {\em CERN-LHCC} 98-6
\bibitem{TkTDRadd} CMS Collaboration 2000 Addendum to the CMS Tracker TDR {\em
   CERN-LHCC} 2000-16
\bibitem{trackreco} CMS Collaboration 2010 Tracking and Vertexing Results from First Collisions
{\em CMS-PAS} TRK-10-001
\bibitem{cmssw} CMS Collaboration 2006 CMS Physics Technical Design
    Report Detector Performance and Software {\em
      CERN-LHCC} 2006-001
\bibitem{posterConv} Giordano D and Sguazzoni G [CMS
  Collaboration] 2012 An innovative seeding technique for photon conversion
  reconstruction at CMS {\em these proceedings}
\bibitem{pf1} CMS Collaboration 2009 Particle--Flow Event
  Reconstruction in CMS and Performance for Jets, Taus, and MET {\em CMS-PAS} PFT-09-001
\bibitem{pf2} CMS Collaboration 2010 Commissioning of the
  Particle-flow Event Reconstruction with the first LHC collisions
  recorded in the CMS detector {\em CMS-PAS} PFT-10-001
\bibitem{kdtree} Bentley J L 1975 Multidimensional binary search
    trees used for associative searching {\em Commun. ACM} {\bf 18} 9 (Sep. 1975) 509-517 
\bibitem{parallel} Hauth T, Innocente V and Piparo D [CMS
  Collaboration] 2012 Development and Evaluation of Vectorised and
    Multi-Core Event Reconstruction Algorithms within the CMS Software
    Framework {\em these proceedings}
\end{thebibliography}


\end{document}
