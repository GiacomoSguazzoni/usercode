%\thispagestyle{empty}
\section{Conclusions}
\label{conclusions}

The CMS track reconstruction has been recently streamlined to allow
for quasi real-time prompt reconstruction to be performed with the
available computing resources during 2011 and 2012 data taking,
despite the steady increase of instantaneous luminosity delivered by
the LHC thanks to its impressive performance.

This goal has been achieved by a number of actions taken at any level
of the track reconstruction workflow by means of smarter coding
techniques and by using technological improvements as well. The
overall gain can be approximately estimated in a factor 2 
reduction of memory load and a factor 7 reduction of CPU time for the
typical 2012 event.

The LHC operating conditions after the long shutdown, foreseen in
2013-2014, requires further performance improvements that will be
object of future activities. New tracking techniques and
parallelization will be implemented. Prototype applications are
already being investigated.
