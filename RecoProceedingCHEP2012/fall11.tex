\section{Fall 2011}
\label{fall11}

The modifications put in place and optimized in the first phase of the
improvement campaign have been developed on top of CMSSW version 4.2.x
and implemented in the CMSSW version 4.4.x. 

A first group of improvements are purely based on smarter coding
and better algorithm implementation and do not
change the physics outcome of the tracking reconstruction
workflow. They are mainly targeted to reduction and better handling of
the memory and in fact they allow for a 40\% cut of the
memory budget. More in detail, these modifications are described below.
\begin{description}
\item[Copy-less cluster masking within the iterative tracking.] Each
  step of the iterative tracking, but the first, works on the hits not
  yet associated to any track. Technically this was implemented by
  creating a new collection of surviving hits at each step. To save
  memory, a masking algorithm has been implemented adding to the hit
  object an appropriate data member for the masking bits. Results are
  unchanged with a major reduction of the allocated memory.
\item[Batch cleaning of track candidates.] The track candidate results
  from a seed that has been successfully propagated. Before being
  declared as a reconstructed track the track candidate must undergo
  a filtering selection to reject fakes. To
  avoid storing too much track candidates in memory the
  cleaning procedure is done once a subsample of 1000 track candidates
  has been accumulated with large benefit on the overall required memory.
\item[Efficient quality assignment.] Each step in
  the iterative tracking assigns tracks to a quality tier. Old
  implementation of the algorithm just created a copy of the same
  track per each quality tier it was belonging to; this has been
  modified by removing the copying and adding an appropriate data
  member to store the quality tier bits with an obvious advantage on
  the memory consumption.
\item[Efficient track merging.] After all iterative steps, the
  resulting track collections have to be merged and further cleaned up
  by potential fake tracks and duplicated tracks. In fact, only hits
  associated to tracks with highest quality, know in CMS as {\em high 
    purity tracks}, are not used in the following steps. But hits
  associated to lower quality tracks are retained, in the attempt to
  build better tracks out of them with different seed and propagation
  parameters. Old implementation of merging algorithm compared the
  collection created by the various steps in pairs creating intermediate
  collections to be further compared with other collections up to the
  end of the process. In the updated version all track collections
  feed a merging module that works without creating any intermediate
  collection. This is pictorially shown in figure~\ref{fig:merge}.
\end{description}
\begin{figure}[b]
\begin{center}
\includegraphics*[width=0.492\textwidth]{figs/oldMerge.png}
\hskip 0.5mm
\includegraphics*[width=0.492\textwidth]{figs/newMerge.png}
\caption{Schematical representation of the old 
  (left) and the new (right) merging algorithm for an hypothetical
  iterative tracking with five steps; ``intermediate'' track
  collections are avoided in the new
  algorithm and this allows for consistent memory savings.}
\label{fig:merge}
%\vskip -5mm
\end{center}
\end{figure}


The second group of ameliorations directly affects
the algorithms and thus the outcome on observables and have to
be evaluated also with respect to performances on physics. These
modifications target the CMSSW modules related to tracking that are
dominating, in terms of CPU time, the entire reconstruction chain and are
described in the following.
\begin{description}
\item[Iterative tracking.] A factor 2.5 of improvement in the CPU time
  has been obtained by optimizing the iterative tracking, as detailed
  in table~\ref{table:44xIterativeSteps} to be compared with
  table~\ref{table:42xIterativeSteps} that summarizes the baseline
  configuration of CMSSW 42x. As can be seen, the net effect is an
  increase of the effective $P_T$ threshold for track reconstruction
  together with tighter constraint on impact parameter. This
  configuration results into a reduced efficiency for $P_T$ lower than
  $300\MeVc$ but a larger efficiency for $P_T$ greater than
  $0.9\GeVc$.  
%
\begin{table}[b]
  \caption{  \label{table:42xIterativeSteps}Relevant parameters of the
    six tracking iterative steps in 
    CMSSW 42x, i.e. before the reconstruction improvement campaign
    described in this paper; $\sigma$ represent the beam spot size
    along the $z$ axis.}
  \centering
  \begin{tabular}{llllll}
\br
  \#step & seed type & seed subdetectors & $P^{\rm min}_T$ [$\GeVc$] &
  $d_0$ cut & $z_0$ cut \\ \mr
  0 & triplet & pixel & 0.8 & 0.2$\cm$ & 3.0$\sigma$ \\
  1 & pair    & pixel/TEC & 0.6 & 0.05$\cm$ & 0.6$\cm$ \\
  2 & triplet & pixel & 0.075 & 0.2$\cm$ & 3.3$\sigma$ \\
  3 & triplet & pixel/TIB/TID/TEC & 0.25-0.35 & 2.0$\cm$ & 10.0$\cm$ \\
  4 & pair    & TIB/TID/TEC & 0.5 & 2.0$\cm$ & 12.0$\cm$ \\
  5 & pair    & TOB/TEC & 0.6 & 6.0$\cm$ & 30.0$\cm$ \\
\br
\end{tabular}
\end{table}
%
\begin{table}[b]
  \caption{  \label{table:44xIterativeSteps} Relevant parameters of
    the seven tracking iterative steps in 
    CMSSW 44x, after the first phase of the improvement campaign in
    fall 2011; in bold the parameters changed with respect to the
    corresponding steps in CMSSW 42x (see
    table~\ref{table:42xIterativeSteps}); step \#1 is brand new with
    respect to CMSSW 42x; $\sigma$ represent the beam spot size
    along the $z$ axis.}
  \centering
  \begin{tabular}{llllll}
\br
  \#step & seed type & seed subdetectors & $P^{\rm min}_T$ [$\GeVc$] &
  $d_0$ cut & $z_0$ cut \\ \mr
  0 & triplet             & pixel & {\bf 0.6} & {\bf 0.03}$\cm$ & {\bf 4.0}$\sigma$ \\
  1 & {\bf triplet} & {\bf pixel} & {\bf 0.2} & {\bf 0.03}$\cm$ & {\bf 4.0}$\sigma$ \\
  2 & pair          & {\bf pixel} & 0.6       & {\bf 0.01}$\cm$ & {\bf 0.09}$\cm$ \\
  3 & triplet             & pixel & {\bf 0.2} & {\bf 1.0}$\cm$ & {\bf 4.0}$\sigma$ \\
  4 & triplet & pixel/TIB/TID/TEC & {\bf 0.35-0.5} & 2.0$\cm$ & 10.0$\cm$ \\
  5 & pair    & TIB/TID/TEC       & {\bf 0.6} & 2.0$\cm$ & {\bf 10.0}$\cm$ \\
  6 & pair    & TOB/TEC           & 0.6       & 2.0$\cm$ & 30.0$\cm$ \\
\br
\end{tabular}
\end{table}
%
\item[Reconstruction of photon conversions.] Reconstruction of photon
  conversion in the tracker volume results heavily affected by the
  higher $P_T$ threshold and by the tighter impact parameter cuts
  since conversion leg tracks are typically soft and displaced. To
  recover this loss, a dedicated seeding has been
  deployed~\cite{posterConv} and the photon conversion reconstruction
  has been further optimized resulting in a factor 12 improvement of
  the CPU time for conversion reconstruction.
\item[Reconstruction of primary vertices.] The reconstruction of
  primary vertices in the event has been optimized by integrating into
  the same module all the different reconstruction methods; the
  removal of the overhead due to the module split we had beforehand
  was enough to gain a factor two in CPU time in this specific context.
\item[Reconstruction of nuclear interactions.] Similarly to photon
  conversions, also nuclear interactions are reconstructed for tracker
  material studies and to correctly estimate the hadronic energy
  fraction in jets within the Particle Flow, a sophisticated event
  reconstruction that consists in reconstructing and identifying each
  single particle with an optimized combination of all subdetector
  information. To avoid consuming CPU
  time in the heavy vertex fit with candidates that are very likely to
  be fakes, a very simple preselection has been implemented: a nuclear
  interaction candidate track is kept only if $P_T$ exceeds
  $800\MeVc$, in case of primary tracks, or if $d_0$ is larger then
  $2\mm$ for secondary tracks; the vertex candidate must have at least
  three tracks (one primary and two secondaries or three secondary)
  and, finally, candidates are discarded if the
  secondary vertex estimate falls within the beam pipe. These simple
  criteria are enough to reduce the combinatorics such that the
  nuclear interaction reconstructions gains a factor 5 in CPU time with
  no sensible degradation in physics performances.
\begin{figure}[t]
\begin{center}
\includegraphics*[width=0.65\textwidth]{figs/kdtree.png}
\caption{Pictorial representation of the kd-tree algorithm in the case
  of a very simple neighbor search problem in the $(\eta, \phi)$
  plane: a track (represented by the ``$\times$'' symbol) needs to be
  associated to one of the calorimetric clusters represented by the
  dots labelled with letters. On the left panel it is shown the way
  $(\eta, \phi)$ is split into domains and the resulting navigation tree is
  sketched on the right.}
\label{fig:kdtree}
%\vskip -5mm
\end{center}
\end{figure}
\begin{figure}[b]
%\begin{center}
%\includegraphics*[width=0.44\textwidth]{fig/sketch.pdf}
%\hskip 2mm
\includegraphics*[width=0.55\textwidth]{figs/recotime.pdf}\hspace{0.02\textwidth}%
\begin{minipage}[b]{0.38\textwidth}\caption{\label{fig:fall11}Breakdown
    of the average CPU time per event in arbitrary units before and after 
  `Fall 2011' for each improvement area of the tracking reconstruction
  software (simulated QCD events with 30 pile-up interactions).}
\end{minipage}
%\vskip -5mm
%\end{center}
\end{figure}
\item[Particle flow links.] The Particle Flow algorithm needs to link
  tracks to calorimetric clusters in the $(\eta, \phi)$ parameter space. This
  problem of nearest neighbor search 
  over a large number of objects in CMSSW 42x is implemented with
  nested loops and results to be rather time intensive. Moreover the
  complexity scale quadratically ($N^2$) as the object multiplicity
  $N$ increases. In CMSSW 44x the well known linearization technique
  known as {\em kd-tree}~\cite{kdtree} has been introduced to replace nested
  loops. The method consists in an algorithm that, starting from a
  collection of objects (calorimetric cluster, for example),
  dynamically splits the $(\eta, \phi)$ space into appropriate domains,
  each containing one single object, organized in a tree. The closest
  cluster to a given track can be found by 
  exploring the $(\eta, \phi)$ space with a very fast binary search
  that ends up in the closest neighbor domain. This algorithm,
  schematically represented in figure~\ref{fig:kdtree}, has a complexity
  that scales as $N\cdot\log N$, thus more convenient with respect to
  standard nested loops especially for large multiplicity, and allows to
  gain a factor 4 in CPU time in this specific application. Its
  extension to other modules of CMSSW is being studied.
\end{description}
 The results of the improvements just described are graphically
represented in figure~\ref{fig:fall11} and figure~\ref{fig:fall11vsPU}. In the
former the breakdown of the CPU time for each improvement area before
and after each improvement is reported for simulated QCD events with
30 pile-up interactions per event; in the latter the total CPU time is
shown as a function of the number of pile-up interactions for simulated QCD events.
\begin{figure}[t]
%\begin{center}
%\includegraphics*[width=0.44\textwidth]{fig/sketch.pdf}
%\hskip 2mm
\includegraphics*[width=0.55\textwidth]{figs/recotime_vs_pu.pdf}\hspace{0.02\textwidth}%
\begin{minipage}[b]{0.38\textwidth}\caption{\label{fig:fall11vsPU}Total reconstruction CPU time per event (in arbitrary units)
  for as a function of pile-up events for simulated QCD
  events for the baseline CMSSW and the 
improved version.}
\end{minipage}
%\vskip -5mm
%\end{center}
\end{figure}

