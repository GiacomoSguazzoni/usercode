\section{Introduction}
\label{introduction}

The Compact Muon Solenoid, CMS, is one of the two general-purpose
experiments installed at the Large Hadron Collider (LHC) at
CERN~\cite{cms}. The core of the CMS detector is
the superconducting solenoid, $6\m$ in diameter and $13\m$ long, that
produces a magnetic field of $3.8\T$. The solenoid contains, from
outside to inside, the calorimeter system and the silicon 
tracking system for the reconstruction of charged particles
trajectories.
 
The silicon tracking system, shown in figure~\ref{fig:tracker}, is composed of a Pixel
Silicon detector with three barrel layers at radii between $4.4\cm$
and $10.2\cm$ and two endcap disks at each end. Pixel sensors feature single pixel size
of $100\times150\um^2$ for a total of 66M channels.  
\begin{figure}[t]
%\begin{center}
%\includegraphics*[width=0.44\textwidth]{fig/sketch.pdf}
%\hskip 2mm
\includegraphics*[width=0.65\textwidth]{figs/layout_rz.pdf}\hspace{0.02\textwidth}%
\begin{minipage}[b]{0.33\textwidth}\caption{\label{fig:tracker}A simplified
    sketch of a quadrant of the $Rz$ section of the CMS Tracker (bold
    lines represent double sided module assemblies).}
\end{minipage}
%\end{center}
\end{figure}
The Silicon Strip Tracker
covers the radial range between $20\cm$ and $110\cm$ around the LHC
interaction point. The barrel region ($| z |  < 110\cm$) is split into
a Tracker Inner Barrel (TIB), made of four detector layers, and a
Tracker Outer Barrel (TOB), made of six detector layers. The TIB is
complemented by three Tracker Inner Disks 
per side (TID). The forward and backward
regions ($120\cm < |z| < 280\cm$) are covered by nine Tracker
End-Cap (TEC) disks per side
thus extending the overall acceptance to cover the region
$|\eta|<2.5$. In some of the layers and in the innermost 
rings, special double-sided modules
%(figure~\ref{fig:sst}, right panel)
are able to provide accurate three-dimensional position measurement of the charged
particle hits.  The Silicon Strip Tracker is the world's largest silicon
strip detector with a volume of approximately $23\m^3$, instrumented by about 15,000 modules
with different strip pitches ranging from 80 to $180\um$, for a total
of 198$\m^2$ of Silicon active area and about 9.6 million channels with full optical analog readout~\cite{cms}\cite{TkTDR}\cite{TkTDRadd}.
%The granularity was chosen to balance the need
%for a low occupancy, which is estimated to be a few percent at the
%largest expected LHC luminosity, and the requirement of minimising the
%power density and the amount of 
%material.

%The tracking detector features a 
%transverse momentum resolution of about 1-2\% for muons of
%$P_T\sim100\GeV$, an impact parameter resolution of about $10-20\um$
%for tracks with $P_T\sim10-20\GeV$, a reconstruction efficiency of
%tracks within jets of about 0.85-0.90 with a few percent fake rate.

%The operation of the CMS Pixel detector and of the CMS Silicon Tracker
%is described elsewhere~\cite{pxlop}~\cite{tkop}.

The CMS track reconstruction~\cite{trackreco} starts with the appropriate grouping of the hits
in the innermost layers to build up {\em seeds}. The seed is an initial track
estimate and consists of a pair or a triplet of hits, sufficient for a
basic prediction of the trajectory parameters if the primary vertex is
 also used. Starting from a given seed, pattern recognition using a Kalman Filter
is performed to build inside-out trajectories. Then each identified
track undergoes a procedure to reject possible outlier hits and is
refitted, also using a Kalman Filter. Finally, a quality selection is
performed. Reconstruction efficiency relies on several iterations (steps) of the
tracking procedure; every step, except the first, works on the
not-yet-associated hits surviving the previous step. Each step is
optimized with respect to the seeding topology and to the final
quality cuts. This recursive procedure is referred to as {\em
  iterative tracking}.

The algorithms that build up the CMS event reconstruction software, CMSSW~\cite{cmssw},
output physics objects (e.g., tracks, electrons, jets, ...) from the
raw data recorded by the detector. Ideally, all events collected by CMS are
reconstructed by the CMS prompt reconstruction system quasi real-time, soon after being
collected. The prompt reconstruction stream is of masterpiece
importance for a discovery experiment and prompt reconstruction data
samples have been the base of most of the CMS physics results so
far. Clearly, they are also crucial for fast and accurate feedback on detector
conditions.

The complexity and the granularity of the tracker system in
connection with the large LHC instantaneous luminosity, resulting in a
large number of interactions (pile-up) per bunch crossing, make the
track reconstruction largely dominating the entire reconstruction
chain memory-wise and in terms of CPU time.

%In addition, a subset of the event reconstruction
%algorithms make up part of the high-level trigger system (HLT), which
%reduces the data recording rate from approximately 100 kHz of the Level-1
%trigger accepts down to a few hundred events/second of data written to
%tape rate. 

The LHC instantaneous luminosity is steadily growing since the
beginning of the run in 2010. In figure~\ref{fig:LHCinstLumi2011} and
figure~\ref{fig:LHCinstLumi2012} the maximum LHC instantaneous
luminosity per day delivered to CMS is shown as a function of time for 2011 and 
2012, respectively. During the 2012 the LHC instantaneous luminosity is expected to
reach a value of $7\cdot10^{33}\cm^{-2}\sec^{-1}$ that corresponds to a number
of primary vertices per bunch crossing of about 25.
\begin{figure}[t]
%\begin{center}
\includegraphics*[width=0.55\textwidth]{figs/lumipeak-2011.png}\hspace{0.02\textwidth}%
\begin{minipage}[b]{0.33\textwidth}\caption{\label{fig:LHCinstLumi2011}Maximum
    instantaneous luminosity per day delivered to CMS for pp running at $7\TeV$ centre-of-mass energy in 2011.} 
\end{minipage}
%\vskip -5mm
%\end{center}
\end{figure}
\begin{figure}[b]
%\begin{center}
\includegraphics*[width=0.55\textwidth]{figs/lumipeak-pp-2012.png}\hspace{0.02\textwidth}%
\begin{minipage}[b]{0.33\textwidth}\caption{\label{fig:LHCinstLumi2012}Maximum
    instantaneous luminosity per day delivered to CMS for 
  pp running at $8\TeV$ centre-of-mass energy in 2012.} 
%\vskip -5mm
%\end{center}
\end{minipage}
\end{figure}

The other worsening factor that was not foreseen at the design level
of LHC and consequently not taken into account in writing first
versions of reconstruction software, is the bunch crossing
frequency. LHC was supposed to run with 25ns between subsequent bunch
crossings. But as a consequence of the LHC operating conditions and
beam optics, it is preferable to run with 50ns between bunch
crossings. This allows for fatter bunches in the machine and,
eventually, for a larger overall delivered luminosity. On the other
hand, this results into a larger pileup per bunch crossing and
considerably increases the computing time per event that, being due to
combinatorics, scales faster than purely linear dependence.
During 2011 it was soon clear that the current version of
reconstruction software was not performing well enough to withstand prompt
reconstruction for last part of 2011 and 2012. 

This paper describes the actions that have been put in place to
improve the tracking reconstruction software, the major responsible of
the CPU load at the reconstruction level, in order to be compliant
with the expected luminosity in 2011 and 2012. This improvement
campaign took place in two phases: the first during the 2011 and
completed in September 2011 (Fall~2011), the second started at the
beginning of 2012 and completed in March 2012 (Spring~2012).
