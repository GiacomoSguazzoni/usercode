\section{Material Budget Estimation}
\label{correctionFactors}

The following approach to the material budget estimation is along the lines of ref.~\cite{steve}. 

The number of photon conversions $dN_{\rm conv}$ in a
small volume filled with a homogenous material is:
\begin{equation}
dN_{\rm conv} = dN_{\gamma} \frac{P}{X_0} dt .
\label{eq:1}
\end{equation}
where: $dN_\gamma$ is the number of impinging photons; $P$ is the energy dependent conversion probability per unit
radiation length ($P\sim 7/9$); $dt$ is the volume effective thickness (i.e. with respect to the photon arrival direction); $X_0$ is the radiation length (in $\cm$).

Given a portion $R^2 \sin \theta\, d\theta\, d\phi$ centered in $(R,\theta,\phi)$ of a homogenous spherical skin of 
thickness $dR$ in a spherical reference system $(R, \theta, \phi)$,
relation~(\ref{eq:1}) is:
\begin{equation}
dN_{\rm conv} = N_{\gamma}(R, \theta, \phi) \cdot R^2 \sin \theta \, d\theta\, d\phi \cdot \frac{P}{X_0} dR \;,
\label{eq:2}
\end{equation}
where $N_\gamma(R,\theta,\phi)$ is the photon flux impinging the surface element.

However in a particle physics experiment the most common geometrical
structure is a cylindrical skin for which cylindrical reference system
is more appropriate and in some case also cartesian reference system
is convenient.


% \noindent{\bf Pseudo-cylindrical reference system.} 
% In a pseudo-cylindrical
% reference system $(r,\theta,\phi)$ ($r=R\sin\theta$), the relation~(\ref{eq:2}) can be
% conveniently rewritten having care of using the appropriate Jacobian
% factor, i.e.
% \begin{equation}
%  dr\, d\theta\, d\phi = \left| \frac{dr\,d\theta\,d\phi}{dR\,d\theta\,d\phi}
%    \right| \cdot dR\,d\theta\,d\phi
% \label{eq:3}
% \end{equation}
% where
% \begin{equation}
% \begin{split}
% \frac{dr\,d\theta,d\phi}{dR\,d\theta\,d\phi}
% & = \det \left( \begin{array}{ccc}
% \nicefrac{dr}{dR}      & \nicefrac{dr}{d\theta}      & \nicefrac{dr}{d\phi} \\
% \nicefrac{d\theta}{dR} & \nicefrac{d\theta}{d\theta} & \nicefrac{d\theta}{d\phi} \\
% \nicefrac{d\phi}{dR}   & \nicefrac{d\phi}{d\theta}   & \nicefrac{d\phi}{d\phi}
% \end{array} \right) =\\
% & = \det \left( \begin{array}{ccc}
% \sin \theta & R\cos \theta & 0\\
%  0 & 1 & 0 \\
%  0 & 0 & 1 
% \end{array} \right)
% = \sin \theta .
% \end{split}
% \label{eq:4}
% \end{equation}
% Using~(\ref{eq:4}) in~(\ref{eq:2}) we get:
% \begin{equation}
% dN_{\rm conv} = N_{\gamma}(r, \theta, \phi) 
% \frac{P}{X_0} \frac{r^2}{\sin^2\theta} \,dr\,d\theta\,d\phi .
% \label{eq:5}
% \end{equation}

\noindent{\bf Cylindrical reference system.} 
In a cylindrical reference system $(r,z,\phi)$ ($r=R\sin\theta$, $z=R\cos\theta$) the relation~(\ref{eq:2}) can be
conveniently rewritten having care of using the appropriate Jacobian
factor, i.e.
\begin{equation}
 dr\,dz\,d\phi = \left| \frac{dr\, dz\, d\phi}{dR\,d\theta\,d\phi}
   \right| \cdot dR\,d\theta\,d\phi
\label{eq:3bis}
\end{equation}
where
\begin{equation}
\begin{split}
\frac{dr\,dz\,d\phi}{dR\,d\theta\, d\phi}
& = \det \left( \begin{array}{ccc}
\nicefrac{dr}{dR}      & \nicefrac{dr}{d\theta}      & \nicefrac{dr}{d\phi} \\
\nicefrac{dz}{dR}      & \nicefrac{dz}{d\theta}      & \nicefrac{dz}{d\phi} \\
\nicefrac{d\phi}{dR}   & \nicefrac{d\phi}{d\theta}   & \nicefrac{d\phi}{d\phi}
\end{array} \right) = \\
& = \det \left( \begin{array}{ccc}
\sin \theta & R\cos\theta  & 0\\
\cos\theta  & -R\sin\theta & 0 \\
 0 & 0 & 1 
\end{array} \right)
= R .
\end{split}
\label{eq:4bis}
\end{equation}
Using~(\ref{eq:4bis}) in~(\ref{eq:2}) we get:
\begin{equation}
dN_{\rm conv} = N_{\gamma}(r, z, \phi) 
\frac{P}{X_0} r\, \,dr\,dz\,d\phi .
\label{eq:5bis}
\end{equation}

\noindent{\bf Cartesian reference system.} 
Similarly, in a cartesian reference system $(x, y, z)$ ($x=R\sin\theta\cos\phi$, $y=R\sin\theta\sin\phi$, $z=R\cos\theta$)
\begin{equation}
 dx\,dy\,dz  = \left| \frac{dx\, dy\, dz}{dR\,d\theta\,d\phi}
   \right| \cdot dR\,d\theta\,d\phi
\label{eq:3tris}
\end{equation}
where
\begin{equation}
\begin{split}
\frac{dx\,dy\,dz}{dR\,d\theta\,d\phi}
& = \det \left( \begin{array}{ccc}
\nicefrac{dx}{dR}      & \nicefrac{dx}{d\theta}      & \nicefrac{dx}{d\phi} \\
\nicefrac{dy}{dR}      & \nicefrac{dy}{d\theta}      & \nicefrac{dy}{d\phi} \\
\nicefrac{dz}{dR}      & \nicefrac{dz}{d\theta}   & \nicefrac{dz}{d\phi}
\end{array} \right)
= \\
& = \det \left( \begin{array}{ccc}
\sin\theta\cos\phi & R\cos\theta\cos\phi & -R\sin\theta\sin\phi\\
\sin\theta\sin\phi & R\cos\theta\sin\phi &  R\sin\theta\cos\phi\\
\cos\theta & -R\sin\theta & 0 
\end{array} \right)
= \\
& = R^2 \sin \theta = \sqrt{x^2+y^2}\sqrt{x^2+y^2+z^2}.
\end{split}
\label{eq:4tris}
\end{equation}
Using~(\ref{eq:4bis}) in~(\ref{eq:2}) we get:
\begin{equation}
dN_{\rm conv} = N_{\gamma}(x, y, z) 
\frac{P}{X_0}
%\frac{\sqrt{x^2+y^2+z^2}}{\sqrt{x^2+y^2}}
 \,dx\,dy\,dz .
\label{eq:5tris}
\end{equation}

\noindent{\bf Photon flux.} 
A reasonable but approximate guess of the form of
$N_{\gamma}(R, \theta, \phi)$ in the pp collisions at LHC can be
inferred assuming the following:
\begin{itemize}
\item[a)] all photons are originating at the interaction point $(0, 0, 0)$;
\item[b)] all photons come from QCD events ($\pi_0$ decays);
\item[c)] the number of photons interacting with the material is negligible. 
\end{itemize}
All these three assumption are to some extent not true, c) especially,
but let's give them for granted for the present preliminary study.

%\clearpage

The dependence on the distance from the interaction point is easily inferred observing that, givem the above assumptions, 
the flux is the same in the same portion of solid angle $\delta \Omega=\sin \theta\delta\theta d\phi$:%$\d \Omega=\sin \theta\d\theta d\phi$: 
\begin{equation}
N_{\gamma}(R', \theta, \phi) R'^2 d\Omega = N_{\gamma}(R, \theta, \phi) R^2 d\Omega,
\label{eq:6pre}
\end{equation}
from which immediately follows that
\begin{equation}
N_{\gamma}(R) \propto \frac{1}{R^2} = \frac{\sin^2 \theta}{r^2} = \frac{1}{x^2+y^2+z^2} .
\label{eq:6}
\end{equation}

As a consequence of the cylindrical symmetry of pp interactions
$N_\gamma$ does not depend on $\phi$.

As far as the $\theta$ dependance is concerned, given that 
\begin{equation}
\eta = -\ln \left( \tan \nicefrac{\theta}{2} \right)
\label{eq:7}
\end{equation}
it follows that
\begin{equation}
\begin{split}
dN_\gamma & = N_\gamma(\eta) d\eta  =
N_\gamma(\eta)\frac{d\eta}{d\theta} d\theta  = N_\gamma(\eta)\frac{1+\tan^2\nicefrac[]{\theta}{2}
 }{2\tan\nicefrac{\theta}{2}}  d\theta =\\ 
 & =
 N_\gamma(\eta)\frac{1}{2\sin\nicefrac[]{\theta}{2}\cos\nicefrac[]{\theta}{2}}  d\theta  
 = N_\gamma(\eta)\frac{1}{\sin\theta}  d\theta .
\label{eq:8}
\end{split}
\end{equation}
Eq.~(\ref{eq:8}) allows to identify $N_\gamma(\theta)$ with $N_\gamma(\eta(\theta))/\sin{\theta}$.
%\begin{equation}
%\begin{split}
%dN_\gamma(\theta) & = \frac{dN_\gamma}{d\theta} d\theta  =
%\frac{dN_\gamma}{d\eta}\frac{d\eta}{d\theta}  d\theta  = %\frac{dN_\gamma}{d\eta}\frac{1+\tan^2\nicefrac[]{\theta}{2}
% }{2\tan\nicefrac{\theta}{2}}  d\theta =\\ 
% & =
% \frac{dN_\gamma}{d\eta}\frac{1}{2\sin\nicefrac[]{\theta}{2}\cos\nicefrac[]{\theta}{2}}  d\theta  
% = \frac{dN_\gamma}{d\eta}\frac{1}{\sin\theta}  d\theta .
%\label{eq:8}
%\end{split}
%\end{equation}
Since 'all' photons come from $\pi_0$'s that show an almost flat
distribution vs. $\eta$, also photon $\eta$-distribution is flat as
well and $N_\gamma(\eta)$ does not depend on $\eta$, thus:
\begin{equation}
N_\gamma(\theta) \propto \frac{1}{\sin \theta} = \frac{\sqrt{x^2+y^2+z^2}}{\sqrt{x^2+y^2}}.
\label{eq:9}
\end{equation}
Putting~(\ref{eq:6}) and~(\ref{eq:9}) together we get 
\begin{equation}
N_{\gamma} (R, \theta, \phi) = k \frac{1}{R^2\sin \theta} \, \, \, ,
\label{eq:10pre}
\end{equation}
% \begin{equation}
% N_{\gamma} (r, \theta, \phi) = k \frac{\sin \theta}{r^2} \,\,\, ,
% \label{eq:10}
% \end{equation}
\begin{equation}
N_{\gamma} (r, z, \phi) = k \frac{1}{r\sqrt{r^2+z^2}} \,\,\, ,
\label{eq:10bis}
\end{equation}
and
\begin{equation}
N_{\gamma} (x, y, z) = k \frac{1}{\sqrt{x^2+y^2}\sqrt{x^2+y^2+z^2}} \,\,\, ,
\label{eq:10tris}
\end{equation}
for spherical,
%pseudo-cylindrical, 
cylindrical and cartesian coordinates
respectively. In all cases $k$ is an appropriate dimensional factor
that accounts for all necessary constants.

\noindent{\bf Geometrical dependence of photon conversions.} 
Equations
%~(\ref{eq:5}) with~(\ref{eq:10}),
~(\ref{eq:5bis})
with~(\ref{eq:10bis}), and~(\ref{eq:5tris}) with~(\ref{eq:10tris}),
respectively, allow the following expressions for the number of photon
conversion to be written:
% \begin{equation}
% d N_{\rm conv} = k \frac{\sin \theta}{r^2} 
% \frac{P}{X_0} \frac{r^2}{\sin^2\theta}\, dr\, d\theta\, d\phi = k
% \frac{P}{X_0} \frac{1}{\sin\theta}\, dr\, d\theta\, d\phi ,
% \label{eq:11}
% \end{equation}
\begin{equation}
d N_{\rm conv} = k \frac{1}{r\sqrt{r^2+z^2}} 
\frac{P}{X_0} r dr\, dz\, d\phi = k
\frac{P}{X_0} \frac{1}{\sqrt{r^2+z^2}}\, dr\, dz\, d\phi ,
\label{eq:11bis}
\end{equation}
\begin{equation}
\begin{split}
d N_{\rm conv} & = k \frac{P}{X_0} \frac{1}{\sqrt{x^2+y^2}\sqrt{x^2+y^2+z^2}} \,dx\,dy\,dz .
\end{split}
\label{eq:11tris}
\end{equation}
After the appropriate integration,
%Eq.~(\ref{eq:11}),
Eq.~(\ref{eq:11bis}) and~(\ref{eq:11tris}) can be used to extract the
geometrical factor to translate the observed number of conversion in a given 
volume into an estimate of $P/X_0$ (for the moment let's assume that
the conversion reconstruction efficiency is 1 with no background).

Few relevant examples follow.

\begin{description}
% \item[$r_1<r<r_2$,~$\, \theta_1<\theta<\theta_2$,~$\, 0<\phi<2\pi$;]
%   from~Eq.~(\ref{eq:11}): 
% \begin{equation}
% \begin{split}
% N_{\rm conv} & = k \frac{P}{X_0} \int_{\theta_1}^{\theta_2}
% \frac{d\theta}{\sin\theta} \int_{r_1}^{r_2}dr \int_0^{2\pi}d\phi = \\
% & = 2\pi k \frac{P}{X_0} (r_2-r_1) \cdot \left. \ln \tan
%     \frac{\theta}{2} \right|_{\theta_1}^{\theta_2} =\\
% & = 2\pi k \frac{P}{X_0} (r_2-r_1) (\eta_1 - \eta_2).
% \end{split}
% \label{eq:12}
% \end{equation}
\item[$r_1<r<r_2$,~$\, z_1<z<z_2$,~$\, 0<\phi<2\pi$;]
  from~Eq.~(\ref{eq:11bis}): 
\begin{equation}
\begin{split}
N_{\rm conv} & = k
\frac{P}{X_0} \int_{r_1}^{r_2} \int_{z_1}^{z_2} \frac{dr \, dz}{\sqrt{r^2+z^2}}
\int_0^{2\pi} d\phi = \\
& = 2\pi k \frac{P}{X_0} \int_{r_1}^{r_2} dr \left. \ln \left( 2\sqrt{r^2+z^2}+2z \right) \right|_{z_1}^{z_2} = \\
& = 2\pi k \frac{P}{X_0} \int_{r_1}^{r_2} dr \ln \frac{\sqrt{r^2+{z_2}^2}+z_2}{\sqrt{r^2+{z_1}^2}+{z_1}} = \\ 
& = 2\pi k \frac{P}{X_0} \left[ r \ln \frac{\sqrt{r^2+{z_2}^2}+z_2}{\sqrt{r^2+{z_1}^2}+{z_1}} + \right. \\
& - z_1 \ln \left( 2\sqrt{r^2+{z_1}^2}+r \right)  + \Biggl. \Biggl. z_2 \ln \left( 2\sqrt{r^2+{z_2}^2}+r \right) \Biggr] \Biggr|_{r_1}^{r_2} = \\
& = 2\pi k \frac{P}{X_0} \left[ r_2 \ln \frac{\sqrt{{r_2}^2+{z_2}^2}+z_2}{\sqrt{{r_2}^2+{z_1}^2}+z_1} - r_1 \ln \frac{\sqrt{{r_1}^2+{z_2}^2}+z_2}{\sqrt{{r_1}^2+{z_1}^2}+z_1} + \right. \\
& \left. + z_2 \ln \frac{\sqrt{{r_2}^2+{z_2}^2}+r_2}{\sqrt{{r_1}^2+{z_2}^2}+r_1} - z_1 \ln \frac{\sqrt{{r_2}^2+{z_1}^2}+r_2}{\sqrt{{r_1}^2+{z_1}^2}+r_1} \right] \\
& \equiv k \frac{P}{X_0} f^{r, z}_{\rm geom}.\\
\end{split}
\label{eq:13}
\end{equation}
\item[$x_1<x<x_2$,~$\, y_1<y<y_2$,~$\, z_1<z<z_2$;]
  from~Eq.~(\ref{eq:11tris}): 
\begin{equation}
\begin{split}
N_{\rm conv} & = k
\frac{P}{X_0} \int_{x_1}^{x_2} \int_{y_1}^{y_2} \int_{z_1}^{z_2}  \frac{dx\,
  dy \, dz}{\sqrt{x^2+y^2}\sqrt{x^2+y^2+z^2}}. 
\end{split}
\label{eq:14}
\end{equation}
The analytical solution of the integral does not exist~\cite{integrals}. Within
our needs (i.e. for a scatter plot in the $xy$ plane) we can integrate first in $dz$:
\begin{equation}
\begin{split}
N_{\rm conv} & = k
\frac{P}{X_0} \int_{x_1}^{x_2} \int_{y_1}^{y_2} dx \, dy \left. \frac{1}{r} \ln \left( 2\sqrt{r^2+z^2} + 2z \right) \right|_{z_1}^{z_2} \\
& = k \frac{P}{X_0} \int_{x_1}^{x_2} \int_{y_1}^{y_2} dx \, dy  \ln \frac{\sqrt{r^2+{z_2}^2} + z_2}{\sqrt{r^2+{z_1}^2} + z_1} ,
\end{split}
\label{eq:15pre}
\end{equation}
where $r=\sqrt{x^2+y^2}$. If $r\gg |x_2-x_1|$ and $r\gg |y_2
- y_1|$, a likely sufficient approximate
solution is
\begin{equation}
\begin{split}
N_{\rm conv} & = k
\frac{P}{X_0} \Delta x\, \Delta y\,  \ln \frac{\sqrt{\overline{r}^2+{z_2}^2} + z_2}{\sqrt{\overline{r}^2+{z_1}^2} + z_1} \\
& \equiv k \frac{P}{X_0} f^{x, y}_{\rm geom}
\end{split}
\label{eq:15}
\end{equation}
where $\overline{r}$ is the ``average'' radius, i.e. the one taken on
the middle of the ``box'', $\overline{r}=\nicefrac{1}{2}\sqrt{(x_1+x_2)^2+(y_1+y_2)^2}$.
%$\overline{r}=\sqrt{\nicefrac{(x_1+x_2)^2}{4}+\nicefrac{(y_1+y_2)^2}{4}}$.
\end{description}

%\end{document}










  