%\thispagestyle{empty}
\section{Conclusions}
\label{section_conclusions}

A first attempt of estimating the Tracker Material Budget using photon conversions with 2009 $\sqrt{s}=900\GeV$ CMS 
data has been completed.
Results are still qualitative since the statistics is too low. 
However, a good match with MC data is already visible in a region including the beam pipe and the two innermost pixel barrel layer.

The method here presented provides an uncalibrated measurement of the material, after correcting the number of conversions per unit volume 
for appropriate factors due to reconstruction efficiency, photon flux and geometrical effects. 

Current problems of this method are the low photon reconstruction efficiency at radii exceeding the first pixel barrel layer, the poor
conversion radius resolution and the lack of an absolute measurement scale.
The conversion reconstruction efficiency could be improved by retuning the track reconstruction parameters and 
by looking for possible constraints provided by the second photon from the pion decay 
(which may have reached the calorimeter or converted in a different part of the tracker).
The poor resolution can be improved by accounting for hits possibly shared by the two electron legs and properly take into account by means of 
unfolding techniques.
An absolute scale of the material estimation could be provided by the knowledge of the photon flux or by gauging the measurement 
counting the number of conversions in a volume of known radiation length.

Final remark is that the validity of the proposed method is not necessarily limited to converted photons and could be applied to
nuclear interactions as well.

