%%%%%%%%%%%%%%%%%%%%%%%%%%%%%%%%%%%%%%%%%%%%%%%%%%%%%%%%%%%%%%%%%%%%
%
%   Style for CMS Computing / Physics Technical Design Reports
%
%   Lucas Taylor  4 Feb 2005,   Revised  12 Oct 2005
%
%%%%%%%%%%%%%%%%%%%%%%%%%%%%%%%%%%%%%%%%%%%%%%%%%%%%%%%%%%%%%%%%%%%%

%  the following line is edited by the tdr script to change or to pass
%  additional options:
\documentclass{cmspaper}
\usepackage{slashed}
\usepackage{subfigure}
%\usepackage{rotating}
\usepackage{multirow}
\usepackage{amsmath}
\usepackage{graphicx}
\usepackage{units}
\setkeys{Gin}{width=\linewidth,totalheight=\textheight,keepaspectratio}
\graphicspath{{fig/}}

%\usepackage{amsmath}
%\usepackage{bm}
%\usepackage{float}
%\usepackage{axodraw}
%\usepackage{sparticles} 	%Package for displaying sparticle names. 
%\usepackage{feynmf}		%Package for feynman diagrams. 
\def\centeron#1#2{{\setbox0=\hbox{#1}\setbox1=\hbox{#2}\ifdim
\wd1>\wd0\kern.5\wd1\kern-.5\wd0\fi
\copy0\kern-.5\wd0\kern-.5\wd1\copy1\ifdim\wd0>\wd1
\kern.5\wd0\kern-.5\wd1\fi}}
\def\ltap{\;\centeron{\raise.35ex\hbox{$<$}}{\lower.65ex\hbox{$\sim$}}\;}
\def\gtap{\;\centeron{\raise.35ex\hbox{$>$}}{\lower.65ex\hbox{$\sim$}}\;}
\def\gsim{\mathrel{\gtap}}
\def\lsim{\mathrel{\ltap}}

%%%%%%%%%%%%%%%%%%%%%%%%%%%%%%%%%%%%%%%%%%%%%%%%%%%%%%%%%%%%%%%%%%%%

%
% My Macros
%
\usepackage{graphicx}
%\usepackage{drftcite}
\usepackage{pstricks}
\usepackage[figuresright]{rotating}
%
%
% Macro declarations
%
%---- SLASH
\def\slasha#1{#1\hskip-0.65em /}  %slasha per caratteri piccoli
\def\slashb#1{#1\hskip-1.3em /}   %slashb per quelli grandi
\def\slashc#1{#1\hskip-.4em /}
%
%---- UNITA` DI MISURA
\def \pb        {{\rm \, pb}}
\def \fb        {{\rm \, fb}}
\def \ipb       {{\rm \, pb^{-1}}}
\def \ifb       {{\rm \, fb^{-1}}}
\def \eV        {{\rm \,  eV}}
\def \keV       {{\rm \, keV}}
\def \MeV       {{\rm \, MeV}}
\def \GeV       {{\rm \, GeV}}
\def \TeV       {{\rm \, TeV}}
\def \TeVc      {\TeV/c}
\def \TeVcc     {\TeV/c^2}
\def \GeVc      {\GeV/c}
\def \GeVcc     {\GeV/c^2}
\def \MeVc      {\MeV/c}
\def \MeVcc     {\MeV/c^2}
%
%---- SIMBOLI
\def\ga{\mathrel{\raise.3ex\hbox{$>$\kern-.75em\lower1ex\hbox{$\sim$}}}}
\def\la{\mathrel{\raise.3ex\hbox{$<$\kern-.75em\lower1ex\hbox{$\sim$}}}}
\newcommand {\lesssim}
     {\,\raisebox{-0.6ex}{$\stackrel{\textstyle<}{\textstyle\sim}$}\,}
\newcommand {\gtrsim}
     {\,\raisebox{-0.6ex}{$\stackrel{\textstyle>}{\textstyle\sim}$}\,}
\newcommand{\ckm}{$\checkmark$}
%
%---- MISCELLANEA
%\newcommand {\slashed}[1] { \mbox{\rlap{\hbox{/}} #1 }}
\newcommand {\onehalf}    {\raisebox{0.1ex}{${\frac{1}{2}}$}}
\newcommand {\fivethirds} {\raisebox{0.1ex}{${\frac{5}{3}}$}}
\newcommand {\OR}         {{\tt OR}\,}
\newcommand {\BR}         {{\rm BR}\,}
\newcommand {\rts}        {\sqrt{s}}
\newcommand {\lumi}       {\mathcal{L}}
\newcommand {\Lumi}       {\int\lumi\mathrm{d}t}
\newcommand {\gradi}    {^\circ}
\newcommand {\de}         {\partial}
\newcommand {\um}         {\, \mu \rm m}
\newcommand {\nm}         {\rm \, nm}
\newcommand {\us}         {\, \mu \rm s}
\newcommand {\cm}         {\rm \, cm}
\newcommand {\mm}         {\rm \, mm}
\newcommand {\m}          {\rm \, m}
\newcommand {\km}         {\rm \, km}
\newcommand {\V}          {\rm \, V}
\newcommand {\T}          {\rm \, T}
\newcommand {\kV}         {\rm \, kV}
\newcommand {\kVm}        {\rm \, kV\! / \! m} 
\newcommand {\MVm}        {\rm \, MV\! / \! m} 
\newcommand {\ns}         {\rm \, ns} 
\newcommand {\ps}         {\rm \, ps} 
%
%---- THEORY groups & AOB
\newcommand {\gws}        {\mathrm{SU(2)_L \otimes U(1)_Y}}
\newcommand {\sul}        {\mathrm{SU(2)_L}}
\newcommand {\suc}        {\mathrm{SU(3)_C}}
\newcommand {\ul}         {\mathrm{U(1)_Y}}
\newcommand {\uem}        {\mathrm{U(1)_{em}}}
\newcommand {\sigmabar}   {\overline{\sigma}}
\newcommand {\gmunu}      {g^{\mu \nu}}
\newcommand {\munu}       {{\mu \nu}}
\newcommand {\obra}       {\langle 0 |}
\newcommand {\oket}       {| 0 \rangle}
%
%---- THEORY lepton fields
\newcommand {\LL}         {L^{\alpha}_{\mathrm L}}
\newcommand {\LLd}        {L^{\dagger \alpha}_{\mathrm L}}
\newcommand {\lL}         {\ell^{\alpha}_{\mathrm L}}
\newcommand {\lLd}        {\ell^{\dagger \alpha}_{\mathrm L}}
\newcommand {\ld}         {\ell^{\dagger \alpha}}
\newcommand {\lb}         {\overline{\ell}^{\alpha}}
\newcommand {\lR}         {\ell^{\alpha}_{\mathrm R}}
\newcommand {\lRd}        {\ell^{\dagger \alpha}_{\mathrm R}}
\newcommand {\nuL}        {\nu^{\alpha}_{\mathrm L}}
\newcommand {\nuLb}       {\overline{\nu}^{\alpha}_{\mathrm L}}
\newcommand {\nub}        {\overline{\nu}^{\alpha}}
\newcommand {\lept}       {\ell^\alpha}
\newcommand {\neut}       {\nu^{\alpha}}
\newcommand {\nuLd}       {\nu^{\dagger \alpha}_{\mathrm L}}
\newcommand {\Phid}       {\Phi^\dagger}
%
%---- THEORY quark fields
\newcommand {\up}         {u^{\alpha}}
\newcommand {\ub}         {\overline{u}^{\alpha}}
\newcommand {\down}       {d^{\alpha}}
\newcommand {\db}         {\overline{d}^{\alpha}}
\newcommand {\QL}         {Q^{\alpha}_{\mathrm L}}
\newcommand {\QLd}        {Q^{\dagger \alpha}_{\mathrm L}}
\newcommand {\UL}         {U^{\alpha}_{\mathrm L}}
\newcommand {\ULd}        {U^{\dagger \alpha}_{\mathrm L}}
\newcommand {\UR}         {U^{\alpha}_{\mathrm R}}
\newcommand {\URd}        {U^{\dagger \alpha}_{\mathrm R}}
\newcommand {\DL}         {D^{\alpha}_{\mathrm L}}
\newcommand {\DLd}        {D^{\dagger \alpha}_{\mathrm L}}
\newcommand {\DR}         {D^{\alpha}_{\mathrm R}}
\newcommand {\DRd}        {D^{\dagger \alpha}_{\mathrm R}}
\newcommand {\bfell}      {\ell\kern-0.4em
                           \ell\kern-0.4em
                           \ell\kern-0.4em
                           \ell }
\newcommand {\obfell}     {\overline{\ell}\kern-0.4em
                           \overline{\ell}\kern-0.4em
                           \overline{\ell}\kern-0.4em
                           \overline{\ell}}
\newcommand {\bfH}      {\, {\cal H}\kern-0.5em \kern-0.4em
                           {\cal H}\kern-0.5em \kern-0.4em
                           {\cal H}\kern0.1em }
\newcommand {\obfH}     {\, \overline{\cal H}\kern-0.5em \kern-0.4em 
                           \overline{\cal H}\kern-0.5em \kern-0.4em 
                           \overline{\cal H}\kern0.1em }
%
%---- PARTICELLE
\def \b             {{\mathrm b}}
\def \t             {{\mathrm t}}
\def \charm         {{\mathrm c}}
\def \d             {{\mathrm d}}
\def \u             {{\mathrm u}}
\def \e             {{\mathrm e}}
\def \q             {{\mathrm q}}
\def \g             {{\mathrm g}}
\def \p             {{\mathrm p}}
\def \s             {{\mathrm s}}
\def \n             {{\mathrm n}}
\def \h             {{\mathrm h}}
\def \l             {\ell} 
\def \f             {{\mathrm f}} 
%\def \f             {{f}} 
\def \A             {{\mathrm A}}
\def \B             {{\mathrm B}}
\def \D             {{\mathrm D}}
\def \K             {{\mathrm K}}
\def \X             {{\mathrm X}}
\def \Y             {{\mathrm Y}}
\def \W             {{\mathrm W}}
\def \H             {{\mathrm H}}
\def \Z             {{\mathrm Z}}
\def \S             {{\mathrm S}}
\def \N             {{\mathrm N}}
\def \L             {{\mathrm L}}
\def \R             {{\mathrm R}}
\def \P             {{\mathrm P}}
\def \G             {{\mathrm G}}
%
%---- Higgs
\newcommand {\ho}         {{\h^0}}
\newcommand {\Ho}         {{\H^0}}
\newcommand {\Ao}         {{\A^0}}
\newcommand {\Hpm}        {{\H^\pm}}
\newcommand {\clsb}       {{\mathrm CL_{\rm s+b}}}
\newcommand {\clb}        {{\mathrm CL_{\rm b}}}
%
%---- SUSY
\newcommand {\dm}         {\Delta m}
\newcommand {\dM}         {\Delta M}
\newcommand {\ldm}        {\mbox{``low $\dm$''}}
\newcommand {\hdm}        {\mbox{``high $\dm$''}}
\newcommand {\nnc}        {{\overline{\mathrm N}_{95}}}
\newcommand {\snc}        {{\overline{\sigma}_{95}}}
\newcommand {\susy}       {{supersymmetry}}
\newcommand {\susyc}      {{supersymmetric}}
\newcommand {\aj}         {\mbox{\sf AJ}}
\newcommand {\ajl}        {\mbox{\sf AJL}}
\newcommand {\llh}        {\mbox{\sf LLH}}
%
%---- SPARTICELLE
\newcommand {\rpc}     {{\rm RPC}}
\newcommand {\rpv}     {{\rm RPV}}
\newcommand {\sfe}     {{\tilde{\f}}}
\newcommand {\sfL}     {{\tilde{\f}_{\mathrm L}}}
\newcommand {\sfR}     {{\tilde{\f}_{\mathrm R}}}
\newcommand {\sfone}   {{\tilde{\f}_{1}}}
\newcommand {\sftwo}   {{\tilde{\f}_{2}}}
\newcommand {\sneu}    {{\tilde{\nu}}}
\newcommand {\wino}    {{\mathrm{\widetilde{W}}}}
\newcommand {\bino}    {{\mathrm{\widetilde{B}}}}
\newcommand {\se}      {{\mathrm{\tilde{e}}}}
\newcommand {\seR}     {{\mathrm{\tilde{e}_{R}}}}
\newcommand {\seL}     {{\mathrm{\tilde{e}_{L}}}}
\newcommand {\st}      {{\mathrm{\tilde{\tau}}}}
\newcommand {\stR}     {{\mathrm{\tilde{\tau}_{R}}}}
\newcommand {\stL}     {{\mathrm{\tilde{\tau}_{L}}}}
\newcommand {\stone}   {{\mathrm{\tilde{\tau}_{1}}}}
\newcommand {\sttwo}   {{\mathrm{\tilde{\tau}_{2}}}}
\newcommand {\sm}      {{\mathrm{\tilde{\mu}}}}
\newcommand {\smR}     {{\mathrm{\tilde{\mu}_{R}}}}
\newcommand {\smL}     {{\mathrm{\tilde{\mu}_{L}}}}
\newcommand {\Sup}     {{\mathrm{\tilde{u}}}}
\newcommand {\suR}     {{\mathrm{\tilde{u}_{R}}}}
\newcommand {\suL}     {{\mathrm{\tilde{u}_{L}}}}
\newcommand {\sdo}     {{\mathrm{\tilde{d}}}}
\newcommand {\sdR}     {{\mathrm{\tilde{d}_{R}}}}
\newcommand {\sdL}     {{\mathrm{\tilde{d}_{L}}}}
\newcommand {\sch}     {{\mathrm{\tilde{c}}}}
\newcommand {\scR}     {{\mathrm{\tilde{c}_{R}}}}
\newcommand {\scL}     {{\mathrm{\tilde{c}_{L}}}}
\newcommand {\sst}     {{\mathrm{\tilde{s}}}}
\newcommand {\ssR}     {{\mathrm{\tilde{s}_{R}}}}
\newcommand {\ssL}     {{\mathrm{\tilde{s}_{L}}}}
\newcommand {\stopR}   {{\tilde{\mathrm{t}}_{R}}}
\newcommand {\stopL}   {{\tilde{\mathrm{t}}_{L}}}
\newcommand {\stopone} {{\tilde{\mathrm{t}}_{1}}}
\newcommand {\stoptwo} {{\mathrm{\tilde{t}_{2}}}}
\newcommand {\sto}     {{\tilde{\mathrm{t}}}}
\newcommand {\SQ}      {{\mathrm{\widetilde{Q}}}}
\newcommand {\STO}     {{\mathrm{\widetilde{T}}}}
\newcommand {\glu}     {{\mathrm{\tilde{g}}}}
\newcommand {\sbotR}   {{\mathrm{\tilde{b}_{R}}}}
\newcommand {\sbotL}   {{\mathrm{\tilde{b}_{L}}}}
\newcommand {\sbotone} {{\mathrm{\tilde{b}_{1}}}}
\newcommand {\sbottwo} {{\mathrm{\tilde{b}_{2}}}}
\newcommand {\sbot}    {{\tilde{\mathrm{b}}}}
\newcommand {\squa}    {{\tilde{\mathrm{q}}}}
\newcommand {\squal}   {{\tilde{\mathrm{q}}_{\rm L}}}
\newcommand {\squar}   {{\tilde{\mathrm{q}}_{\rm R}}}
\newcommand {\sqL}     {{\tilde{\mathrm{q}}_{\rm L}}}
\newcommand {\sqR}     {{\tilde{\mathrm{q}}_{\rm R}}}
\newcommand {\snu}     {{\tilde{\nu}}}
\newcommand {\snue}    {{\tilde{\nu}_{\mathrm e}}}
\newcommand {\snum}    {{\tilde{\nu}_{\mu}}}
\newcommand {\snut}    {{\tilde{\nu}_{\tau}}}
\newcommand {\neu}     {{\chi}}
\newcommand {\chap}    {{\chi^+}}
\newcommand {\cham}    {{\chi^-}}
\newcommand {\chapm}   {{\chi^\pm}}

%
%---- SUSY PARAMETRI
\newcommand {\thstop} {\mathrm{\theta_{\tilde{t}}}}
\newcommand {\thsbot} {\mathrm{\theta_{\tilde{b}}}}
\newcommand {\thsqua} {\mathrm{\theta_{\tilde{q}}}}
\newcommand {\Mcha}{M_{\chi^\pm}}
\newcommand {\Mchi}{M_\chi}
\newcommand {\Msnu}{M_{\tilde{\nu}}}
\newcommand {\tanb}{\tan\beta}
%
%---- ABBREVIAZIONI

%
%---- PROCESSI FISICI
\newcommand {\rb}    {{\rm R_{\b}}}
\newcommand {\qq}    {{\q \overline{\q}}}
\newcommand {\bb}    {{\b \overline{\b}}}
\newcommand {\cc}    {{\charm \overline{\charm}}}
\newcommand {\ff}    {{\f \overline{\f}}}
\newcommand {\el}    {{\e ^+}}
\newcommand {\po}    {{\e ^-}}
\newcommand {\ee}    {{\e ^+ \e ^-}}
\newcommand {\fbody} {{\sto \to \b \chi {\rm f \bar{f}'}}}
\newcommand {\gaga}  {\gamma\gamma}
\newcommand {\ggqq}  {\gamma\gamma \rightarrow \q\overline{\q}}
\newcommand {\ggtt}  {\gamma\gamma \rightarrow \tau^{+}\tau^{-}}
\newcommand {\qqg}   {\q\overline{\q}\gamma}
\newcommand {\ttg}   {\tau^{+}\tau^{-}\gamma}
\newcommand {\wenu}  {{\rm We\nu_\e}}
\newcommand {\gsZ}   {\gamma^\star\mathrm{Z}}
\newcommand {\ggh}   {\gamma\gamma\rightarrow{\mathrm{hadrons}}}
\newcommand {\ZZg}   {\mathrm ZZ^{*}/\gamma^{*}}
\newcommand {\ZZ}    {{\mathrm ZZ}}
%
%---- VARIABILI
\newcommand {\zo}      {{z_0}}
\newcommand {\ip}      {{d_0}}
%\newcommand {\thr}     {{T_{\rm thrust}}}
\newcommand {\thr}     {{{\rm thrust}}}
\newcommand {\athr}    {{\hat{\rm a}_{\rm thrust}}}
\newcommand {\ththr}   {{\theta_{\rm thrust}}}
\newcommand {\acol}    {{\Phi_{\rm acol}}}
\newcommand {\acop}    {{\Phi_{\rm acop}}}
\newcommand {\acopt}   {{\Phi_{\rm acop_T}}}
\newcommand {\thpoint} {\theta_{\rm point}}
\newcommand {\thscat}  {\theta_{\rm scat}}
\newcommand {\etwelve} {E_{12\gradi}}
\newcommand {\ethirty} {E_{30\gradi}}
\newcommand {\eiso}[1] {E^{\, \triangleleft 30\gradi}_{#1}}
\newcommand {\phimiss} {{\phi_{\vec{p}_{\rm miss}}}}
\newcommand {\ewedge}  {E(\phi_{\vec{p}_{\rm miss}}\pm 15\gradi)}
%\newcommand {\ewedge}  {{E_{\rm w}}}
\newcommand {\evis}    {E_{\rm vis}}
\newcommand {\etot}    {E_{\rm vis}}
\newcommand {\emis}    {E_{\rm miss}}
\newcommand {\mvis}    {M_{\rm vis}}
\newcommand {\mtot}    {M_{\rm vis}}
\newcommand {\mmis}    {M_{\rm miss}}
\newcommand {\mhad}    {M^{\rm ex \, \ell_1}_{\rm vis}}
\newcommand {\mhadtwo} {M^{\rm ex \, \ell_1\ell_2}_{\rm vis}}
\newcommand {\ehad}    {E^{\rm NH}_{\rm vis}}
\newcommand {\epho}    {E^{\gamma}_{\rm vis}}
\newcommand {\echa}    {E^{\rm ch}_{\rm vis}}
\newcommand {\nch}     {{N_{\rm ch}}}
\newcommand {\elept}   {E_{\rm lept}}
\newcommand {\elepone} {E_{\ell _1}}
\newcommand {\eleptwo} {E_{\ell _2}}
\newcommand {\pvis}    {{\vec{p}_{\rm vis}}}
\newcommand {\pmis}    {{\vec{p}_{\rm miss}}}
\newcommand {\thmiss}  {{\theta_{\pmis}}}
\newcommand {\pt}      {{p_{\rm t}}}
\newcommand {\ptch}    {{p_{\rm t}^{\rm ch}}}
\newcommand {\pch}    {{p^{\rm ch}}}
\newcommand {\pz}      {{p_z}}
\newcommand {\ptnoNH}  {{p_{\rm t}^{\rm ex \, NH}}}
\newcommand {\puds}    {{P_{\rm uds}}}
%
\newcommand {\pmiss}   {{P\!\!\!\,\!/ }}
\newcommand {\emiss}   {{E\!\!\!\,\!/ }}
%
%
% no more of Christian's random capitalization!
% more of mine
\newcommand{\brchal}{\cal{B}($\PCha \rightarrow \ell\nu\PChi\ $)}
\newcommand{\M}{M_{2}}
\newcommand{\Mp}{M_{2}}
\newcommand{\sigbg}{\sigma_{\mathrm{bg}}}
\newcommand{\ww}   {\mathrm {WW}}
\newcommand{\zz}   {\mathrm Z\gamma^{*}}
\newcommand{\ewnu} {\mathrm{eW}\nu}
\newcommand{\eez}  {\mathrm {eeZ}}
\newcommand{\gagall}{{\gamma\gamma\rightarrow \ell\ell }}
\newcommand{\Pstaup}{{\widetilde{\tau}_{1}}}
\newcommand{\Pstaul}{{\widetilde{\tau}_{L}}}
\newcommand{\Pstaur}{{\widetilde{\tau}_{R}}}
\newcommand{\mzero}{m_{0}}
\newcommand{\msnu}{M_{\tilde{\nu}}}
\newcommand{\mcha}{M_{\chi^{\pm}}}
\newcommand{\mchi}{M_{\chi}}
\newcommand{\mstau}{M_{{\widetilde{\tau}_{1}}}}
\newcommand{\atau}{A_{\tau}}
\newcommand{\chsnu}{\PCha \rightarrow \ell \tilde{\nu}}
\newcommand{\chstau}{\PCha \rightarrow \tilde{\tau}_{1}\nu}
\newcommand{\chlep}{\PCha \rightarrow \ell\nu\chi}
\newcommand{\Tcsq}{\mathrm{TeV}/c^2}
% new for thesis
\newcommand{\nobs}{N_{\mathrm{obs}}}
\newcommand{\nlim}{N_{\mathrm{lim}}}
\newcommand{\Brl}{\cal{B}_{\ell}}
\newcommand{\leff} {\mathcal{L}_{\mathrm{eff}}}
\newcommand{\dedx}{{\mathrm{d}}E/{\mathrm{d}}x}
\newcommand{\chtau}{\PCha \rightarrow \tau\nu\chi}
\newcommand{\ssqtw}{\sin^{2}\theta_{\mathrm W}}
%\newcommand{\PSql}{\tilde{\mathrm q}_L}
%\newcommand{\PSqr}{\tilde{\mathrm q}_R}
%\newcommand{\PSq1}{\tilde{\mathrm q}_1}
%\newcommand{\PSq2}{\tilde{\mathrm q}_2}
%\newcommand{\ww}{{\mathrm WW}}
%\newcommand{\zz}{{\mathrm Z\gamma^{*}}}
%\newcommand{\eez}{{\mathrm eeZ}}
\newcommand{\nnz}{{\mathrm \nu\bar{\nu}Z}}
% added by bill
\def \ggll    {\gamma\gamma \rightarrow \ell^{+}{\ell}^{-}}
\def \tautau  {\mathrm \tau^{+}\tau^{-}}
\def \ffg  {f\bar{f}(\gamma)}
\def \lll   {\ell^{+}{\ell}^{-}}
\def \ww   {\mathrm WW}
\def \zz   {\mathrm Z\gamma^{*}}
\def \znn  {\mathrm Z\nu\nu}
\def \zee  {\mathrm Zee}
\def \rts  {\sqrt{s}}
\def \mstop {m_{\tilde{\mathrm{t}}}}
\def \msnu  {m_{\tilde{\nu}}}
\def \elow   {E_{12^{\circ}}}
\def \gev    { \, \mathrm{GeV}/\it{c}^{\mathrm{2}}}
\def \gvm    { \, \mathrm{GeV}/\it{c}}
\def \mx     {M_{\mathrm{eff}}} 
\newcommand{\neutr}{\chi}
%end fabio



%dalla mia pretesi

%\def \X             {\mathrm X} 
%\def \V             {\mathrm V} 
\def \Zcc           {\Z \to \charm \bar{\charm} }
\def \Zbb           {\Z \to \b \bar{\b} }
\def \decDS         {\D^{*+} \to \D^0 \pi^+}
\def \decsDS        {\D^{*+} \to \D^0 \pi^+_s}
\def \deckp         {\D^{0} \to \K^- \pi^+}
\def \deckppp       {\D^{0} \to \K^- \pi^+ \pi^+ \pi^-}
\def \deckpp        {\D^{0} \to \K^- \pi^+ \pi^0}
\def \deckpS        {\D^{0} \to \K^- \pi^+ (\pi^0)}
\def \decskp        {\D^{*+} \to \pi^{+}_{s} \K^- \pi^+}
\def \decskppp      {\D^{*+} \to \pi^{+}_{s} \K^- \pi^+ \pi^+ \pi^-}
\def \decskpp       {\D^{*+} \to \pi^{+}_{s} \K^- \pi^+ \pi^0}
\def \decskpS       {\D^{*+} \to \pi^{+}_{s} \K^- \pi^+ (\pi^0)}
\def \epsc          {\varepsilon_{\charm}}
\def \epsb          {\varepsilon_{\b}}
\def \pctod         {P_{\charm \to \D^*}}
\def \pbtod         {P_{\b \to \D^*}}
%\def \R             {{\mathrm R}}
\def \Gbb           {\Gamma_{\b\bar{\b}}}
\def \Gcc           {\Gamma_{\charm\bar{\charm}}}
\def \Gh            {\Gamma_{\mathrm h}}
%
% End of my macros
%


\begin{document}

%%%%%%%%%%%%% ptdr definitions %%%%%%%%%%%%%%%%%%%%%
%\input{ptdr-definitions}
%%%%%%%%%%%%%%%  Title page %%%%%%%%%%%%%%%%%%%%%%%%
% [Not required for PAS notes -- derived from directory name.] Please replace 2006/000 with your note number in the following line:
%\cmsNoteHeader{XXX-10-000}

\begin{titlepage}
   \analysisnote{2010/XXX}
   \date{9 February 2010}

  \title{Tracker Material Budget Estimation with Photon Conversions in Minimum Bias Events}

  \begin{Authlist}
    G.~Cerati
        \Instfoot{infn}{INFN Milano-Bicocca, CERN}
    G.~Sguazzoni
        \Instfoot{cern}{INFN Firenze}
  \end{Authlist}


  \begin{abstract}
A first attempt to quantitatively estimate the tracker material budget
based on the counting experiment of photon conversions is presented. The number
of photon conversions counted in a volume can be translated into a
material budget estimate for that volume once the shape and position
of the volume, the impinging photon flux and the conversion
reconstruction efficiencies are correctly taken into consideration. The
detailed accounting of all this issues is given and some preliminary
results based on $900\GeV$ late 2009 data are shown.
  \end{abstract}

%\note{Preliminary version}
\end{titlepage}

\setcounter{page}{2}%JPP


%%%%%%%%%%%%%%%%%%%%%%%%%%%%%%%%  Begin text %%%%%%%%%%%%%%%%%%%%%%%%%%%%%

\section{Introduction}
\label{introduction}

The Compact Muon Solenoid, CMS, is one of the two general-purpose
experiments installed at the Large Hadron Collider (LHC) at
CERN~\cite{cms}. The core of the CMS detector is
the superconducting solenoid, $6\m$ in diameter and $13\m$ long, that
produces a magnetic field of $3.8\T$. The solenoid contains, from
outside to inside, the calorimeter system and the silicon 
tracking system for the reconstruction of charged particles
trajectories.

CMS uses a right-handed coordinate system, with the origin at the
nominal interaction point, the $x$-axis pointing to the centre of the
LHC, the $y$-axis pointing up (perpendicular to the LHC plane), and
the $z$-axis along the anticlockwise-beam direction. The polar angle,
$\theta$, is measured from the positive $z$-axis and the azimuthal
angle, $\phi$, is measured in the $x$-$y$ plane. Pseudo-rapidity is
$\eta = -\log \tan \theta/2$.

The silicon tracking system, shown in figure~\ref{fig:tracker}, is composed of a Pixel
Silicon detector with three barrel layers at radii between $4.4\cm$
and $10.2\cm$ and two endcap disks at each end. Pixel sensors feature single pixel size
of $100\times150\um^2$ for a total of 66M channels.  
\begin{figure}[t]
%\begin{center}
%\includegraphics*[width=0.44\textwidth]{fig/sketch.pdf}
%\hskip 2mm
\includegraphics*[width=0.65\textwidth]{figs/layout_rz.pdf}\hspace{0.02\textwidth}%
\begin{minipage}[b]{0.33\textwidth}\caption{\label{fig:tracker}A simplified
    sketch of a quadrant of the $Rz$ section of the CMS Tracker (bold
    lines represent double sided module assemblies).}
\end{minipage}
%\end{center}
\end{figure}
The Silicon Strip Tracker
covers the radial range between $20\cm$ and $110\cm$ around the LHC
interaction point. The barrel region ($| z |  < 110\cm$) is split into
a Tracker Inner Barrel (TIB), made of four detector layers, and a
Tracker Outer Barrel (TOB), made of six detector layers. The TIB is
complemented by three Tracker Inner Disks 
per side (TID). The forward and backward
regions ($120\cm < |z| < 280\cm$) are covered by nine Tracker
End-Cap (TEC) disks per side
thus extending the overall acceptance to cover the region
$|\eta|<2.5$. In some of the layers and in the innermost 
rings, special double-sided modules
%(figure~\ref{fig:sst}, right panel)
are able to provide accurate three-dimensional position measurement of the charged
particle hits.  The Silicon Strip Tracker is the world's largest silicon
strip detector with a volume of approximately $23\m^3$, instrumented by about 15,000 modules
with different strip pitches ranging from 80 to $180\um$, for a total
of 198$\m^2$ of Silicon active area and about 9.6 million channels with full optical analog readout~\cite{cms}\cite{TkTDR}\cite{TkTDRadd}.
%The granularity was chosen to balance the need
%for a low occupancy, which is estimated to be a few percent at the
%largest expected LHC luminosity, and the requirement of minimising the
%power density and the amount of 
%material.

%The tracking detector features a 
%transverse momentum resolution of about 1-2\% for muons of
%$P_T\sim100\GeV$, an impact parameter resolution of about $10-20\um$
%for tracks with $P_T\sim10-20\GeV$, a reconstruction efficiency of
%tracks within jets of about 0.85-0.90 with a few percent fake rate.

%The operation of the CMS Pixel detector and of the CMS Silicon Tracker
%is described elsewhere~\cite{pxlop}~\cite{tkop}.

The CMS track reconstruction~\cite{trackreco} starts with the appropriate grouping of the hits
in the innermost layers to build up {\em seeds}. The seed is an initial track
estimate and consists of a triplet or a pair of hits, sufficient for a
basic prediction of the trajectory parameters if the primary vertex is
 also used. Starting from a given seed, pattern recognition using a Kalman Filter
is performed to build inside-out trajectories. Then each identified
track undergoes a procedure to reject possible outlier hits and is
refitted, also using a Kalman Filter. Finally, a quality selection is
performed. Reconstruction efficiency relies on several iterations (steps) of the
tracking procedure; every step, except the first, works on the
not-yet-associated hits surviving the previous step. Each step is
optimized with respect to the seeding topology and to the final
quality cuts. This recursive procedure is referred to as {\em
  iterative tracking}.

The algorithms that build up the CMS event reconstruction software, CMSSW~\cite{cmssw},
output physics objects (e.g., tracks, electrons, jets, ...) from the
raw data recorded by the detector. Ideally, all events collected by CMS are
reconstructed by the CMS prompt reconstruction system quasi real-time, soon after being
collected. The prompt reconstruction stream is of utmost
importance for a discovery experiment and prompt reconstruction data
samples have been the base of most of the CMS physics results so
far. Clearly, they are also crucial for fast and accurate feedback on detector
conditions.

The complexity and the granularity of the tracker system in
connection with the large LHC instantaneous luminosity, resulting in a
large number of interactions (pile-up, PU) per bunch crossing, make the
track reconstruction largely dominating the entire reconstruction
chain memory-wise and in terms of CPU time.

%In addition, a subset of the event reconstruction
%algorithms make up part of the high-level trigger system (HLT), which
%reduces the data recording rate from approximately 100 kHz of the Level-1
%trigger accepts down to a few hundred events/second of data written to
%tape rate. 

The LHC instantaneous luminosity is steadily growing since the
beginning of the run in 2010. In figure~\ref{fig:LHCinstLumi2011} and
figure~\ref{fig:LHCinstLumi2012} the maximum LHC instantaneous
luminosity per day delivered to CMS is shown as a function of time for 2011 and 
2012, respectively. During 2012 the LHC instantaneous luminosity is expected to
reach a value of about $7\cdot10^{33}\cm^{-2}\sec^{-1}$ that corresponds to a number
of primary vertices per bunch crossing of about 25.
\begin{figure}[t]
%\begin{center}
\includegraphics*[width=0.55\textwidth]{figs/peak_lumi_per_day_pp_2011.png}\hspace{0.0002\textwidth}%
\begin{minipage}[b]{0.33\textwidth}\caption{\label{fig:LHCinstLumi2011}Maximum
    instantaneous luminosity per day delivered to CMS for pp running at $7\TeV$ centre-of-mass energy in 2011.} 
\end{minipage}
%\vskip -5mm
%\end{center}
\end{figure}
\begin{figure}[b]
%\begin{center}
\includegraphics*[width=0.55\textwidth]{figs/peak_lumi_per_day_pp_2012.png}\hspace{0.0002\textwidth}%
\begin{minipage}[b]{0.33\textwidth}\caption{\label{fig:LHCinstLumi2012}Maximum
    instantaneous luminosity per day delivered to CMS for 
  pp running at $8\TeV$ centre-of-mass energy in 2012.} 
%\vskip -5mm
%\end{center}
\end{minipage}
\end{figure}

The other worsening factor that was not foreseen at the design level
of LHC and consequently not taken into account in writing first
versions of reconstruction software, is the bunch crossing
frequency. LHC was supposed to run with 25ns between subsequent bunch
crossings. But as a consequence of the LHC operating conditions and
beam optics, it is preferable to run with 50ns between bunch
crossings. This allows for fatter bunches in the machine and,
eventually, for a larger overall delivered luminosity. On the other
hand, this results into a larger pile-up per bunch crossing and
considerably increases the computing time per event that, being due to
combinatorics, scales faster than with purely linear dependence.
During 2011 it was soon clear that the current version of
reconstruction software was not performing well enough to withstand prompt
reconstruction for the last part of 2011 and 2012. 

This paper describes the actions that have been put in place to
improve the tracking reconstruction software, the major responsible of
the CPU load at the reconstruction level, in order to be compliant
with the expected luminosity in 2011 and 2012. This improvement
campaign took place in two phases: the first phase took place during
2011 and was completed in September 2011 (so-called {\em Fall 2011
  campaign}), the second phase started at the beginning of 2012 and
was completed in March 2012 ({\em Spring 2012 campaign}).

\section{Analysed Samples}
\label{sample}

Results shown in this note have been obtained with the release {\tt 3\_3\_6\_patch3} of {\tt CMSSW}.

Data sample corresponds to the dataset\\
{\tt /MinimumBias/BeamCommissioning09-BSCNOBEAMHALO-Dec19thSkim\_336p3\_v1/RAW-RECO}.\\
The list of analyzed runs and lumisections is taken from the table recommended by the Tracker 
DPG\footnote{https://twiki.cern.ch/twiki/bin/viewauth/CMSTKPOGCollisions900GeVDec\#Run\_and\_LumiSection\_selections},
i.e. all runs at $900\GeV$ with detector in stable conditions and magnet fully operational. 
Runs at $\sqrt{s}=2.36\TeV$ are not included.

As far as Monte Carlo simulation is concerned, a sample of $\sim11$M events of minimum bias at $900\GeV$ is used; this sample features a displaced beam spot reproducing the actual beam spot position as observed in collision data. In particular the dataset is:\\
{\tt /MinBias/Summer09-STARTUP3X\_V8K\_900GeV-v1/GEN-SIM-RECO}.

Events are selected with the following requirements:
\begin{itemize}
\item Trigger: bits 0 and 40, plus a veto of beam-halo triggers (bits 36-39); in MC events only bit 40 is required;
\item beam scraping events are rejected requiring a fraction of tracks with \emph{highPurity} quality label greater than 20\%;
\item a {\em good} reconstructed primary vertex is required; a reconstructed primary vertex is defined as good if it originates at least four tracks and if its position lies within $15\cm$ in the longitudinal 
direction and within $2\cm$ in the transverse plane.
\end{itemize}

The results shown in this note are normalized to the number of events surviving the above described selection; in particular 257,526 for data and 6,142,518 for MC, thus the weight applied to MC entries is $4.2\cdot 10^{-2}$. 


\section{Data and MC Performance}
\label{dataVsMc}
Given the limited number of recorded events passing the trigger and
vertex requirements, the collected statistics allow us to investigate
only the pixel barrel vlolume, that is the tracker region presently
giving the best performance in terms of conversion reconstruction.

A sample of conversions in this region is selected by applying the
following cuts designed to guarantee a good balance between purity and
efficiency:
\begin{itemize}
\item tracks with at least 4 hits;
\item track $d_0\cdot q > 0.1\cm$;
\item track pair opening angle on X-Y plane $\Delta\phi<0.2$;
\item reconstructed conversion vertex radius $r<15\cm$;
\item reconstructed conversion vertex longitudinal position $|z|< 26\cm$;
\item reconstructed transverse momentum of the converted photon $p_T<5\GeV$.
\end{itemize}

Expected performance in terms of efficiency and purity have been
estimated on the MC sample.
Efficiency is defined as the number of associated conversions
$N^{\rm assoc}_{\gamma}$ divided by the total number of simulated
conversions $N^{\rm sim}_{\gamma}$, while purity is defined as
$N^{\rm assoc}_{\gamma}$ divided by the number of reconstructed
conversions $N^{\rm reco}_{\gamma}$.
A reconstructed conversion is considered associated if simulated and reconstructed vertex positions match within a $10\times10\times10\cm^3$ 
box\footnote{A more proper association criteria, based on the hit
  association of the electron tracks, cannot be applied on the used MC
  sample. However, it was proven that the hit and vertex position
  association methods provide compatible results}.
Figure~\ref{efficpurity} shows that performance degrade for radius greater than $5\cm$, i.e. outside the first pixel barrel layer.
Very likely the main reason for this effect is the current tracking configuration, allowing for the reconstruction of very low $p_T$ tracks from
pixel hit-triplets only. A different tune of $p_T$ thresholds in iterative tracking steps could improve performance at larger radii.
%efficiency
%purity
\begin{figure}[!hbtp]
\subfigure[]{
\centering
\label{efficpurity_vs_r}
\includegraphics[width=.45\textwidth]{efficpurity_vs_r.pdf}}
\subfigure[]{
\centering
\label{efficpurity_vs_pt}
\includegraphics[width=.45\textwidth]{efficpurity_vs_pt.pdf}}
\caption{Efficiency (blue) and purity (red) vs $r$\subref{efficpurity_vs_r} and $p_T$\subref{efficpurity_vs_pt}.}
\label{efficpurity}
\end{figure}

In order to check that the conversion algorithm is providing
reasonable results with collision data, the basic conversion
distributions are compared to what is expected form MC. Converted
photon $p_T$, $\eta$ and $r$ are reported in
Figures~\ref{fig:eta}-\ref{fig:r}. Shapes show a nice agreement
between data and MC, while the number of comversions (normalized to
the total number of events) is $\sim10\%$ higher in data with respect
to MC. Such discrepancy could be due to several effects, like a higher
number of fakes in data or differences in the MC photon flux
and it is acceptable for the sake of the present preliminary study.
\begin{figure}[!hbtp]
\centering
\subfigure[]{
\label{subfig:eta_lin}
\includegraphics[width=.45\textwidth]{eta_lin.png}}
\subfigure[]{
\label{subfig:eta_log}
\includegraphics[width=.45\textwidth]{eta_log.png}}
\caption{Eta distribution of reconstructed conversions in linear\subref{subfig:eta_lin} and logaritmic\subref{subfig:eta_log} scale.
Data is shown in black dots and MC in blue boxes. Red boxes represent the expected fake contribution.}
\label{fig:eta}
\end{figure}

\begin{figure}[!hbtp]
\centering
\subfigure[]{
\label{subfig:pt_lin}
\includegraphics[width=.45\textwidth]{pt_lin.png}}
\subfigure[]{
\label{subfig:pt_log}
\includegraphics[width=.45\textwidth]{pt_log.png}}
\caption{$p_T$ distribution of reconstructed conversions in linear\subref{subfig:pt_lin} and logaritmic\subref{subfig:pt_log} scale.
Data is shown in black dots and MC in blue boxes. Red boxes represent the expected fake contribution.}
\label{fig:pt}
\end{figure}

\begin{figure}[!hbtp]
\centering
\subfigure[]{
\label{subfig:r_lin}
\includegraphics[width=.45\textwidth]{r_lin.png}}
\subfigure[]{
\label{subfig:r_log}
\includegraphics[width=.45\textwidth]{r_log.png}}
\caption{Radius distribution of reconstructed conversions in linear\subref{subfig:r_lin} and logaritmic\subref{subfig:r_log} scale.
Data is shown in black dots and MC in blue boxes. Red boxes represent the expected fake contribution.}
\label{fig:r}
\end{figure}


\section{Material Budget Estimation}
\label{correctionFactors}

The following approach to the material budget estimation is along the lines of ref.~\cite{steve}. 

The number of photon conversions $dN_{\rm conv}$ in a
small volume filled with a homogenous material is:
\begin{equation}
dN_{\rm conv} = dN_{\gamma} \frac{P}{X_0} dt .
\label{eq:1}
\end{equation}
where: $dN_\gamma$ is the number of impinging photons; $P$ is the energy dependent conversion probability per unit
radiation length ($P\sim 7/9$); $dt$ is the volume effective thickness (i.e. with respect to the photon arrival direction); $X_0$ is the radiation length (in $\cm$).

Given a portion $R^2 \sin \theta\, d\theta\, d\phi$ centered in $(R,\theta,\phi)$ of a homogenous spherical skin of 
thickness $dR$ in a spherical reference system $(R, \theta, \phi)$,
relation~(\ref{eq:1}) is:
\begin{equation}
dN_{\rm conv} = N_{\gamma}(R, \theta, \phi) \cdot R^2 \sin \theta \, d\theta\, d\phi \cdot \frac{P}{X_0} dR \;,
\label{eq:2}
\end{equation}
where $N_\gamma(R,\theta,\phi)$ is the photon flux impinging the surface element.

However in a particle physics experiment the most common geometrical
structure is a cylindrical skin for which cylindrical reference system
is more appropriate and in some case also cartesian reference system
is convenient.


% \noindent{\bf Pseudo-cylindrical reference system.} 
% In a pseudo-cylindrical
% reference system $(r,\theta,\phi)$ ($r=R\sin\theta$), the relation~(\ref{eq:2}) can be
% conveniently rewritten having care of using the appropriate Jacobian
% factor, i.e.
% \begin{equation}
%  dr\, d\theta\, d\phi = \left| \frac{dr\,d\theta\,d\phi}{dR\,d\theta\,d\phi}
%    \right| \cdot dR\,d\theta\,d\phi
% \label{eq:3}
% \end{equation}
% where
% \begin{equation}
% \begin{split}
% \frac{dr\,d\theta,d\phi}{dR\,d\theta\,d\phi}
% & = \det \left( \begin{array}{ccc}
% \nicefrac{dr}{dR}      & \nicefrac{dr}{d\theta}      & \nicefrac{dr}{d\phi} \\
% \nicefrac{d\theta}{dR} & \nicefrac{d\theta}{d\theta} & \nicefrac{d\theta}{d\phi} \\
% \nicefrac{d\phi}{dR}   & \nicefrac{d\phi}{d\theta}   & \nicefrac{d\phi}{d\phi}
% \end{array} \right) =\\
% & = \det \left( \begin{array}{ccc}
% \sin \theta & R\cos \theta & 0\\
%  0 & 1 & 0 \\
%  0 & 0 & 1 
% \end{array} \right)
% = \sin \theta .
% \end{split}
% \label{eq:4}
% \end{equation}
% Using~(\ref{eq:4}) in~(\ref{eq:2}) we get:
% \begin{equation}
% dN_{\rm conv} = N_{\gamma}(r, \theta, \phi) 
% \frac{P}{X_0} \frac{r^2}{\sin^2\theta} \,dr\,d\theta\,d\phi .
% \label{eq:5}
% \end{equation}

\noindent{\bf Cylindrical reference system.} 
In a cylindrical reference system $(r,z,\phi)$ ($r=R\sin\theta$, $z=R\cos\theta$) the relation~(\ref{eq:2}) can be
conveniently rewritten having care of using the appropriate Jacobian
factor, i.e.
\begin{equation}
 dr\,dz\,d\phi = \left| \frac{dr\, dz\, d\phi}{dR\,d\theta\,d\phi}
   \right| \cdot dR\,d\theta\,d\phi
\label{eq:3bis}
\end{equation}
where
\begin{equation}
\begin{split}
\frac{dr\,dz\,d\phi}{dR\,d\theta\, d\phi}
& = \det \left( \begin{array}{ccc}
\nicefrac{dr}{dR}      & \nicefrac{dr}{d\theta}      & \nicefrac{dr}{d\phi} \\
\nicefrac{dz}{dR}      & \nicefrac{dz}{d\theta}      & \nicefrac{dz}{d\phi} \\
\nicefrac{d\phi}{dR}   & \nicefrac{d\phi}{d\theta}   & \nicefrac{d\phi}{d\phi}
\end{array} \right) = \\
& = \det \left( \begin{array}{ccc}
\sin \theta & R\cos\theta  & 0\\
\cos\theta  & -R\sin\theta & 0 \\
 0 & 0 & 1 
\end{array} \right)
= R .
\end{split}
\label{eq:4bis}
\end{equation}
Using~(\ref{eq:4bis}) in~(\ref{eq:2}) we get:
\begin{equation}
dN_{\rm conv} = N_{\gamma}(r, z, \phi) 
\frac{P}{X_0} r\, \,dr\,dz\,d\phi .
\label{eq:5bis}
\end{equation}

\noindent{\bf Cartesian reference system.} 
Similarly, in a cartesian reference system $(x, y, z)$ ($x=R\sin\theta\cos\phi$, $y=R\sin\theta\sin\phi$, $z=R\cos\theta$)
\begin{equation}
 dx\,dy\,dz  = \left| \frac{dx\, dy\, dz}{dR\,d\theta\,d\phi}
   \right| \cdot dR\,d\theta\,d\phi
\label{eq:3tris}
\end{equation}
where
\begin{equation}
\begin{split}
\frac{dx\,dy\,dz}{dR\,d\theta\,d\phi}
& = \det \left( \begin{array}{ccc}
\nicefrac{dx}{dR}      & \nicefrac{dx}{d\theta}      & \nicefrac{dx}{d\phi} \\
\nicefrac{dy}{dR}      & \nicefrac{dy}{d\theta}      & \nicefrac{dy}{d\phi} \\
\nicefrac{dz}{dR}      & \nicefrac{dz}{d\theta}   & \nicefrac{dz}{d\phi}
\end{array} \right)
= \\
& = \det \left( \begin{array}{ccc}
\sin\theta\cos\phi & R\cos\theta\cos\phi & -R\sin\theta\sin\phi\\
\sin\theta\sin\phi & R\cos\theta\sin\phi &  R\sin\theta\cos\phi\\
\cos\theta & -R\sin\theta & 0 
\end{array} \right)
= \\
& = R^2 \sin \theta = \sqrt{x^2+y^2}\sqrt{x^2+y^2+z^2}.
\end{split}
\label{eq:4tris}
\end{equation}
Using~(\ref{eq:4bis}) in~(\ref{eq:2}) we get:
\begin{equation}
dN_{\rm conv} = N_{\gamma}(x, y, z) 
\frac{P}{X_0}
%\frac{\sqrt{x^2+y^2+z^2}}{\sqrt{x^2+y^2}}
 \,dx\,dy\,dz .
\label{eq:5tris}
\end{equation}

\noindent{\bf Photon flux.} 
A reasonable but approximate guess of the form of
$N_{\gamma}(R, \theta, \phi)$ in the pp collisions at LHC can be
inferred assuming the following:
\begin{itemize}
\item[a)] all photons are originating at the interaction point $(0, 0, 0)$;
\item[b)] all photons come from QCD events ($\pi_0$ decays);
\item[c)] the number of photons interacting with the material is negligible. 
\end{itemize}
All these three assumption are to some extent not true, c) especially,
but let's give them for granted for the present preliminary study.

%\clearpage

The dependence on the distance from the interaction point is easily inferred observing that, givem the above assumptions, 
the flux is the same in the same portion of solid angle $\delta \Omega=\sin \theta\delta\theta d\phi$:%$\d \Omega=\sin \theta\d\theta d\phi$: 
\begin{equation}
N_{\gamma}(R', \theta, \phi) R'^2 d\Omega = N_{\gamma}(R, \theta, \phi) R^2 d\Omega,
\label{eq:6pre}
\end{equation}
from which immediately follows that
\begin{equation}
N_{\gamma}(R) \propto \frac{1}{R^2} = \frac{\sin^2 \theta}{r^2} = \frac{1}{x^2+y^2+z^2} .
\label{eq:6}
\end{equation}

As a consequence of the cylindrical symmetry of pp interactions
$N_\gamma$ does not depend on $\phi$.

As far as the $\theta$ dependance is concerned, given that 
\begin{equation}
\eta = -\ln \left( \tan \nicefrac{\theta}{2} \right)
\label{eq:7}
\end{equation}
it follows that
\begin{equation}
\begin{split}
dN_\gamma & = N_\gamma(\eta) d\eta  =
N_\gamma(\eta)\frac{d\eta}{d\cos\theta} d\cos\theta = N_\gamma(\eta)\frac{d\eta}{d\theta}\frac{d\theta}{d\cos\theta} d\cos\theta=\\ 
 & =N_\gamma(\eta)\frac{1+\tan^2\nicefrac[]{\theta}{2}
 }{2\tan\nicefrac{\theta}{2}} \frac{1}{\sin\theta} d\cos\theta =
 N_\gamma(\eta)\frac{1}{2\sin\nicefrac[]{\theta}{2}\cos\nicefrac[]{\theta}{2}} \frac{1}{\sin\theta}  d\cos\theta  
 = N_\gamma(\eta)\frac{1}{\sin^2\theta}  d\cos\theta .
\label{eq:8}
\end{split}
\end{equation}
Eq.~(\ref{eq:8}) allows to identify $N_\gamma(\theta)$ with $N_\gamma(\eta(\theta))/\sin^2{\theta}$.
%\begin{equation}
%\begin{split}
%dN_\gamma(\theta) & = \frac{dN_\gamma}{d\theta} d\theta  =
%\frac{dN_\gamma}{d\eta}\frac{d\eta}{d\theta}  d\theta  = %\frac{dN_\gamma}{d\eta}\frac{1+\tan^2\nicefrac[]{\theta}{2}
% }{2\tan\nicefrac{\theta}{2}}  d\theta =\\ 
% & =
% \frac{dN_\gamma}{d\eta}\frac{1}{2\sin\nicefrac[]{\theta}{2}\cos\nicefrac[]{\theta}{2}}  d\theta  
% = \frac{dN_\gamma}{d\eta}\frac{1}{\sin\theta}  d\theta .
%\label{eq:8}
%\end{split}
%\end{equation}
Since 'all' photons come from $\pi_0$'s that show an almost flat
distribution vs. $\eta$, also photon $\eta$-distribution is flat as
well and $N_\gamma(\eta)$ does not depend on $\eta$, thus:
\begin{equation}
N_\gamma(\theta) \propto \frac{1}{\sin^2 \theta} = \frac{x^2+y^2+z^2}{x^2+y^2}.
\label{eq:9}
\end{equation}
Putting~(\ref{eq:6}) and~(\ref{eq:9}) together we get 
\begin{equation}
N_{\gamma} (R, \theta, \phi) = k \frac{1}{R^2\sin^2 \theta} \, \, \, ,
\label{eq:10pre}
\end{equation}
% \begin{equation}
% N_{\gamma} (r, \theta, \phi) = k \frac{\sin \theta}{r^2} \,\,\, ,
% \label{eq:10}
% \end{equation}
\begin{equation}
N_{\gamma} (r, z, \phi) = k \frac{1}{r^2} \,\,\, ,
\label{eq:10bis}
\end{equation}
and
\begin{equation}
N_{\gamma} (x, y, z) = k \frac{1}{x^2+y^2} \,\,\, ,
\label{eq:10tris}
\end{equation}
for spherical,
%pseudo-cylindrical, 
cylindrical and cartesian coordinates
respectively. In all cases $k$ is an appropriate dimensional factor
that accounts for all necessary constants.

\noindent{\bf Geometrical dependence of photon conversions.} 
Equations
%~(\ref{eq:5}) with~(\ref{eq:10}),
~(\ref{eq:5bis})
with~(\ref{eq:10bis}), and~(\ref{eq:5tris}) with~(\ref{eq:10tris}),
respectively, allow for the following expressions for the number of photon
conversion to be written:
\begin{equation}
d N_{\rm conv} = k \frac{1}{r^2} 
\frac{P}{X_0} r dr\, dz\, d\phi = k
\frac{P}{X_0} \frac{1}{r}\, dr\, dz\, d\phi ,
\label{eq:11bis}
\end{equation}
\begin{equation}
\begin{split}
d N_{\rm conv} & = k \frac{P}{X_0} \frac{1}{x^2+y^2} \,dx\,dy\,dz .
\end{split}
\label{eq:11tris}
\end{equation}
After the appropriate integration,
%Eq.~(\ref{eq:11}),
Eq.~(\ref{eq:11bis}) and~(\ref{eq:11tris}) can be used to extract the
geometrical factor to translate the observed number of conversion in a given 
volume into an estimate of $P/X_0$ (for the moment let's assume that
the conversion reconstruction efficiency is 1 with no background).

Few relevant examples follow.

\begin{description}
% \item[$r_1<r<r_2$,~$\, \theta_1<\theta<\theta_2$,~$\, 0<\phi<2\pi$;]
%   from~Eq.~(\ref{eq:11}): 
% \begin{equation}
% \begin{split}
% N_{\rm conv} & = k \frac{P}{X_0} \int_{\theta_1}^{\theta_2}
% \frac{d\theta}{\sin\theta} \int_{r_1}^{r_2}dr \int_0^{2\pi}d\phi = \\
% & = 2\pi k \frac{P}{X_0} (r_2-r_1) \cdot \left. \ln \tan
%     \frac{\theta}{2} \right|_{\theta_1}^{\theta_2} =\\
% & = 2\pi k \frac{P}{X_0} (r_2-r_1) (\eta_1 - \eta_2).
% \end{split}
% \label{eq:12}
% \end{equation}
%
%%%%% Versione precendete (ma da conservare perche` forse e` il conteggio dei fotoni)
%
\item[$r_1<r<r_2$,~$\, z_1<z<z_2$,~$\, 0<\phi<2\pi$;]
  from~Eq.~(\ref{eq:11bis}): 
\begin{equation}
\begin{split}
N_{\rm conv} & = k
\frac{P}{X_0} \int_{r_1}^{r_2} \frac{dr}{r} \int_{z_1}^{z_2} dz
\int_0^{2\pi} d\phi = \\
& = 2\pi k \frac{P}{X_0}  (z_2 - z_1) \ln \frac{r_2}{r_1} \equiv k \frac{P}{X_0} f^{r, z}_{\rm geom}.\\
\end{split}
\label{eq:13}
\end{equation}
%
%%%%%%%%%%%%%%%%%%%%%%%%%%%%%%%%%%%%%%%%%%%%%%%%%%%%
%%%%% Versione precendete (ma da conservare perche` forse e` il conteggio dei fotoni)
%
% \item[$r_1<r<r_2$,~$\, z_1<z<z_2$,~$\, 0<\phi<2\pi$;]
%   from~Eq.~(\ref{eq:11bis}): 
% \begin{equation}
% \begin{split}
% N_{\rm conv} & = k
% \frac{P}{X_0} \int_{r_1}^{r_2} \int_{z_1}^{z_2} \frac{dr \, dz}{\sqrt{r^2+z^2}}
% \int_0^{2\pi} d\phi = \\
% & = 2\pi k \frac{P}{X_0} \int_{r_1}^{r_2} dr \left. \ln \left( 2\sqrt{r^2+z^2}+2z \right) \right|_{z_1}^{z_2} = \\
% & = 2\pi k \frac{P}{X_0} \int_{r_1}^{r_2} dr \ln \frac{\sqrt{r^2+{z_2}^2}+z_2}{\sqrt{r^2+{z_1}^2}+{z_1}} = \\ 
% & = 2\pi k \frac{P}{X_0} \left[ r \ln \frac{\sqrt{r^2+{z_2}^2}+z_2}{\sqrt{r^2+{z_1}^2}+{z_1}} + \right. \\
% & - z_1 \ln \left( 2\sqrt{r^2+{z_1}^2}+r \right)  + \Biggl. \Biggl. z_2 \ln \left( 2\sqrt{r^2+{z_2}^2}+r \right) \Biggr] \Biggr|_{r_1}^{r_2} = \\
% & = 2\pi k \frac{P}{X_0} \left[ r_2 \ln \frac{\sqrt{{r_2}^2+{z_2}^2}+z_2}{\sqrt{{r_2}^2+{z_1}^2}+z_1} - r_1 \ln \frac{\sqrt{{r_1}^2+{z_2}^2}+z_2}{\sqrt{{r_1}^2+{z_1}^2}+z_1} + \right. \\
% & \left. + z_2 \ln \frac{\sqrt{{r_2}^2+{z_2}^2}+r_2}{\sqrt{{r_1}^2+{z_2}^2}+r_1} - z_1 \ln \frac{\sqrt{{r_2}^2+{z_1}^2}+r_2}{\sqrt{{r_1}^2+{z_1}^2}+r_1} \right] \\
% & \equiv k \frac{P}{X_0} f^{r, z}_{\rm geom}.\\
% \end{split}
% \label{eq:13}
% \end{equation}
%
%%%%%%%%%%%%%%%%%%%%%%%%%%%%%%%%%%%%%%%%%%%%%%%%%%%%
%
\item[$x_1<x<x_2$,~$\, y_1<y<y_2$,~$\, z_1<z<z_2$;]
  from~Eq.~(\ref{eq:11tris}): 
\begin{equation}
\begin{split}
N_{\rm conv} & = k
\frac{P}{X_0} \int_{x_1}^{x_2} \int_{y_1}^{y_2} \frac{dx\,
  dy }{x^2+y^2} \int_{z_1}^{z_2} dz  . 
\end{split}
\label{eq:14}
\end{equation}
The analytical solution~\cite{integrals} of the integral would require the computation of polylogarithmic functions. Within
our needs (i.e. for a scatter plot in the $xy$ plane), if $r\gg |x_2-x_1|$ and $r\gg |y_2
- y_1|$, a likely sufficient approximate
solution is
\begin{equation}
N_{\rm conv} = k
\frac{P}{X_0} \frac{\Delta x\, \Delta y\, \Delta z}{\overline{r}^2} \equiv k \frac{P}{X_0} f^{x, y}_{\rm geom}
\label{eq:15}
\end{equation}
where $\overline{r}$ is the ``average'' transverse radius, i.e. the one taken on
the middle of the ``box'':
\begin{equation}
 \overline{r}=\nicefrac{1}{2}\sqrt{(x_1+x_2)^2+(y_1+y_2)^2}.
\label{eq:16}
\end{equation}
%$\overline{r}=\sqrt{\nicefrac{(x_1+x_2)^2}{4}+\nicefrac{(y_1+y_2)^2}{4}}$.
% \item[$x_1<x<x_2$,~$\, y_1<y<y_2$,~$\, z_1<z<z_2$;]
%   from~Eq.~(\ref{eq:11tris}): 
% \begin{equation}
% \begin{split}
% N_{\rm conv} & = k
% \frac{P}{X_0} \int_{x_1}^{x_2} \int_{y_1}^{y_2} \int_{z_1}^{z_2}  \frac{dx\,
%   dy \, dz}{\sqrt{x^2+y^2}\sqrt{x^2+y^2+z^2}}. 
% \end{split}
% \label{eq:14}
% \end{equation}
% The analytical solution of the integral does not exist~\cite{integrals}. Within
% our needs (i.e. for a scatter plot in the $xy$ plane) we can integrate first in $dz$:
% \begin{equation}
% \begin{split}
% N_{\rm conv} & = k
% \frac{P}{X_0} \int_{x_1}^{x_2} \int_{y_1}^{y_2} dx \, dy \left. \frac{1}{r} \ln \left( 2\sqrt{r^2+z^2} + 2z \right) \right|_{z_1}^{z_2} \\
% & = k \frac{P}{X_0} \int_{x_1}^{x_2} \int_{y_1}^{y_2} dx \, dy  \frac{1}{r} \ln \frac{\sqrt{r^2+{z_2}^2} + z_2}{\sqrt{r^2+{z_1}^2} + z_1} ,
% \end{split}
% \label{eq:15pre}
% \end{equation}
% where $r=\sqrt{x^2+y^2}$. If $r\gg |x_2-x_1|$ and $r\gg |y_2
% - y_1|$, a likely sufficient approximate
% solution is
% \begin{equation}
% \begin{split}
% N_{\rm conv} & = k
% \frac{P}{X_0} \frac{\Delta x\, \Delta y}{\overline{r}} \,  \ln \frac{\sqrt{\overline{r}^2+{z_2}^2} + z_2}{\sqrt{\overline{r}^2+{z_1}^2} + z_1} \\
% & \equiv k \frac{P}{X_0} f^{x, y}_{\rm geom}
% \end{split}
% \label{eq:15}
% \end{equation}
% where $\overline{r}$ is the ``average'' radius, i.e. the one taken on
% the middle of the ``box'', $\overline{r}=\nicefrac{1}{2}\sqrt{(x_1+x_2)^2+(y_1+y_2)^2}$.
% %$\overline{r}=\sqrt{\nicefrac{(x_1+x_2)^2}{4}+\nicefrac{(y_1+y_2)^2}{4}}$.
\end{description}

%\end{document}










  

\section{Results}
\label{results}

The conversion counting in a given volume can be translated into a material budget estimate once the efficiency and the impinging 
photon flux are correctly taken into account. 
In particular, assuming a negligible background, the number of reconstructed photon conversion $N_{\rm reco}$ in a given volume bin is
\begin{equation}
N_{\rm reco} \propto \varepsilon \cdot \langle \frac{P}{X_0} \rangle \cdot f_{\rm geom}
\end{equation}
where $\varepsilon$ is the reconstruction efficiency and $\langle P/{X_0} \rangle$ is the average conversion probability ($P\sim 7/9$).

The present statistics allow only the pixel barrel to be studied. 
It is identified by choosing the following fiducial volume: $|z|<26~{\rm cm}$ and $r$ comprised between $0.8$ and $18\cm$.
Such region is divided for convenience into four parts corresponding to the beam pipe and the three pixel barrel layers, as detailed in Table~\ref{table:bins}.


\begin{table}[h]
\begin{center}
  \begin{tabular}{rccc}
    Label & $r_{\rm min}$ [$\cm$] & $r_{\rm max}$ [$\cm$] & Efficiency \\  
\hline
BP & 0.8 & 3.2 & 2.7\% \\
PXL1 & 3.2 & 6.0   & 3.5\% \\
PXL2 & 6.0 & 8.8 & 0.5\% \\
PXL3 & 8.8 & 18.0  & 0.1\% \\
  \end{tabular}
  \caption{Parameters and conversion reconstruction efficiency of the four regions of the pixel barrel volume corresponding to the beam pipe and the three pixel barrel layers, respectively.}
\label{table:bins}
\end{center}
\end{table}


%$1~{\rm cm}<r_{\rm BP}<3.5~{\rm cm}<r_{\rm PXL1}<6~{\rm cm}<r_{\rm PXL2}<8.4~{\rm cm}<r_{\rm PXL3}<15~{\rm cm}$. \\

The conversion finding efficiency $\varepsilon$, estimated from MC data for each sub-region, is also reported in Table~\ref{table:bins}.
%%2.8\%, 3.5\%, 0.5\% and 0.1\% respectively.

Figure~\ref{figMBvsr} represents an uncalibrated estimation, given in arbitrary units, 
comparing the material budget of the beam pipe and of the pixel barrel layers as a function of the radius $r$. The bin width is $0.2\cm$.
The correction factor $f^{r, z}_{\rm geom}$ is given by Eq.~\ref{eq:13}.
The green line shows the case of ideal resolution and efficiency, corrected for $f^{r, z}_{\rm geom}$.
Blue boxes represent MC pseudo data and black dots data; they are both corrected for $\varepsilon$ and $f^{r, z}_{\rm geom}$.
The conversion radius in data is computed with respect to the pixel center $(-0.1475\cm, -0.3782\cm, -0.4847\cm)$, 
as estimated from tracker alignment algorithms. 
The expected fake contribution, shown in red boxes, is not subtracted. 
Data show a good agreement with MC, both in shapes and in the overall number of entries.

\begin{figure}[!htbp]
 \begin{center}
   \includegraphics[width=\textwidth]{rPlot.png}
 \end{center}
 \caption{Uncalibrated material budget vs. radius; bin width equal to $0.2\cm$.}
\label{figMBvsr}
\end{figure}

Statistical fluctuations dominates the region of the third pixel layer in Fig.~\ref{figMBvsr}. The same plot done with a bin width of $0.4\cm$, shown in Fig.~\ref{figMBvsr_coarse}, allows a better reading of the third pixel layer region that is anyhow characterised by a large background contribution. 

\begin{figure}[!htbp]
 \begin{center}
   \includegraphics[width=\textwidth]{rPlot_coarse.png}
 \end{center}
 \caption{Uncalibrated material budget vs. radius; bin width equal to $0.4\cm$.}
\label{figMBvsr_coarse}
\end{figure}


Figure~\ref{xyPlot} and Figure~\ref{zrPlot} show the comparison in the $x$$-$$y$ and $z$$-$$r$ planes respectively, where the grey density 
gives an estimate of the amount of material in bins of size $0.25\cm\times0.25\cm$. 
The material in the $x$$-$$y$ plane is computed using the correction factor $f^{x, y}_{\rm geom}$ defined in Eq.~\ref{eq:15}.
The low statistics of reconstructed photon conversions does not allow a quantitative comparison of the results from data yet.

\begin{figure}[!hbtp]
\centering
\includegraphics[height=0.9\textheight]{xyPlotVert.png}
\caption{Material budget estimation $x$$-$$y$ map. Top: MC truth; Center: MC pseudo-data; Bottom: data.}
\label{xyPlot}
\end{figure}

\begin{figure}[!hbtp]
\centering
\includegraphics[width=\textwidth]{zrPlot.png}
\caption{Material budget estimation $z$$-$$r$ map. Top: MC truth; Center: MC pseudo-data; Bottom: data.}
\label{zrPlot}
\end{figure}


%\section{Unfolding}
\label{unfolding}

A more complete analysis of the material budget estimate needs to take into account the non ideal conversion vertex position resolution.
In fact, the radius of simulated conversion vertices in the pixel barrel region is well separated in the four peaks due to the beam pipe
and the the pixel layers (Fig.~\ref{Rmc}).


\begin{figure}[|htbp]
 \begin{center}
   \includegraphics[width=\textwidth]{Rmc.png}
 \end{center}
 \caption{.}
\label{Rmc}
\end{figure}

\begin{figure}[|htbp]
 \begin{center}
   \includegraphics[width=\textwidth]{Rreco.png}
 \end{center}
 \caption{.}
\label{Rreco}
\end{figure}


%\thispagestyle{empty}
\section{Conclusions}
\label{section_conclusions}

A first attempt of estimating the Tracker Material Budget using photon conversions with 2009 $\sqrt{s}=900\GeV$ CMS 
data has been completed.
Results are still qualitative since the statistics is too low. 
However, a good match with MC data is already visible in a region including the beam pipe and the two innermost pixel barrel layer.

The method here presented provides an uncalibrated measurement of the material, after correcting the number of conversions per unit volume 
for appropriate factors due to reconstruction efficiency, photon flux and geometrical effects. 

Current problems of this method are the low photon reconstruction efficiency at radii exceeding the first pixel barrel layer, the poor
conversion radius resolution and the lack of an absolute measurement scale.
The conversion reconstruction efficiency could be improved by retuning the track reconstruction parameters and 
by looking for possible constraints provided by the second photon from the pion decay 
(which may have reached the calorimeter or converted in a different part of the tracker).
The poor resolution can be improved by accounting for hits possibly shared by the two electron legs and properly take into account by means of 
unfolding techniques.
An absolute scale of the material estimation could be provided by the knowledge of the photon flux or by gauging the measurement 
counting the number of conversions in a volume of known radiation length.

Final remark is that the validity of the proposed method is not necessarily limited to converted photons and could be applied to
nuclear interactions as well.



%\newpage
\section*{References}

\begin{thebibliography}{99}
\bibitem{cms}
  Adolphi R et al. [CMS Collaboration] 2008 The CMS experiment at the CERN LHC
  {\em JINST} {\bf 3} S08004 
\bibitem{TkTDR} CMS Collaboration 1998 The Tracker System Project Technical Design Report
  {\em CERN-LHCC} 98-6
\bibitem{TkTDRadd} CMS Collaboration 2000 Addendum to the CMS Tracker TDR {\em
   CERN-LHCC} 2000-16
\bibitem{trackreco} CMS Collaboration 2010 Tracking and Vertexing Results from First Collisions
{\em CMS-PAS} TRK-10-001
\bibitem{cmssw} CMS Collaboration 2006 CMS Physics Technical Design
    Report Detector Performance and Software {\em
      CERN-LHCC} 2006-001
\bibitem{posterConv} Giordano D and Sguazzoni G [CMS
  Collaboration] 2012 An innovative seeding technique for photon conversion
  reconstruction at CMS {\em these proceedings}
\bibitem{pf1} CMS Collaboration 2009 Particle--Flow Event
  Reconstruction in CMS and Performance for Jets, Taus, and MET {\em CMS-PAS} PFT-09-001
\bibitem{pf2} CMS Collaboration 2010 Commissioning of the
  Particle-flow Event Reconstruction with the first LHC collisions
  recorded in the CMS detector {\em CMS-PAS} PFT-10-001
\bibitem{kdtree} Bentley J L 1975 Multidimensional binary search
    trees used for associative searching {\em Commun. ACM} {\bf 18} 9 (Sep. 1975) 509-517 
\bibitem{parallel} Hauth T, Innocente V and Piparo D [CMS
  Collaboration] 2012 Development and Evaluation of Vectorised and
    Multi-Core Event Reconstruction Algorithms within the CMS Software
    Framework {\em these proceedings}
\end{thebibliography}


\end{document}
